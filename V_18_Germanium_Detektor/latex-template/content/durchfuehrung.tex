\section{Versuchsaufbau}
In diesem versuch wird ein koaxialer Germanium-Detektor verwendet,
dessen Oberfläche mit Lithium-Atomen n-dotiert ist. Das heißt, es handelt 
sich um eine Zylinder mit einem Durchmesser von $\SI{45}{\milli\meter}$ und einer 
Länge von $\SI{39}{\milli\meter}$. 
im inneren des Detektors wurde eine koaxiale Bohrung vorenommen und mit Gold 
bedampft, um eine Verarmungszone zu erzeugen. 
Über dem Detektor eifnde sih eine Aluminumkappe, die die Gammastrahlung dazu zwingt, 
durch sie und durch die dotierte Lithiumoberfläche durzudringen um detektiert zu werden.
Dadurch kommt es zu einer unteren enregetischen nachweisgrenze bei 
$\SI{40}{\kilo\electronvolt}$ bis $\SI{50}{\kilo\electronvolt}$.
Um eine volle Energeinachweiswahrschienlihckeit zu erhlaten, sollten  Energien über 
$\SI{150}{\kilo\electronvolt}$ verwendet werden. 
Der Aufbau ist in Abbildung \ref{abb1} graphisch dargestellt \cite{sample}.

\begin{figure}
    \centering
    \includegraphics[width=\textwidth]{figure/Aufbau.pdf}
    \caption{Diese Abbildung veranschaulicht den Versuchsaufbau des Halbleiter-Germamium-Detektors \cite{sample}.}
    \label{abb1}
\end{figure}



\section{Durchführung}
\label{sec:Durchführung}

In diesem Versuch werden vier Messreihen durchgeführt. 
Zunächst soll eine Energiekalibrierung der Apparatur vorgenommen werden und eine 
Messung der Vollenergienachweiswahrschienlihckeit durchgeführt werden.
Dafür wird das Spektrum eines \ce{^152Eu}-Strahlers aufgenommen, sodass die Lage 
als auch der Inhalt der Linien bestimmt werden kann.
Danach wird das Spektrum eines \ce{^137Cs}-Strahlers aufgenommen, damit seine 
Aktivität sowie einige Detektoreigenschaften bestmmt werden können.
Für die Strahungquellen \ce{^125Sb} und \ce{^133Ba} wird ebenfalls die Aktivität 
bestimmmt.
Als letztes wird ein Anwendungsfall durchgespielt. 
Es wird das Spektrum eines unbekannten Strahlers aufgenommen und danch wird die 
Chemische bezeichnung und seine Aktivität bestimmt \cite{sample}.
