\section{Auswertung}
\label{sec:Auswertung}
Im Folgenden werden die einzelnen Messreihen ausgewertet. Dazu zählt die Kalibration des Detektors, die Bestimmung der Vollenergienachweiswahrscheinlichkeit,
die Untersuchung einer \ce{^137Cs} Probe und die Untersuchung von zwei unbekannten Proben.
\subsection{Kalibration des Detektors}
Um den Detektor zu kalibrieren, wird das Spektrum eines \ce{^152Eu}-Strahlers untersucht. Hierfür wird das Spektrum in Abhängigkeit des 
Kanals geplottet und die Peaks der Gammaquanten werden bestimmt. Da zwischen den Kanälen des Detektor und der Energie 
der gemessenen Gammaquanten ein linearer Zusammenhang besteht, kann eine lineare Ausgleichskurve bestimmt werden.
Das gemessene Spektrum ist in folgender Abbildung dargestellt.
\FloatBarrier
\begin{figure}
  \centering
  \includegraphics[width = \textwidth,keepaspectratio]{figure/Peaks_01.pdf}
  \caption{Spektrum des \ce{^152Eu} Strahlers für die Kalibration des Detektors. Abgeschnitten bei Kanal 4000 von 8192, da 
  in diesem Bereich der Untergrund dominiert.}
  \label{fig:Peaks_01}
\end{figure}
\FloatBarrier
Um die Peaks eindeutig bestimmen zu können, helfen charakteristische nahe beieinander liegende Peaks.
Die Literaturwerte der einzelnen Energien der Gammaquanten werden in Quelle \cite{Gamma_lit} dargestellt.
Die identifizierten Peaks sind in Tabelle \ref{tab:peaks_01} aufgelistet.
\FloatBarrier
\begin{table}
  \centering
  \caption{Identifizierte Peaks für die Kalibration des Detektors}
  \label{tab:peaks_01}
  \begin{tabular}{c c c c}
    \toprule
    Peaknummer&Kanal&Energie / $\SI{}{\kilo\eV}$&Intensität / \%\\
    \midrule
    0  &309   &$\num{121.8}$&$\num{28.41}$\\
    1   &614   &$\num{244.7}$&$\num{7.55}$\\
    2   &861   &$\num{344.3}$&$\num{26.59}$\\
    3   &1027  &$\num{411.1}$&$\num{2.24}$\\
    4   &1108  &$\num{444.0}$&$\num{3.12}$\\
    5   &1938  &$\num{778.9}$&$\num{12.97}$\\
    6   &2157  &$\num{867.4}$&$\num{4.24}$\\
    7   &2399  &$\num{964.1}$&$\num{14.5}$\\
    8   &2701  &$\num{1085.8}$&$\num{10.13}$\\
    9  &2765  &$\num{1112.1}$&$\num{13.41}$\\
    10  &3500  &$\num{1408.0}$&$\num{20.9}$\\
    \bottomrule
  \end{tabular}
\end{table}
Aus Tabelle \ref{tab:peaks_01} wird die Energie gegen den Kanal geplottet und eine lineare Funktion an die Daten 
angepasst.
\begin{equation*}
  \label{eq:linear}
  E(k) = m\cdot k + b
\end{equation*}
\FloatBarrier
\begin{figure}
  \centering
  \includegraphics[width=\textwidth,keepaspectratio]{figure/Lin_Fit_01.pdf}
  \caption{Messdaten und Fitfunktion für die Kalibration des Detektors}
  \label{fig:Lin_Fit:01}
\end{figure}
\FloatBarrier
Die Fitparameter der Funktion aus \ref{fig:Lin_Fit:01} sind:
\begin{equation*}
  m = \SI{0.40312(9)}{\kilo\eV\per Kanal} \quad b = \SI{-2.72(18)}{\kilo\eV}
\end{equation*}
Dadurch kann das Spektrum auch gegen die Energie aufgetragen werden.

\subsection{Vollenergienachweiswahrscheinlichkeit}
\label{cap:Vollenergienachweiswahrscheinlichkeit}

Die \ce{^152Eu} Probe hatte am 01.10.2000 eine Aktivität von $\SI{4130(60)}{\becquerel}$. Mit einer 
Halbwertszeit von $T_{1/2} = \SI{13.52}{yr}$ kann über das Zerfallsgesetz die Aktivität 
am Messtag bestimmt werden, diese liegt bei $\SI{1552(23)}{\becquerel}$.
Um die Vollenergienachweiswahrscheinlichkeit $\eta(E)$ bestimmen zu können, wird folgende Formel benötigt.
\begin{equation}
  \label{eq:eta_N}
  N = tA_{\text{Messtag}}I\frac{\Omega}{4\pi}\eta(E)
\end{equation}
Hierbei ist $t=\SI{4109}{\second}$ die Messzeit, $I$ die Intensität  des Peaks und $\Omega$ der vom Detektor abgedeckte
Raumwinkel.
Da die Probe als punktförmig angesehen werden kann, kann $\Omega$ mithilfe der Formel \eqref{eq:Omega} bestimmt werden.
\begin{equation}
  \label{eq:Omega}
  \Omega = \int_{0}^{2\pi}\int_{0}^{\frac{\theta}{2}} \sin\left(\theta'\right)\symup{d}\theta'\symup{d}\varphi = 
  2\pi \left(1-\cos\left(\frac{\theta}{2}\right)\right)
\end{equation}
Mit der Winkelbeziehung 
\begin{equation*}
  \tan\left(\frac{\theta}{2}\right) = \frac{r}{d} \implies \frac{\theta}{2} = \arctan\left(\frac{r}{d}\right)
\end{equation*}
und den Werten $d = \SI{8}{\centi\meter}+\SI{1.5}{\centi\meter}=\SI{9.5}{\centi\meter}$ und $r = \SI{2.25}{\centi\meter}$ kann $\Omega$ bestimmt werden.
Der Abstand $d$ ergibt sich aus dem Abstand der Probe und der Schutzhaube und dem Abstand von Schutzhaube und Detektor.
\begin{equation*}
  \Omega = 2\pi \left(1-\cos\left(\frac{\theta}{2}\right)\right) = 2\pi\left(1- \frac{1}{\sqrt{\left(\frac{r}{d}\right)^2+1}}\right) \approx \num{0.17}
\end{equation*}
Um den Linieninhalt $N$ zu bestimmen, wird eine Gaußglocke \eqref{eq:Gauß} an die Daten angepasst.
\begin{equation}
  \label{eq:Gauß}
  n(k)= A_0 \symup{e}^{-c(k-k_0)^2} + b
\end{equation}
Hierbei ist $k_0$ kein Fitparameter sondern der Kanal des zu fittenden Peaks.
Die Fits sind in Abbildung \ref{fig:subplots_01} zu sehen.
\FloatBarrier
\begin{figure}
  \centering
  \caption{Ausgleichskurven für die Bestimmung des Linieninhaltes $N$.}
  \label{fig:subplots_01}
  \includegraphics[width=\textwidth,keepaspectratio]{figure/Subplot_01.pdf}
\end{figure}
\FloatBarrier
Die Fitparameter sind in Tabelle \ref{tab:fit_params} aufgelistet. Mit der Gleichung 
\begin{equation*}
  N = A_0\sqrt{\frac{\pi}{c}}
\end{equation*}
kann der Linieninhalt bestimmt werden. Bei dem Peak der Nummer 3 ist der Fitparameter $c$ negativ und wird daher verworfen.
\FloatBarrier
\begin{table}
  \centering
  \caption{Fitparameter und Linieninhalte der einzelnen Peaks.}
  \label{tab:fit_params}
  \begin{tabular}{c c c c c}
    \toprule
    Peaknummer&$A_0$&$c$&$b$&$N$\\
    \midrule
    0   &$\num{3.45(7)e+03} $&$\num{0.277(14)}$ &$\num{105(16)}$&$\num{1.16(4)e+04}$\\
    1   &$\num{530(13)}     $&$\num{0.235(13)}$ &$\num{45(3)}$&$\num{1.94(7)e+03}$\\  
    2   &$\num{1121(30)}    $&$\num{0.175(11)}$ &$\num{24(7)}$&$\num{4.75(20)e+03}$\\
    4   &$\num{95(3)}     $&$\num{0.153(12)}$ &$\num{16(1)}$&$\num{430(22)}$\\
    5   &$\num{140(6)}      $&$\num{0.066(6)}$  &$\num{12(2)}$&$\num{9.7(6)e+02}$\\
    6   &$\num{39(4)}       $&$\num{0.047(11)}$ &$\num{11(1)}$&$\num{3.2(5)e+02}$\\
    7   &$\num{117(6)}      $&$\num{0.054(6)}$  &$\num{8(2)}$&$\num{8.9(7)e+02}$\\
    8   &$\num{65(4)}       $&$\num{0.033(5)}$  &$\num{7(2)}$&$\num{6.3(6)e+02}$\\
    9   &$\num{88(3)}       $&$\num{0.0430(30)}$&$\num{6(1)}$&$\num{7.5(4)e+02}$\\
    10  &$\num{90(2)}       $&$\num{0.0360(20)}$&$\num{2(1)}$&$\num{841(30)}$\\
    \bottomrule
  \end{tabular}
\end{table}
\FloatBarrier
Mit der Gleichung \eqref{eq:eta_N} können die $\eta$-Werte bestimmt werden, diese sind in Tabelle \ref{tab:eta_Werte} aufgelistet.
\FloatBarrier
\begin{table}
  \centering
  \caption{$\eta$-Werte bestimmt aus den Daten der Tabelle \ref{tab:fit_params}.}
  \label{tab:eta_Werte}
  \begin{tabular}{c c}
    \toprule
    Peaknummer& $\eta$\\
    \midrule
    0   &$\num{0.476(7)}$\\
    1   &$\num{0.299(4)}$\\
    2   &$\num{0.2081(30)}$\\
    4   &$\num{0.1601(23)}$\\
    5   &$\num{0.0863(13)}$\\
    6   &$\num{0.0865(13)}$\\
    7   &$\num{0.0719(10)}$\\
    8   &$\num{0.0728(11)}$\\
    9   &$\num{0.0656(10)}$\\
    10  &$\num{0.0472(7)}$\\
    \bottomrule
  \end{tabular}
\end{table}
An die Werte aus Tabelle \ref{tab:eta_Werte} und dazugehörigen Energien aus Tabelle \ref{tab:peaks_01}, wird die Funktion
\begin{equation*}
  \eta(E) = A\cdot \left(\frac{E}{\SI{1}{\kilo\eV}}\right)^z
\end{equation*}
angepasst. 
Die Fitparameter für den Fit aus Abbildung \ref{fig:Vollenergienachweiswahrscheinlichkeit} sind 
\begin{equation*}
  A = \num{80(9)} \quad z = \num{-1.02(2)}.
\end{equation*}
\begin{figure}
  \centering
  \caption{Messwerte und Fitfunktion für die Bestimmung der Vollenergienachweiswahrscheinlichkeit. Der schwarz markierte Messwert wird nicht für die Bestimmung der Fitfunktion verwendet.}
  \label{fig:Vollenergienachweiswahrscheinlichkeit}
  \includegraphics[width = \textwidth, keepaspectratio]{figure/Vollenergienachweiswahrscheinlichkeit.pdf}
\end{figure}
\FloatBarrier
\subsection{Untersuchung der \ce{^137Cs} Probe}
Das gemessene Spektrum ist in Abbildung \ref{fig:Cs137} dargestellt.
\FloatBarrier
\begin{figure}
  \centering
  \caption{Spektrum der \ce{^137Cs} Probe, hierbei wird der Kanal direkt in eine Energie umgerechnet. Das Spektrum wird nur bis zur einer Energie von $\SI{1000}{\kilo\eV}$ angegeben, da bei höheren Energien der Untergrund dominiert.}
  \label{fig:Cs137}
  \includegraphics[width= \textwidth,keepaspectratio]{figure/02_peaks.pdf}
\end{figure}
\begin{figure}
\centering
\caption{Spektrum der \ce{^137Cs} Probe im Bereich des Comptonkontinuums, hierbei wird der Kanal direkt in eine Energie umgerechnet.}
\label{fig:Cs137_log}
\includegraphics[width= \textwidth,keepaspectratio]{figure/02_peaks_Log.pdf}
\end{figure}

\FloatBarrier
Der Peak des Gammaquants liegt bei $E_{\gamma}=\SI{661.6(2)}{\kilo\eV}$. Mit den Gleichungen aus \eqref{eq:Comp} kann 
die Comptonkante und die Rückstreuenergie bestimmt werden.
\begin{equation*}
  \epsilon = \frac{E_{\gamma}}{m_{\symup{e}}c^2}
\end{equation*}
\begin{align}
  T_{\text{max}_{\text{theo}}} &= E_{\gamma}\frac{2\epsilon}{1+2\epsilon} \quad &E_{\text{rückstreu}_{\text{theo}}} =&{\gamma}\frac{1}{1+2\epsilon} \label{eq:Comp}\\
  T_{\text{max}_{\text{theo}}} &= \SI{477.3(2)}{\kilo\eV} \quad &E_{\text{rückstreu}_{\text{theo}}} =& \SI{184.32(1)}{\kilo\eV}\\
  T_{\text{max}_{\text{exp}}} &= \SI{470(6)}{\kilo\eV} \quad & E_{\text{rückstreu}_{\text{exp}}} =& \SI{190(6)}{\kilo\eV}
\end{align}
Die experimentell bestimmten Werte können aus Abbildung \ref{fig:Cs137_log} abgelesen und eine Unsicherheit 
kann abgeschätzt werden.
Um die Halbwertsbreite und die Zehntelwertsbreite zu bestimmen werden die Messwerte ermittelt, die am nächsten an der 
Hälfte beziehungsweise an dem Zehntel des Maximalwertes liegen.
\begin{align*}
  \text{Treffer}_{\text{Suchwert}}&=824\\
  \text{Treffer}_{\text{min:gefunden}} &= 764 \rightarrow E_{\text{min}} = \SI{660.41(23)}{\kilo\eV}\\
  \text{Treffer}_\text{{max:gefunden}} &= 1000 \rightarrow E_{\text{max}} = \SI{662.43(23)}{\kilo\eV}\\
  \Gamma_{1/2} &= \SI{2.0156(5)}{\kilo\eV}
\end{align*}
\begin{align*}
  \text{Treffer}_{\text{Suchwert}}&=\num{164.8}\\
  \text{Treffer}_{\text{min:gefunden}} &= 133 \rightarrow E_{\text{min}} = \SI{659.20(23)}{\kilo\eV}\\
  \text{Treffer}_{\text{max:gefunden}} &= 90  \rightarrow E_{\text{max}} = \SI{663.64(23)}{\kilo\eV}\\
  \Gamma_{1/10} &= \SI{4.4343(10)}{\kilo\eV}
\end{align*}
\begin{equation*}
  \frac{\Gamma_{1/10}}{\Gamma_{1/2}}=\num{2.2}
\end{equation*}
Die Unsicherheit des Verhältnisses ist nicht vorhanden, da beide Unsicherheiten der Breiten in der selben Größenordnung sind 
und durch die Division sich gegenseitig eliminieren. Dazu muss gesagt werden, dass die Unsicherheiten von $E_{\text{min}}$ und
$E_{\text{max}}$ nur durch die lineare Ausgleichsgraden kommen, da aber schon die Treffer stark von dem gesuchten Wert abweichen müssten die 
Unsicherheiten einiges größer sein.
Bei einer Gaußglocke ist das Verhältnis von $\Gamma_{1/2}$ und $\Gamma_{1/10}$ 
\begin{equation*}
  \frac{\Gamma_{1/10}}{\Gamma_{1/2}}= \sqrt{\frac{\ln{10}}{\ln{2}}} \approx \num{1.82}.
\end{equation*}
Daher kann der Peak als Gaußglocke \eqref{eq:Gauß} approximiert werden. 
Die Fitparameter aus der Abbildung \ref{fig:02_fit} sind 
\begin{equation*}
  A_0 = \num{1618(30)}\quad c = \num{0.58(2)} \quad b = \num{8(9)}
\end{equation*}
\FloatBarrier
\begin{figure}
  \centering
  \caption{Messdaten um den Peak des Gammaquants und Gaußglockenfit für die Bestimmung des Linieninhalt des Peaks. Zusätzlich sind die 
  Linien für die Bestimmung der Halbwertsbreite und Zehntelwertsbreite mit den Messdaten und des Fits eingezeichnet.}
  \label{fig:02_fit}
  \includegraphics[width=\textwidth,keepaspectratio]{figure/02_peak_fit.pdf}
\end{figure}
\FloatBarrier
Die beiden Breiten werden bestimmt, indem die Kurve um die Hälfte und um ein Zehntel des maximalen Wertes verringert wird und auf Nullstellen 
untersucht wird.
Die Differenzen der beiden Nullstellen sind die Breiten.
\begin{equation*}
  \Gamma_{1/2} = \SI{2.17(6)}{\kilo\eV}\quad \Gamma_{1/10} = \SI{4.0(1)}{\kilo\eV}
\end{equation*}
Das Verhältnis der beiden Breiten ist $\num{1.85(2)}$. 
Mit der Fitkurve aus Abbildung \ref{fig:02_fit} kann der Linieninhalt bestimmt werden. Hierbei ist darauf zu achten, dass der Fit ein zweites Mal gemacht wird 
und hierbei nicht die Energie sondern der Kanal verwendet wird. Damit lässt sich der Lineninhalt des Peaks auf $N=\num{9.33(28)e3}$ bestimmen.
Um den Inhalt des Comptonkontinuums zu bestimmen wird eine Ausgleichskurve der Gleichung \ref{2.12} berechnet.
Hierbei wird nur der Ausdruck vor der Klammer an den grün markierten Bereich aus \ref{fig:Cs137} angepasst.
Der bestimmte Parameter lautet $B=\SI{20.5(4)}{\barn\per Kanal}$. Um den Inhalt der Funktion \ref{2.12} zu erhalten,
wird diese,  mit der Bibliothek scipy \cite{scipy}, integriert. Der Inhalt ist damit $N_{\text{Comp}} = \num{2.85(5)e4}$.
Das Verhältnis der beiden Inhalte ist $N_{\text{Comp}}/N=\num{3.05(10)}$.
Mit der Energie des Gammaquants $E_{\gamma}=\SI{661.6(2)}{\kilo\eV}$ und der Hilfe der Gleichungen \eqref{eq1} und 
\eqref{eq2} kann der Wirkungsquerschnitt des Compoton-Effekts und des Photoeffekts bestimmt werden.
Diese liegen bei 
\begin{equation*}
  \sigma_{\text{Comp}}=\SI{8.196(1)}{b} \quad \sigma_{\text{Photo}}=\SI{0.05674(6)}{b}.
\end{equation*} 
Mit der Annahme, dass die Dichte von Germanium im Detektor bei $\rho = \SI{5.323}{\gram\per\cubic\centi\meter}$ \cite{Germanium_rho} liegt, und der
Gleichung \eqref{2.3} kann der Extinktionskoeffizient bestimmt werden.
Diese liegen bei 
\begin{equation*}
  \mu_{\text{Comp}} = \SI{0.35514(6)}{cm^{-1}} \quad \mu_{\text{Photo}} = \SI{0.002459(2)}{cm^{-1}}.
\end{equation*}
Mit diesen Werten und der Gleichung \eqref{2.2} kann die Wechselwirkungswahrscheinlichkeit bestimmt werden.
Hierbei ist $l=\SI{3.9}{\centi\meter}$ die Länge des Germanium-Kristalls.
Die Wechselwirkungswahrscheinlichkeiten sind 
\begin{equation*}
  P_{\text{Comp}} = \num{0.74968(5)}\quad P_{\text{Photo}} = \num{0.00954(1)}.
\end{equation*}
Das Verhältnis der beiden Größen ist
\begin{equation*}
  \frac{P_{\text{Comp}} }{P_{\text{Photo}}} = \num{78.56(8)}.
\end{equation*}

\subsection{Aktivitätsbestimmung}
Bei diesem Versuchsteil wird zunächst, anhand der Gamma-Linien, das untersuchte Material bestimmt.
Möglich sind \ce{^125Sb} oder \ce{^133Ba}.
Die verwendeten Gamma-Peaks der beiden potentiellen Proben sind in Tabelle \ref{tab:Ba_133Sb_125} aufgelistet.
\FloatBarrier
\begin{table}
  \centering
  \caption{Energie und Intensität der beiden potentiellen Proben \cite{Gamma_lit}.}
  \label{tab:Ba_133Sb_125}
  \begin{tabular}{c c c c c}
    \toprule
    & \multicolumn{2}{c}{\ce{^125Sb}} & \multicolumn{2}{c}{\ce{^133Ba}} \\
    \cmidrule(lr){2-3}\cmidrule(lr){4-5}
    Peaknummer & E / \SI{}{\kilo\eV} & Intensität / \%& E / \SI{}{\kilo\eV} & Intensität / \%\\
    \midrule
    0&$\num{35.489}$&$\num{5.79}$&\textcolor{orange}{$\num{53.1622}$}  &\textcolor{orange}{$\num{2.14}$}\\
    1&$\num{176.314}$&$\num{6.82}$&\textcolor{orange}{$\num{79.6142}$}  &\textcolor{orange}{$\num{2.63}$}\\
    2&$\num{380.452}$&$\num{1.52}$&$\num{80.9979}$  &$\num{33.31}$\\
    3&$\num{427.874}$&$\num{29.55}$&$\num{160.6121}$ &$\num{0.638}$\\
    4&$\num{463.365}$&$\num{10.48}$ &\textcolor{orange}{$\num{223.2368}$} &\textcolor{orange}{$\num{0.45}$}\\
    5&$\num{600.597}$&$\num{17.76}$  &$\num{276.3989}$ &$\num{7.13}$\\
    6&$\num{606.713}$&$\num{5.02}$ &$\num{302.8508}$ &$\num{18.31}$\\
    7&$\num{635.95}$&$\num{11.32}$&$\num{356.0129}$ &$\num{62.05}$\\
    8&$\num{671.441}$&$\num{1.783}$&$\num{383.8485}$ &$\num{8.94}$\\
    \bottomrule
  \end{tabular}
\end{table}
\FloatBarrier
Um die Probe zu identifizieren werden die Peaks im gemessenen Spektrum markiert, dies ist in Abbildung \ref{fig:03_peaks} zu sehen.
\FloatBarrier
\begin{figure}
  \centering
  \includegraphics[width = \textwidth, keepaspectratio]{figure/03_peaks.pdf}
  \caption{Aufgenommenes Spektrum mit den Literaturwerten der Gamma-Peaks, hierbei konnten alle Peaks von \ce{^133Ba} (außer die orange markierten) aus Tabelle \ref{tab:Ba_133Sb_125} identifiziert werden.}
  \label{fig:03_peaks}
\end{figure}
\FloatBarrier
Um die Aktivität zu bestimmen, wird die Gleichung \eqref{eq:eta_N} verwendet. Hierbei ist die Messzeit $t=\SI{3770}{\second}$, 
$\Omega$ ist wie in vorherigen Teilen bestimmt worden und die Intensitäten werden aus Tabelle \ref{tab:Ba_133Sb_125} abgelesen.
Die Funktion für die Vollenergienachweiswahrscheinlichkeit wurde bereits in Kapitel \ref{cap:Vollenergienachweiswahrscheinlichkeit}
bestimmt.
Der Linieninhalt wird wie in den vorherigen Kapiteln bestimmt.
Die Fits für die Bestimmung des Linieninhalts, sowie die Linieninhalte selbst, sind in Tabelle \ref{tab:params_Linieninhalt} aufgelistet.
\FloatBarrier
\begin{table}
  \centering
  \caption{Fitparameter für die Bestimmung der Linieninhalte}
  \label{tab:params_Linieninhalt}
  \begin{tabular}{c c c c c}
    \toprule
    Peaknummer&$A_0$&$c$&$b$&$N$\\
    \midrule
    2&$\num{2.32(10)e+03}$&$\num{0.311(31)}$   &$\num{102(20)}$&$\num{7.4(5)e+3}$\\
    3&$\num{0.0(34)e+08}$   &$\num{-0.0000(29)}$&$\num{0(34)e+08}$&$---$\\
    5&$\num{286(15)}$     &$\num{0.226(27)}$  &$\num{21.1(33)}$&$\num{1.07(8)e+3}$\\
    6&$\num{682(10)}$     &$\num{0.218(8)}$   &$\num{16.8(24)}$&$\num{2.59(6)e+3}$\\
    7&$\num{1769(9)}$     &$\num{0.1834(22)}$ &$\num{9.7(21)}$&$\num{7.32(6)e+3}$\\
    8&$\num{230(5)}$      &$\num{0.186(9)}$   &$\num{6.3(12)}$&$\num{948(31)}$\\
    \bottomrule
  \end{tabular}
\end{table}
\FloatBarrier
Die Fits für die Bestimmung der Lineninhalte sind in Abbildung \ref{fig:Linieninhalt_03} dargestellt.
\FloatBarrier
\begin{figure}
  \centering
  \includegraphics[width=\textwidth,keepaspectratio]{figure/03_subplot.pdf}
  \caption{Fits für die Bestimmung der Linieninhalte}
  \label{fig:Linieninhalt_03}
\end{figure}
Mit den berechneten Linieninhalten kann mit der Gleichung \eqref{eq:eta_N} die Aktivität am Messtag bestimmt werden. 
Diese sind in Tabelle \ref{tab:Akti_03} aufgelistet.
\FloatBarrier
\begin{table}
  \centering
  \caption{Aktivität der \ce{^133Ba} Probe, bestimmt über die einzelnen Peaks und gemittelt.}
  \label{tab:Akti_03}
  \begin{tabular}{c c}
    \toprule
    Peaknummer & Aktivität / $\SI{}{\becquerel}$\\
    \midrule
    2&$\num{4.8(7)e+2}$\\
    5&$\num{1.12(19)e+3}$\\
    6&$\num{1.16(20)e+3}$\\
    7&$\num{1.14(19)e+3}$\\
    8&$\num{1.11(19)e+3}$\\
    \midrule
    mittel& $\num{1.00(17)e+03}$\\
    \bottomrule
  \end{tabular}
\end{table}
\subsection{Bestimmung einer unbekannten Probe}
Bei diesem Versuchsteil wird nur  das Material der Probe bestimmt, da aufgrund der Ausdehnung der Probe diese nicht als punktförmig 
angesehen werden kann und damit $\Omega$ nicht berechnet werden kann, was eine notwendige Größe für die Bestimmung der Aktivität ist.
Das Spektrum ist in Abbildung \ref{fig:Peaks_04} zu sehen.
Von den gefundenen Peaks können 16 Peaks identifiziert werden. Diese sind in Tabelle \ref{tab:ident_Peaks} 
aufgelistet.
\FloatBarrier
\begin{table}
  \centering
  \caption{Energien der identifizierten Peaks aus dem Spektrum und die dazugehörigen Literaturwerte \cite{Gamma_lit}.}
  \label{tab:ident_Peaks}
  \begin{tabular}{c c c c}
    \midrule
    Isotop& Energie (gemessen) / $\SI{}{\kilo\eV}$&  Energie (Lit.) / $\SI{}{\kilo\eV}$& Intensität (Lit.) / \%\\
    \midrule
    234Th&$\num{92.8(1)}$&$\num{92.8}$&$\num{3.75}$\\
    (235U)&$\num{185.94(18)}$&$\num{185.72}$&$\num{57.0}$\\
    226Ra&$\num{185.94(18)}$&$\num{186.21}$&$\num{3.6}$\\
    214Pb&$\num{241.97(19)}$&$\num{242.00}$&$\num{7.3}$\\
    214Pb&$\num{295.19(19)}$&$\num{295.22}$&$\num{18.4}$\\
    214Pb&$\num{352.03(20)}$&$\num{351.93}$&$\num{35.6}$\\
    214Bi&$\num{609.22(23)}$&$\num{609.31}$&$\num{45.5}$\\
    (214Rn)&$\num{609.22(23)}$&$\num{609.31}$&$\num{0.1}$\\
    214Bi&$\num{665.65(25)}$&$\num{665.45}$&$\num{1.5}$\\
    214Bi&$\num{768.05(25)}$&$\num{768.36}$&$\num{4.9}$\\
    214Pb&$\num{786.19(25)}$&$\num{785.96}$&$\num{1.1}$\\
    234Pa&$\num{805.5(2)}$&$\num{805.8}$&$\num{2.5}$\\
    214Bi&$\num{806.34(26)}$&$\num{806.17}$&$\num{1.3}$\\
    214Bi&$\num{934.94(28)}$&$\num{934.06}$&$\num{3.1}$\\
    214Bi&$\num{1120.78(31)}$&$\num{1120.29}$&$\num{14.9}$\\
    214Bi&$\num{1238.89(33)}$&$\num{1238.11}$&$\num{5.8}$\\
    214Bi&$\num{1377.6(4)}$&$\num{1377.7}$&$\num{4.0}$\\
    214Bi&$\num{1764.6(4)}$&$\num{1764.49}$&$\num{15.31}$\\
    \bottomrule
  \end{tabular}
\end{table}
\begin{figure}
  \centering
  \caption{gemessenes Spektrum einer unbekannten Probe.}
  \label{fig:Peaks_04}
  \includegraphics[width=\textwidth,keepaspectratio]{figure/04_peaks.pdf}
\end{figure}
\FloatBarrier
Die Isotope aus Tabelle \ref{tab:ident_Peaks} gehören alle zu der natürlichen Zerfallsreihe des Uran Isotop Uran-238, 
außer der Peak von Uran-235 bei einer Energie von $\SI{185.94(18)}{\kilo\eV}$.
Dieser Peak kann allerdings Radium-226 zugeordnet werden, welches ebenfalls zur Zerfallsreihe von Uran-238 gehört.








