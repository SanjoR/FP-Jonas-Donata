\section{Diskussion}
\label{sec:Diskussion}
Im Folgenden werden die Ergebnisse der verschiedenen Messreihen untersucht.

\subsection{Kalibration des Detektors und Vollenergienachweiswahrscheinlichkeit}
Der Detektor wird kalibriert, indem das Spektrum eines Europium-152-Strahlers untersucht wird.
Hierbei werden die Kanalnummern der Peaks an die Literaturwerte \cite{Gamma_lit} der Gamma-Linien mithilfe einer linearen 
Ausgleichsrechnung angepasst. Dass die Kalibration erfolgreich ist, ist daran zu sehen, dass die Peaks der weiteren
Proben mit den Literaturwerten übereinstimmen.
Die Vollenergienachweiswahrscheinlichkeit ist auch erfolgreich, da eine Potenzfunktion gut an die gemessenen Daten angepasst werden konnte und die 
damit bestimmten Aktivitäten in physikalisch sinnvollen Bereichen liegen. 

\subsection{Untersuchung der Cäsium-137 Probe}
Der gemessene Photopeak der Cäsium-137 Probe liegt bei $\SI{661.6(2)}{\kilo\eV}$ der Literaturwert beträgt $\SI{661.657(3)}{\kilo\eV}$ \cite{Gamma_lit},
daher ist der Literaturwert in einer $1\sigma$ Umgebung des Messwertes. Das bestätigt den gemessenen Wert und die Kalibration des 
Detektors. Die Comptonkante und die Rückstreuenergie weichen jeweils um circa $\SI{5}{\kilo\eV}$ vom berechneten Wert ab, allerdings
kann kein exakter Wert bestimmt werden, da die beiden Werte verrauscht sind. Wenn das gesamte Spektrum angesehen wird, dann ist allerdings 
zu sehen, dass die gemessenen und berechneten Werte im gleichen Bereich liegen.
Der Photopeak lässt sich mit einer Gaußglocke beschreiben. Das wird durch das Verhältnis der Halb- und Zehntelwertsbreite 
bestätigt. 
Die Messwerte zeigen, dass der Comptoneffekt bei dieser Energie über den Photoeffekt dominiert, das ist auch zu erwarten und bestätigt 
damit die Theorie.

\subsection{Aktivitätsbestimmung}
Bei diesem Versuchsteil wird zunächst geguckt ob das untersuchte Material eine Antimon-125 oder eine Barium-133 Probe ist.
Bei der Betrachtung des Spektrums ist zu erkennen, dass es sich um eine Barium-133 Probe handelt.
Hierfür konnten sechs Peaks identifiziert werden und von denen sind fünf geeignet um die Aktivität zu bestimmen.
Bei vier der fünf Peaks liegt die berechnete Aktivität in einer $1\sigma$ Umgebung der anderen Aktivitäten. 
Das bestätigt die gemessene Aktivität, allerdings muss darauf geachtet werden, dass die Probe als punktförmig angesehen wird
und dadurch die Aktivität der Probe etwas geringer ist.

\subsection{Bestimmung einer unbekannten Probe}
Bei diesem Versuchsteil wird eine unbekannte Probe untersucht. Aufgrund der Ausdehnung der Probe und da diese direkt auf dem Detektor
lag, kann hier keine Aktivität bestimmt werden.
Um dem Material eine Zerfallskette zuordnen zu können, konnten 16 Peaks identifiziert werden. Alle Stoffe, welche identifiziert 
werden konnten, können der natürlichen Zerfallskette von Uran-238 zugeordnet werden. Ein Peak kann entweder der Zerfallskette 
oder Uran-235 zugeordnet werden. Dieses Isotop ist ein natürlicher Bestandteil von Uranerz. 