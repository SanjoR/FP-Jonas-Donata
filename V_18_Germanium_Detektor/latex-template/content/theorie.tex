\section{Motivation}
Damit ein Germaniumdetektor die Spektren verschiedener Strahler bezüglich deren
Energie und Aktivität auswerten kann, muss zunächst eine Energiekalibrierung 
durchgeführt und
eine Vollenergienachweiswahrscheinlichkkeit bestimmt werden. 
Das ist das Ziel dieses Versuches

\section{Theorie}
\label{sec:Theorie}

\subsection{Gammastrahlung}
Im Gegensatz zu Alpha- und Betastrahlung ist Gammastrahlung elektrisch neutral. 
Es handelt sich um hochenergetische elektromagnetische Strahlung.
Gammastrahlung wird vorallem beim Zerfall von Atomkernen radioaktiver Nuklide frei, 
wenn angeregte Atomkerne nach bspw. einem Alpha- oder Betazerfall wieder in ihren 
grundzustand zurückkehren.
Daher ist das Gammaspektrum ein diskretes Linienspektrum, das für jede Substanz 
spezifisch ist. Daher kann mit der Gammaspektroskopie eine unbekannte Substanz über 
ihr Linienspektrum erkannt werden.
Daher ist ihre Wechselwirkung mit Materie verhältnismäßig gering und sie 
kann tiefer in feste Substanzen eindringen.
Die Intänsität von geladener Strahlung hängt exponentiell von der 
Eindringtiefe ab. Bei Gammastrahlung ist die Halbwertsdicke hingegen 
von der Wellenlänge und der Ordnungszahl des Materials abhängig.
Gammastrahlung kann auf drei verschiedene Arten mit Materie wechelwirken.
Diese Arten werden Photoeffekt, Compton-Effekt und Paarbildung genannt.
Der Photoeffekt tritt in diesem Zusammenhang eher bei niedrigeren, der 
Compton-Effekt dominanater bei mittleren und die Paarbildung vorrangig
bei hochen Gammaenergien auf.

\subsection{Der Photoeffekt}
Unter dem Photoeffekt werden drei verschiedene Mechanismen zusammengefasst.
In diesem Fall ist der äußere Photoeffekt von Relevanz.
Beim äußeren Photoeffekt werden Elektronen aus einer Halbleiter- oder Metalloberfläche 
durch Strahleneinwirkung herausgelöst. 
Die einfallende Gammastrahlung muss dafür eine Energie aufweisen, die größer als die 
Bindungsenergie des Elektrons ist. Wenn das Gammaquant auf das Elektroon trifft wird es 
annihiliert, während das Elektron frei wird. Der zugehörige 
Wirkungsquerschnitt des Photoeffekts ist in Gleichung \ref{eq1} beschrieben.

\begin{equation}
    \sigma_{ph} = \frac{3}{2} \sigma_{Th} \alpha \frac{Z^5}{\epsilon^5} (\gamma^2 -1)^{\frac{3}{2}} \left(\frac{4}{3} + \gamma\frac{\gamma -2}{\gamma+1} \left(1 - \frac{1}{2 \gamma \sqrt{\gamma^2 -1}} \ln\left(\frac{\gamma + \sqrt{\gamma^2 - 1}}{\gamma - \sqrt{\gamma^2 -1}} \right) \right) \right)
    \label{eq1}
\end{equation}

In Gleichung \ref{eq1} entspricht $\epsilon$ der Ebergie des Photons, $\sigma_{Th}$
ist der Wirkungsquerschnitt der Thompson-Streuung, $\alpha$ ist die 
Sommerfeld´sche Feinstrukturkonstante und $\gamma$ entspricht dem relativistischen 
Gamma-Faktor des Elektrons nachdem es mit dem Photon wechselgewirkt hat.

\subsection{Der Compton-Effekt}
Wenn ein Photon, bspw. Gammastrahlung auf ein Teilchen, bspw. ein Elektron trifft
und an ihm gestreut wird, dann wird seine Wellenlänge vergrößert. 
Dieser Prozess wird Compton-Effekt genannt.
Im Gegensatz zum Photoeffekt wird beim Comptoneffekt nicht die 
gesamte Photonenenergie übertragen, sondern nur ein Teil, sodass das Elektron an 
Energie gewinnt und das Photon die seine verringert, sodass es eine längere 
Wellenlänge bekommt. Das führt dazu, dass die Energie die Elektronen nach dem 
Stoß einem kontinuierlichen Spektrum entspricht.
Der maximale Energieübertrag wird dabei durch einen Frontalstoß erzeugt.
Der Wirkungsquerschnitt des Comptoneffektes wird häufig über die 
Klein-Nishina Formel \ref{eq2} angegeben, die den Wirkungquerschnitt bezüglich 
eines Elektrons angibt. Der totale Wirkungsquerschnitt ergibt sich wenn diese 
Formel über den Raumwinkel integriert wird.

\begin{equation}
    \sigma_{com} = \frac{3}{4}\sigma_{TH} \left(\frac{1 + \epsilon}{\epsilon^2} \left(\frac{2(1+\epsilon)}{1+2\epsilon} - \frac{1}{\epsilon} \ln(1+2\epsilon) \right) + \frac{1}{2\epsilon} \ln(1+2\epsilon) - \frac{1+3\epsilon}{(1+2\epsilon)^2} \right)
    \label{eq2}
\end{equation}

Für ein Atom ist der Compton-Wirkungsquerschnitt entsprechent proportional zur 
Ordnungszahl $Z$, wie es in Gleichung \ref{eq3} dargestellt ist.

\begin{equation}
    \sigma^A_{com} = Z \sigma_{com}
\end{equation}

\subsection{Paarbildung}


\subsection{Halbleiterdetektoren}

\cite{sample}
