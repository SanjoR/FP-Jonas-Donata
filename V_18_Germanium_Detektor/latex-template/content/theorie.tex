\section{Motivation}
Damit ein Germaniumdetektor die Spektren verschiedener Strahler bezüglich deren
Energie und Aktivität auswerten kann, muss zunächst eine Energiekalibrierung 
durchgeführt und
eine Vollenergienachweiswahrscheinlichkeit bestimmt werden. 
Das ist das Ziel dieses Versuches.

\section{Theorie}
\label{sec:Theorie}

\subsection{Gammastrahlung}
Im Gegensatz zu Alpha- und Betastrahlung ist Gammastrahlung elektrisch neutral. 
Deshalb ist ihre Wechselwirkung mit Materie verhältnismäßig gering und sie 
kann tiefer als geladene Strahlung in feste Substanzen eindringen.
Es handelt sich um hochenergetische elektromagnetische Strahlung.
Gammastrahlung wird vorallem beim Zerfall von Atomkernen radioaktiver Nuklide frei, 
wenn angeregte Atomkerne nach bspw. einem Alpha- oder Betazerfall wieder in ihren 
Grundzustand zurückkehren.
Daher ist das Gammaspektrum ein diskretes Linienspektrum, das für jede Substanz 
spezifisch ist. Deshalb kann mit der Gammaspektroskopie eine unbekannte Substanz über 
ihr Linienspektrum ezugeordnet werden.
Die Intensität von geladener Strahlung hängt exponentiell von der 
Eindringtiefe ab. Bei Gammastrahlung ist die Halbwertsdicke hingegen 
von der Wellenlänge und der Ordnungszahl des Materials abhängig.
Die Anzahl $N(l)$ der Teilchen, die nach der Materialtiefe $l$ noch nicht mit dem 
Gammastrahl wechselgewirkt haben, ist gegeben durch:

\begin{equation}
    N(l) = N_0 e^{-l\mu}.
    \label{2.1}
\end{equation}

In Gleichung \ref{2.1} gilt $N_0 = N(0)$ und $\mu$ entspricht dem Extinktionskoeffizienten:

\begin{equation}
    \mu = \sigma n = \frac{\sigma \rho}{uA}.
    \label{2.3}
\end{equation}

In Gleichung \ref{2.3} beschreibt $n$ die Teilchendichte, die ungefähr der 
Dichte $\rho$, dividiert durch das Produkt aus dem Gewicht eines Nukleons $u$
und der atomaren Massenzahl $A$  entspricht.
In diesem Zusammenhang, ergibt sich schließlich die 
Wechselwirkungswahrscheinlichkeit:

\begin{equation}
    P(l) = 1 - e^{-l\mu}.
    \label{2.2}
\end{equation}

Gammastrahlung kann auf drei verschiedene Arten mit Materie wechselwirken.
Sie werden Photoeffekt, Compton-Effekt und Paarbildung genannt.
Der Photoeffekt tritt in diesem Zusammenhang eher bei niedrigeren, der 
Compton-Effekt dominanter bei mittleren und die Paarbildung vorrangig
bei hohen Gammaenergien auf.

\subsection{Der Photoeffekt}
Unter dem Photoeffekt werden drei verschiedene Mechanismen zusammengefasst.
In diesem Fall ist der äußere Photoeffekt von Relevanz.
Beim äußeren Photoeffekt werden Elektronen aus einer Halbleiter- oder Metalloberfläche 
durch Strahleneinwirkung herausgelöst. 
Er dominiert bei Energien bis zu $\SI{1}{\mega\eV}$
Die einfallende Gammastrahlung muss dafür eine Energie aufweisen, die größer als die 
Bindungsenergie des Elektrons ist. Wenn das Gammaquant auf das Elektron trifft, wird es 
annihiliert, während das Elektron frei wird. Der zugehörige 
Wirkungsquerschnitt des Photoeffekts ist in Gleichung \ref{eq1} beschrieben.

\begin{equation}
    \sigma_{\text{ph}} = \frac{3}{2} \sigma_{\text{Th}} \alpha \frac{Z^5}{\epsilon^5} (\gamma^2 -1)^{\frac{3}{2}} \left(\frac{4}{3} + \gamma\frac{\gamma -2}{\gamma+1} \left(1 - \frac{1}{2 \gamma \sqrt{\gamma^2 -1}} \ln\left(\frac{\gamma + \sqrt{\gamma^2 - 1}}{\gamma - \sqrt{\gamma^2 -1}} \right) \right) \right)
    \label{eq1}
\end{equation}

In Gleichung \ref{eq1} entspricht $\epsilon$ der Energie des Photons, $\sigma_{Th}$
ist der Wirkungsquerschnitt der Thompson-Streuung, $\alpha$ ist die 
Sommerfeld´sche Feinstrukturkonstante und $\gamma$ entspricht dem relativistischen 
Gamma-Faktor des Elektrons, nachdem es mit dem Photon wechselgewirkt hat.

\subsection{Der Compton-Effekt}
Wenn ein Photon bspw. Gammastrahlung, auf ein Teilchen bspw. ein Elektron trifft
und an ihm gestreut wird, dann wird seine Wellenlänge vergrößert. 
Dieser Prozess wird Compton-Effekt genannt.
Er tritt vorwiegend bei Photonenergien zwischen $\SI{1}{\mega\electronvolt}$ und 
$\SI{10}{\mega\electronvolt}$ auf.
Im Gegensatz zum Photoeffekt wird beim Compton-Effekt nicht die 
gesamte Photonenenergie übertragen, sondern nur ein Teil, sodass das Elektron an 
Energie gewinnt und das Photon seine Energie verringert, sodass es eine längere 
Wellenlänge bekommt. Das führt dazu, dass die Energie der Elektronen nach dem 
Stoß einem kontinuierlichen Spektrum entspricht.
Der maximale Energieübertrag wird dabei durch einen Frontalstoß erzeugt.
Der Wirkungsquerschnitt des Compton-Effektes wird häufig über die 
Klein-Nishina Formel \ref{eq2} angegeben, die den Wirkungsquerschnitt bezüglich 
eines Elektrons angibt. Der totale Wirkungsquerschnitt ergibt sich, wenn diese 
Formel über den Raumwinkel integriert wird.

\begin{equation}
    \sigma_{\text{com}} = \frac{3}{4}\sigma_{\text{Th}} \left(\frac{1 + \epsilon}{\epsilon^2} \left(\frac{2(1+\epsilon)}{1+2\epsilon} - \frac{1}{\epsilon} \ln(1+2\epsilon) \right) + \frac{1}{2\epsilon} \ln(1+2\epsilon) - \frac{1+3\epsilon}{(1+2\epsilon)^2} \right)
    \label{eq2}
\end{equation}

Für ein Atom ist der Compton-Wirkungsquerschnitt entsprechent proportional zur 
Ordnungszahl $Z$, wie es in Gleichung \ref{eq3} dargestellt ist.

\begin{equation}
    \sigma^A_{\text{com}} = Z \sigma_{\text{com}}
    \label{eq3}
\end{equation}

Alternativ kann der Compton-Wirkungsquerschnitt auch über die kinetische 
Energie $T$ ausgedrückt werden:

\begin{equation}
    \frac{\symup{d} \sigma_{\text{com}}}{\symup{d}T} = \frac{3\sigma_{\text{Th}}}{8 m_{\text{e}} c^2 \epsilon^2} \left(2+ \frac{t^2}{\epsilon^2(1-t)^2} + \frac{t(t-\frac{2}{\epsilon})}{1-t} \right) \cdot \Theta( \frac{2 \epsilon}{1+ 2 \epsilon}-t).
    \label{2.12}
\end{equation}

\subsection{Paarbildung}
Allgemein wird unter Paarbildung ein teilchenphysikalischer Prozess verstanden, bei dem 
ein Teilchen und sein Antiteilchen erzeugt werden. In diesem Versuch bezieht sich die 
Paarbildung jedoch nur auf die Erzeugung von Elektron-Positron-Paaren. 
Sie ist der überwiegende Wechselwikrungsprozess bei Photonen die eine Energie
über $\SI{10}{\mega\electronvolt}$ aufweisen.
Sie entsteht bei Wechselwirkung von Photonen mit Materie, genauer bei der 
Wechselwirkung eines Photons mit dem elektrischen Feld des Atomkerns oder der 
Hüllenelektronen. 
Bei ersterem wird die gesamte Photonenergie an die 
entstehenden Teilchen und deren kinetische Energie abgegeben. Außerdem erfährt 
der Atomkern aufgrund der Impulserhaltung einen Rückstoß.
Die Paarbildung kann über den 
Wirkungsquerschnitt, der in Gleichung \ref{eq3} aufgeführt ist, beschrieben werden.

\begin{equation}
    \sigma_{\text{paar}} = \alpha \frac{3 \sigma_{\text{Th}}}{2 \pi} \frac{7}{9} \left( \left( Z^2 (L_{\text{rad}} - f(Z)) + Z L'_{\text{rad}}) \right) + \frac{1}{42} \left(Z^2 + Z \right) \right)
    \label{eq3}
\end{equation}

Da es sich um geladene Teilchen handelt, müssen Bremsstrahlungseffekte berücksichtigt 
werden. Das wird über die $L_{\text{rad}}$ und $L'_{\text{rad}}$ Funktionen realisiert. 
$f(Z)$ ist dabei eine Funktion, die die Compton-Korrektur miteinbezieht.

\subsection{Halbleiterdetektoren}
Da Gammastrahlung ungeladen ist, kann sie über ihre Wechselwirkung mit Materie 
nachgewiesen werden. Wenn geladene Teilchen frei werden, kann bspw. eine 
Spannung gemessen werden. Dies ist die Grundlage eines Halbleiterdetektors.
Halbleiter definieren sich über ihre, für Elektronen überwindbare,
Bandlücke zwischen ihrem Valenz- und Leitungsband. Daher können sie durch die 
angeregten Elektronen durch Gammabestrahlung leitfähiger werden. 
Die erzeugten Elektron-Loch-Paare, die dabei entstehen, benötigen im Vergleich zu einer 
Ionisationskammer eine Anregungsenergie die um 10 Größenordnungen kleiner ist. 
Das führt zu einer höheren Aflösung von Halbleitern im Vergleich zu Ionisationskammern.
Für einen solchen Halbleiterdetektor wird häufig das Material Germanium verwendet.
Pro Messdurchlauf ergeben sich allerdings statistische Schwankungen der Anzahl an 
erzeugten Elektron-Loch-Paaren. Außerdem ergeben sich bei der Erzeugung Energieverluste,
da sich die Gammastrahlung statistisch auf die Atome des Materials verteilt.
Dies führt dazu, dass die Standardabweichung, die in Gleichung \ref{eq4} dragestellt ist,
um den Fano-Faktor $F$ ergänzt werden muss.

\begin{equation}
    \sigma = \sqrt{F \frac{E_{\gamma}}{E}}
    \label{eq4}
\end{equation}

Für Germanium hat der Fano-Faktor den Wert $\num{0,1}$.
Dieser Abschnitt wurde mittels der Literaturverzeichnisse \cite{lit1},
\cite{lit2}, \cite{lit3} und \cite{lit4} erzeugt.
