\section{Diskussion}

Die Magnetfeldstärke konnte an die Stromstärke mit einer linearen Ausgleichskurve angeglichen werden. 
Das bestätigt den linearen Zusammenhang zwischen Magnetfeldstärke und Stromstärke.

\subsection{Aufspaltung der roten Spektrallinie}

Bei der roten Spektrallinie wurde der $\sigma$-Übergang vermessen. Hierfür werden die $\Delta S$ und $\delta S$ 
Werte bestimmt. Da diese mit dem Programm Inkscape \cite{Inkscape} bestimmt wurden, sind keine Unsicherheiten vorhanden, allerdings 
müsste eine Unsicherheit vorhanden sein, da manche Linien sehr verschwommen sind. Die Verschiebung der Wellenlänge 
beträgt $\overline{\delta\lambda} = \SI{10.6(7)}{\pico\meter}$. Die Unsicherheit kommt nur durch die Standardabweichung des 
Mittelwerts zustande und müsste eigentlich höher sein.
Der Landé-Faktor des $\sigma$-Übergangs des roten Lichtes ist $g=1$. Experimentell wurde der Wert $g=\num{1.39(10)}$ 
bestimmt. Dieser weicht von der Theorie ab. Das kann durch die nicht einfach bestimmtbaren $\delta S$ Werte zustande kommen und 
dadurch, dass die Unsicherheiten unterschätzt werden.

\subsection{Aufspaltung der blauen Spektrallinie}

Die Vermessung des $\sigma$-Übergangs fällt schwer, da das Bild für die Bestimmung der $\delta S$ Werte unterbelichtet ist.
Auch hier wird, durch die fehlenden Unsicherheiten der $\Delta S$ und $\delta S$ Werte, die Unsicherheit der 
Wellenlängeverschiebung unterschätzt. Die Verschiebung liegt bei $\overline{\delta\lambda} = \SI{5.7(4)}{\pico\meter}$.
Der damit bestimmte Landé-Faktor liegt bei $g=\num{1.7(4)}$. Laut Theorie müssten zwei verschiedene Werte bestimmbar sein.
Da aber keine verschiedene Aufspaltung zu sehen sind, kann nur ein Wert bestimmt werden. Die Theoriewerte liegen bei 
$g=\num{1.5}$ und $g=2$. Der Mittelwert der beiden theoretischen Landé-Faktor beträgt $g=\num{1.75}$. der gemessene Wert weicht 
etwas mehr als eine $\sigma$-Umgebung davon ab.
Die Vermessung des $\pi$-Übergangs fällt aufgrund der besseren Belichtung leichter.
Hierbei wird für die Wellenlängeverschiebung der Wert $\overline{\delta\lambda} = \SI{7.8(7)}{\pico\meter}$ bestimmt. 
Auch hier wird die Unsicherheit aus den gleichen Gründen unterschätzt.
Der gemessene Landé-Faktor hat den Wert $g=\num{0.72(6)}$ und weicht daher auch von dem Theoriewerte von $g=\num{0.5}$ ab.
Insgesamt kann gesagt werden, dass die Werte alle etwas von den Theoriewerten abweichen. Diese Abweichung ist allerdings gering genug, dass bei einer 
höheren Unsicherheit die Werte die Theorie bestätigen würden. 