\section{Diskussion}
\label{sec:Diskussion}
Im Folgenden werden die Ergebnisse der verschiedenen Messreihen untersucht.

\subsection{Kalibration des Detektors und Vollenergienachweiswahrscheinlichkeit}
Der Detektor wird kalibriert, indem das Spektrum eines \ce{^152Eu}-Strahlers untersucht wird.
Hierbei werden die Kanalnummern der Peaks an die Literaturwerte \cite{Gamma_lit} der Gamma-Linien mithilfe einer linearen 
Ausgleichsrechnung angepasst. Dass die Kalibration erfolgreich ist, ist daran zu sehen, dass die Peaks der weiteren
Proben mit den Literaturwerten übereinstimmen.
Die Vollenergienachweiswahrscheinlichkeit ist auch erfolgreich, da eine Potenzfunktion gut an die gemessenen Daten angepasst werden konnte und die 
damit bestimmten Aktivitäten in physikalisch sinnvollen Bereichen liegen. 

\subsection{Untersuchung der \ce{^137Cs} Probe}
Der gemessene Peak des Gammaquants der \ce{^137Cs} Probe liegt bei $\SI{661.62(23)}{\kilo\eV}$ der Literaturwert beträgt $\SI{661.6570(30)}{\kilo\eV}$ \cite{Gamma_lit},
daher ist der Literaturwert in einer $1\sigma$ Umgebung des Messwertes. Das bestätigt den gemessenen Wert und die Kalibration des 
Detektors.
Die gemessene Comptonkante und die Rückstreuenergie weichen um etwas mehr als eine $\sigma$-Umgebung von den berechneten Werten ab, da die 
Unsicherheiten selbst gewählt werden mussten, sind die Unsicherheiten weniger aussagekräftig.
Wenn das gesamte Spektrum angesehen wird, dann ist allerdings 
zu sehen, dass die gemessenen und berechneten Werte im gleichen Bereich liegen.
Der Peak des Gammaquants lässt sich mit einer Gaußglocke beschreiben. Das wird durch das Verhältnis der Halb- und Zehntelwertsbreite 
bestätigt. 
Die Messwerte zeigen, dass der Comptoneffekt circa 3 mal öfter auftritt als der Photoeffekt. Die Theorie ergibt allerdings 
ein Wert von circa 80 mal öfter an. Diese Diskrepanz bedeutet, dass entweder Das Comptonkontinuum zu schwach oder der Photopeak
zu stark gemessen wurde. 
Die Diskrepanz kann auch daher kommen, dass in der Theorie nicht berücksichtigt wird, dass ein Gammaquant zunächst durch mehrfach
Comptonstreuungen Energie verliert und daher der Photoeffekt immer wahrscheinlicher wird. Wenn das Gammaquant  
die restliche Energie durch den Photoeffekt abgibt, wurde die gesamte Energie des Gammaquants im Detektor deponiert. Daher wird es 
dem Photopeak zugeordnet. Diese Mehrfachwechselwirkung wird in der Theorie nicht berücksichtigt und dadurch der Inhalt 
unterschätzt.
\subsection{Aktivitätsbestimmung}
Bei diesem Versuchsteil wird zunächst geguckt, ob das untersuchte Material eine \ce{^125Sb} oder eine \ce{^133Ba} Probe ist.
Bei der Betrachtung des Spektrums ist zu erkennen, dass es sich um eine \ce{^133Ba} Probe handelt.
Hierfür konnten sechs Peaks identifiziert werden und von denen sind fünf geeignet, um die Aktivität zu bestimmen.
Bei vier der fünf Peaks liegt die berechnete Aktivität in einer $1\sigma$ Umgebung der anderen Aktivitäten. 
Das bestätigt die gemessene Aktivität.
\subsection{Bestimmung einer unbekannten Probe}
Bei diesem Versuchsteil wird eine unbekannte Probe untersucht. Aufgrund der Ausdehnung der Probe und da diese direkt auf dem Detektor
lag, kann hier keine Aktivität bestimmt werden.
Um dem Material eine Zerfallskette zuordnen zu können, konnten 16 Peaks identifiziert werden. Alle Stoffe, welche identifiziert 
werden konnten, können der natürlichen Zerfallskette von Uran-238 zugeordnet werden. Ein Peak kann entweder der Zerfallskette 
oder Uran-235 zugeordnet werden. Dieses Isotop ist ein natürlicher Bestandteil von Uranerz. 