\section{Diskussion}
\label{sec:Diskussion}
\section{Diskussion}
\label{sec:diskussion}

Die Auswertung der beiden Würfel mit homogenen Materialverteilungen ergab die in
Tabelle~\ref{tab:2und3} aufgeführten Absorptionskoeffizienten. Vergleicht man 
diese mit den Literaturwerten in Tabelle~\ref{tab:lit}, so ergeben sich sehr 
eindeutige Zuordnungen zu den aufgeführten Materialien. Würfel~2 stimmt mit 
einem bestimmten Koeffizienten von $\mu_2 = \SI{0.19(1)}{\per\centi\meter}$ 
sehr gut mit einer Zusammensetzung aus Aluminium überein. Die Abweichung beträgt
etwa $\SI{6}{\percent}$. Würfel~3 wies nach Messung einen 
Absorptionskoeffizienten von $\mu_3 = \SI{1.04(6)}{\per\centi\meter}$ auf.
Dieser stimmt wiederum am besten mit einer Zusammensetzung aus Blei überein. 
Die Abweichung beträgt hierbei etwa $\SI{16.5}{\percent}$.

\begin{table}[htb]
  \centering
  \caption{Absorptionskoeffizienten einiger Metalle. Die Werte folgen aus den Dichten und Absorptionskoeffizienten der einzelnen Elemente~\cite{koeff}.}
  \begin{tabular}{c
                  S[table-format=1.3]
									S[table-format=2.2]
									S[table-format=1.3]}
    \toprule
    {Material} & {$\sigma$, $\si{\centi\meter\squared\per\gram}$} & {$\rho$, $\si{\gram\per\centi\meter^{3}}$} & {$\mu$, $\si{\per\centi\meter}$} \\
		\midrule
    Blei & 0.110 & 11.34 & 1.245 \\
    Messing & 0.073 & 8.41 & 0.614 \\
	Eisen & 0.073 & 7.86 & 0.574 \\
	Aluminium & 0.075 & 2.71 & 0.203 \\
	Delrin & 0.082 & 1.42 & 0.116 \\
    \bottomrule
  \end{tabular}
  \label{tab:lit}
\end{table}

Die Messwerte für Würfel~5 lassen auf die folgende Zusammensetzung aus 
Teilwürfeln schließen.

\begin{table}[htb]
  \centering
  \caption{Aus den verschiedenen Absorptionskoeffizienten bestimmte Zusammensetzung der Teilwürfel von Würfel 5.}
  \begin{tabular}{c|
                  S[table-format=1.4(1)]
                  S
                  c}
    \toprule
    {Teilwürfel} & {Absorptionskoeffizient $\mu$, $\si{\per\centi\meter}$} & {Abweichung, $\si{\percent}$} & {Material} \\
	\midrule
    1 &  0.35(8) & 72.4 & Aluminium \\
    2 &  0.72(6) & 17.3 & Messing (soll: Pb) \\
    3 &  0.31(8) & 52.7 & Aluminium \\
    4 &  0.06(6) & 48.3 & Delrin/Luft (soll: Al) \\
    5 &  1.09(8) & 12.5 & Blei \\
    6 &  0.15(6) & 26.1 & Aluminium \\
    7 &  0.14(8) & 31.0 & Aluminium \\
    8 &  0.28(6) & 37.9 & Aluminium \\
    9 &  0.12(8) & 3.4 & Delrin (soll: Al) \\
    \bottomrule
  \end{tabular}
  \label{tab:ergebnisse5}
\end{table}

Diese Schätzung stellt allerdings lediglich eine Verknüpfung des bestimmten 
Absorptionskoeffizienten mit dem nächsten Wert eines oben aufgeführten 
Materials dar.
Dabei sind die Abweichungen von den Literaturwerten allerdings durchweg recht 
hoch. Außerdem sind die statistischen Fehler der einzelnen Messungen bereits 
ziemlich groß und reichen wie etwa im Fall von Teilwürfel~4 an den Nominalwert.
Hier zeigt sich wohl im Vergleich zu den Vermessungen der ersten beiden Würfel, 
dass der Unterschied in der Statistik, bzw. das Verwenden eines überbestimmten 
Gleichungssystems den Fehler deutlich vermindert. \\
Die größten Fehlerquellen liegen allerdings vermutlich in der Datennahme selbst.
Besonders die Projektionen, welche nur einen Teilwürfel durchlaufen sind 
anfällig für Messfehler. Hier beeinflusst schon eine kleine Abweichung von der 
richtigen Ausrichtung das Ergebnis sehr, weil ein erheblicher Teil des Strahls 
dann durch andere Teilwürfel verläuft. \\
Generell aber ist die Genauigkeit der Justage des Würfels in diesem Versuch 
schwierig, da eine genaue Überprüfung der Ausrichtung wegen fehlender 
Orientierungshilfen nicht möglich ist. Dies sorgt für eine Ungenauigkeit in der 
Wegstrecke des Teilchenstrahls, besonders bei den Diagonalprojektionen. Hier 
tritt aber auch für ein Ausschmieren in nebenliegende Teilwürfel auf. \\
Des Weiteren ist auch die aus der radioaktiven Quelle stammende Strahlung kein 
perfekt fokussierter Strahl. Viel mehr treten schon hier "Verschmierungen" und 
Ungenauigkeiten auf, weswegen besonders bei den diagonalen Projektionen 
ebenfalls Abweichungen durch nebenliegenede Teilwürfel auftreten.
Anhand der Größe der Fehler auf die Ergebnisse ist es sehr schwierig, den 
Messungen für Würfel~5 Materialien zuzuordnen. Dies liegt vorallem auch daran, 
dass viele Materialien sehr ähnliche Absorptionskoeffizienten haben und somit 
eine eindeutige Zuordnung schwierig ist. Allerdings zeigen einige Teilwürfel, 
wie etwa $\mu_5$ auch sehr klare Zuordnungen.
