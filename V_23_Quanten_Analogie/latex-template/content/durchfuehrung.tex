\section{Versuchsaufbau}
\label{sec:Versuchsaufbau}
Die wichigsten Bausteine dieses Versuches sind der Lautsprecher und das Mikrofon, sowie 
die damit verbunden Steuerelektronik. Sie werden für jeden Versuchsteil benötigt.
Innerhalb dieser Steuerelektronik befindet sich ein Frequenz-Spannungskonverter, 
der es ermöglicht Frequenzspektren auf einem Oszilloskop oder COmputer zu visualisieren.
Auf dem Computer erfolgt die Darstellung mit dem Programm SpectrumSLC.
Der Lautsprecher ist mit einem Sinusgenreator verbunden, üder den die passenden 
Frequenzen erzeugt werden können.\\

Für unterschiedliche Versuchsteile werden verschiedene Hohlraumresonatoren benötigt:
\begin{itemize}
    \item 1-dim Festkörper
        \begin{itemize}
            \item Aluminiumzylinder ()
        \end{itemize}
\end{itemize}




\section{Versuchsdurchführung}
\label{sec:Versuchsdurchführung}
