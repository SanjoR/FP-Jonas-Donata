\section{Diskussion}
\label{sec:Diskussion}
Bei der Bestimmung der Schallgeschwindigkeit wurde ein Wert von \SI{339.8(7)}{\meter\per\second} ermittelt. 
Die relative Unsicherheit des Wertes liegt bei $\num{0.2}$\%. Der Theoriewert der Schallgeschwindigkeit bei 
\SI{20}{\celsius} liegt bei \SI{343.2}{\meter\per\second}. Der Theoriewert liegt nicht in der Unsicherheit des 
gemessenen Wertes. Das kann daran liegen, dass die Raumtemperatur nicht konstant auf \SI{20}{\celsius} gehalten wurde.

\subsection{Das Wasserstoffmodel}
Um ein Wasserstoffatom zu simulieren wird ein Kugelresonator verwendet. Die Resonanzen wurden vermessen, sodass in den darauf 
folgenden Messungen die Resonanzen gezielt verwendet werden können.
\subsubsection{Vermessung der Kugelflächenfunktionen}
Um die Kugelflächenfunktionen zu vermessen, wird das akustische Signal auf einer der vorher bestimmten Resonanzen eingestellt 
und die Druckamplitude wird in Abhängigkeit des Drehwinkels aufgenommen.
Von den drei aufgenommenen Verteilungen, konnten zwei genauer bestimmt werden und eine konnte eingegrenz werden.
Die Kugelflächenfunktionen konnten die Ordnungszahlen $n=0$, $n=4$ und $n>4$ zugeordnet werden.
\subsubsection{Peakaufspaltung}
Bei diesem Versuchsteil soll auf ein Zusammenhang der Peakaufspaltung und der Dicker der Zwischenringe geschlossen werden.
Da nur drei verschiedene Zwischenringe verwendet werden konnten, kann keine komplexere Funktion als eine lineare Funktion 
verwendet werden. Auch diese kann nicht gut gefittet werden, allerdings kann gesagt werden, dass die Aufspaltung in dem Bereich
mit der Dicke ansteigt. 
\subsubsection{Bestimmung des Quantenzustandes}
Um den Quantenzustand zu bestimmen, wird ein Zwischenringe in den Kugelresonator eingebaut. Hierdurch wird die 
Entartung aufgehoben. 
Es wird die Druckamplitude in Abhängigkeit des Drehwinkels aufgenommen. Diese wird zum vergleich zusammen mit den verschiedenen
Kugelflächenfunktionen dargestellt.
Anhand der Ähnlichkeit der gemessenen Daten zu den Kugelflächenfunktionen kann der Quantenzustand bestimmt werden.
In diesem Fall entspricht der Kugelresonator einem $n=2$, $m=0$ Zustand.
\subsection{Das Wasserstoffmolekül}
Das Wasserstoffmolekül wird durch zwei aufeinander gesteckten Kugelresonator simuliert, zwischen den Resonatoren kann eine Blende 
eingesetzt werden.
Zunächst sollte ein Zusammenhang von den Resonanzen mit dem Blendendurchmesser herrausgefunden werden. Da nur zwei Blenden
zu Verfühgung standen, kann nur ein linearer Zusammenhang gezogen werden. Hierzu wird die Verschiebung der einzelnen 
Resonanzen berechnet. Da die Verschiebungen nicht eindeutig sind, kann kein Zusammenhang gesehen werden.
\subsubsection{Bestimmung des Quantenzustandes}
Um den Quantenzustand zu bestimmen wird der Drehwinkel gegen die Druckamplitude aufgetragen. Die aufgenommenen Daten werden
mit den verschiedenen Kugelflächenfunktionen verglichen. Daraus lässt sich schließen, dass der Zustand ein $1\pi$ Zustand ist,
dieser kann bindend oder antibindend sein.
\subsection{1D-Festkörper}
Der 1D-Festkörper wird mit einer Kette an Zylindern simuliert, welche mit verschiedenen Blenden getrennt werden.
Je mehr Zylinder, desto mehr spalten sich die Resonanzen auf.
Die einzelnen Zylinder stellen in dem Model Atome dar. Die Blenden können als Stärke der Kopplung zwischen den einzelnen 
Atomen interpretiert werden.
Wenn hierbei ein Zylinder ausgetauscht wird, ist das wie eine Fehlstellung in einem Gitter, sodass die Resonanz geringer wird.