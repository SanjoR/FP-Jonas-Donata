\section{Diskussion}
\label{sec:Diskussion}
Bei der Bestimmung der Schallgeschwindigkeit wurde ein Wert von $\SI{339.8(7)}{\meter\per\second}$ ermittelt. 
Die relative Unsicherheit des Wertes liegt bei $\num{0.2}$\%. Der Theoriewert der Schallgeschwindigkeit bei einer Temperatur von
$\SI{20}{\celsius}$ liegt bei $\SI{343.2}{\meter\per\second}$ \cite{Schallgeschwindigkeit}. Der Theoriewert liegt nicht in der Unsicherheit des 
gemessenen Wertes. Das kann daran liegen, dass die Raumtemperatur nicht konstant auf $\SI{20}{\celsius}$ gehalten wurde.

\subsection{Das Wasserstoffmodel}
Um ein Wasserstoffatom zu simulieren wird ein Kugelresonator verwendet. Die Resonanzen wurden vermessen, sodass in den darauf 
folgenden Messungen die Resonanzen gezielt verwendet werden können.
\subsubsection{Vermessung der Kugelflächenfunktionen}
Um die Kugelflächenfunktionen zu vermessen, wird das akustische Signal auf einer der vorher bestimmten Resonanzen eingestellt 
und die Druckamplitude wird in Abhängigkeit des Drehwinkels aufgenommen.
Den aufgenommenen Verteilungen kann jeweils eine Kugelflächenfunktion zugeordnet werden. Bei einer Resonanzfrequenz von 
\SI{2.3164}{\kilo\hertz} wurde eine konstante Druckverteilung gemessen und daher kann auf $l=1,m=0$ geschlossen werden.
Bei \SI{5.0074}{\kilo\hertz} können die Messwerte mit einer Kugelflächenfunktion mit $l=3$ identifiziert werden.
Da eine Resonanz von \SI{6.5908}{\kilo\hertz} höher energetisch ist, jedoch die Kugelflächenfunktion mit $l=2$ zu den Messdaten
passt, muss die radiale Mode sich erhöht haben.
\subsubsection{Peakaufspaltung}
Bei diesem Versuchsteil soll auf einen Zusammenhang zwischen der Peakaufspaltung und der Dicke der Zwischenringe geschlossen werden.
Da nur drei verschiedene Zwischenringe verwendet werden konnten, kann keine komplexere Funktion als eine lineare Funktion 
verwendet werden. Auch diese kann nicht gut gefittet werden. Allerdings kann gesagt werden, dass die Aufspaltung im gemessenen Bereich
mit der Dicke ansteigt. 
\subsubsection{Bestimmung des Quantenzustandes}
Um den Quantenzustand zu bestimmen, wird ein Zwischenring in den Kugelresonator eingebaut. Hierdurch wird die 
Entartung aufgehoben. 
Es wird die Druckamplitude in Abhängigkeit des Drehwinkels aufgenommen. Diese wird zum Vergleich zusammen mit den verschiedenen
Kugelflächenfunktionen dargestellt.
Da in einem vorherigen Versuch bei einer Resonanz von \SI{2.3164}{\kilo\hertz} $l=1$ bestimmt wurde, muss nur noch $m$ bestimmt werden.
Wie an den Kugelflächenfunktionen zu erkennen ist kann $m$ nur $\pm 1 $ sein.
\subsection{Das Wasserstoffmolekül}
Das Wasserstoffmolekül wird durch zwei aufeinander gesteckte Kugelresonatoren simuliert, zwischen den Resonatoren kann eine Blende 
eingesetzt werden.
Zunächst sollte ein Zusammenhang zwischen den Resonanzen und dem Blendendurchmesser herausgefunden werden. Da nur zwei Blenden
zu Verfühgung standen, kann nur ein linearer Zusammenhang gezogen werden. Hierzu wird die Verschiebung der einzelnen 
Resonanzen berechnet. Da die Verschiebungen nicht eindeutig sind, kann kein Zusammenhang erkannt werden.
\subsubsection{Bestimmung des Quantenzustandes}
Um den Quantenzustand zu bestimmen, wird der Drehwinkel gegen die Druckamplitude aufgetragen. Die aufgenommenen Daten werden
mit den verschiedenen Kugelflächenfunktionen verglichen. Dadaurch kann der Zustand auf einen $2\sigma$-Zustand eingegrenzt werden, 
allerdings kann nicht gesagt werden, ob der Zustand bindend oder antibindend ist.
\subsection{Der eindimensionale Festkörper}
Der eindimensionale Festkörper wird mit einer Kette aus Zylindern simuliert, welche mit verschiedenen Blenden getrennt werden.
Je mehr Zylinder, desto öfter werden die resonanzen aufgespalten.
Die einzelnen Zylinder stellen in diesem Model Atome dar. Die Blenden können als Stärke der Kopplung zwischen den einzelnen 
Atomen interpretiert werden.
Wenn dabei ein Zylinder ausgetauscht wird, ist das äquivalent zu einer Fehlstellung in einem Gitter. Hierbei ist 
ein längerer Zylinder wie ein Donator und ein kürzerer verhält sich wie ein Akzeptor.