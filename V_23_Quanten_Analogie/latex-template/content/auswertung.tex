\section{Auswertung}
\label{sec:Auswertung}
\subsection{Vorbereitende Versuche}
Zur Vorbereitung werden zwei Versuche durchgeführt. Als erstes wird die Schallgeschwindigkeit über die Resonanzfrequenzen bestimmt. Danach wird ein Vergleich
zwischen der Datennahme mittels Oszilloskop und mit Hilfe des Programms \texttt{SpectrumSLC} gezogen.
\subsubsection{Bestimmung der Schallgeschwindigkeit}
Aus \eqref{bragg} kann die Schallgeschwindigkeit über die Differenz der Resonanzfrequenzen
bestimmt werden. Hierfür werden die Frequenzen gegen die Länge der Röhren aufgetragen. Die Länge der Röhren ist bekannt, da nur Röhren 
der gleichen Länge hinzugefühgt werden.
\FloatBarrier
\begin{table}
    \centering
    \caption{Messwerte für die Bestimmung der Schallgeschwindigkeit}
    \label{tab:Schallgeschwindigkeit}
    \begin{tabular}{c c c c}
        \toprule
        Länge /\SI{}{\milli\meter}& erste Resonanz /\SI{}{\kilo\hertz} & zweite Resonanz /\SI{}{\kilo\hertz}& Differenz /\SI{}{\kilo\hertz}\\
        \midrule
        $\num{50}$&$\num{6.87}$&$\num{10.28}$&$\num{3.41}$\\
        $\num{100}$&$\num{6.89}$&$\num{8.6}$&$\num{1.71}$\\
        $\num{150}$&$\num{6.895}$&$\num{8.05}$&$\num{1.16}$\\
        $\num{200}$&$\num{6.897}$&$\num{7.759}$&$\num{0.86}$\\
        $\num{250}$&$\num{6.9}$&$\num{7.59}$&$\num{0.69}$\\
        $\num{300}$&$\num{6.9}$&$\num{7.477}$&$\num{0.58}$\\
        \bottomrule
    \end{tabular}
\end{table}
\FloatBarrier
Durch die Messwerte aus \ref{tab:Schallgeschwindigkeit} wird die Funktion
\begin{equation*}
    f(x) = a \frac{1}{x} + b
\end{equation*}
gefittet.\\
Mit der Gleichung 
\begin{equation*}
    nc = 2df +b =2a+b
\end{equation*}
kann die Schallgeschwindigkeit bestimmt werden.
\FloatBarrier
\begin{figure}
    \includegraphics[width = 0.6\textwidth]{figure/Schallgeschwindigkeit.pdf}
    \caption{Messwerte und Fitfunktion für die Bestimmung der Schallgeschwindigkeit}
\end{figure}
\FloatBarrier
Die Parameter der Funktion sind:
\begin{align*}
    a&= \num{169.9(3)}\\
    b&= \num{0.013(3)}
\end{align*}
Daraus kann der Wert $c=\SI{339.8(7)}{\meter\per\second}$ bestimmt werden.

\subsubsection{Vergleich der Datennahme}
Um die beiden Methoden der Datennahme zu testen wird das Spektrum des Zylinders vermessen.
Hierbei wird die Anzahl der Zylinder schrittweise von eins auf sechs erhöht.
\FloatBarrier
\begin{figure}
    \subfloat[][]{\includegraphics[width=0.5\textwidth]{figure/1Zylinder.png}}
    \subfloat[][]{\includegraphics[width=0.5\textwidth]{figure/1_Zylinder.png}}
    \caption{Spektrum von einem Zylinder mittels (a) Oszilloskop, (b) \texttt{SpectrumSLC}}
\end{figure}
\begin{figure}
    \subfloat[][]{\includegraphics[width=0.5\textwidth]{figure/2Zylinder.png}}
    \subfloat[][]{\includegraphics[width=0.5\textwidth]{figure/2_Zylinder.png}}
    \caption{Spektrum von zwei Zylindern mittels (a) Oszilloskop, (b) \texttt{SpectrumSLC}}
\end{figure}
\begin{figure}
    \subfloat[][]{\includegraphics[width=0.5\textwidth]{figure/3Zylinder.png}}
    \subfloat[][]{\includegraphics[width=0.5\textwidth]{figure/3_Zylinder.png}}
    \caption{Spektrum von drei Zylindern mittels (a) Oszilloskop, (b) \texttt{SpectrumSLC}}
\end{figure}
\begin{figure}
    \subfloat[][]{\includegraphics[width=0.5\textwidth]{figure/4Zylinder.png}}
    \subfloat[][]{\includegraphics[width=0.5\textwidth]{figure/4_Zylinder.png}}
    \caption{Spektrum von vier Zylindern mittels (a) Oszilloskop, (b) \texttt{SpectrumSLC}}
\end{figure}
\begin{figure}
    \subfloat[][]{\includegraphics[width=0.5\textwidth]{figure/5Zylinder.png}}
    \subfloat[][]{\includegraphics[width=0.5\textwidth]{figure/5_Zylinder.png}}
    \caption{Spektrum von fünf Zylindern mittels (a) Oszilloskop, (b) \texttt{SpectrumSLC}}
\end{figure}
\begin{figure}
    \subfloat[][]{\includegraphics[width=0.5\textwidth]{figure/6Zylinder.png}}
    \subfloat[][]{\includegraphics[width=0.5\textwidth]{figure/6_Zylinder.png}}
    \caption{Spektrum von sechs Zylindern mittels (a) Oszilloskop, (b) \texttt{SpectrumSLC}}
\end{figure}
\FloatBarrier
Wie an den Bildern erkannbar ist, misst das Programm \texttt{SpectrumSLC} präziser.
Die Positionen der Maxima ändern sich nicht aufgrund der unterschiedlichen Messmethoden.
Außerdem kann das Programm die Daten abspeichern. Das vereinfacht die Auswertung der folgenden Messreihen.

\subsection{Das Wasserstoffatom}
Für den ersten Versuch dieser Messreihe, werden die Frequenzen von \SI{100}{\hertz} bis \SI{10}{\kilo\hertz} vermessen.
Die Resonanzfrequenzen sind in Tabelle \ref{tab:Resonanzfrequenzen} aufgelistet.
\FloatBarrier
\begin{table}
    \centering
    \caption{Messwerte für die Bestimmung der Ordnung der Resonanzfrequenz}
    \label{tab:Resonanzfrequenzen}
    \begin{tabular}{c c c c}
        \toprule
        Ordnung&Resonanzfrequenz /\SI{}{\kilo\hertz} &Amplitude /\SI{}{\milli\volt}& Phasenverschiebung /\SI{}{\degree}\\
        \midrule
        $\num{1}$&$\num{0.4376}$&$\num{160}$&$\num{0}-  \num{30}$\\
        $\num{2}$&$\num{2.3164}$&$\num{150}$&$\num{-20}-\num{4}$\\
        $\num{3}$&$\num{3.7095}$&$\num{150}$&$\num{-20}-\num{0}$\\
        $\num{4}$&$\num{5.0074}$&$\num{160}$&$\num{-20}-\num{0}$\\
        $\num{5}$&$\num{6.2361}$&$\num{160}$&$\num{-20}-\num{0}$\\
        $\num{6}$&$\num{6.5908}$&$\num{160}$&$\num{-20}-\num{4}$\\
        $\num{7}$&$\num{7.4648}$&$\num{160}$&$\num{-15}-\num{7}$\\
        $\num{8}$&$\num{8.0705}$&$\num{165}$&$\num{-20}-\num{0}$\\
        $\num{9}$&$\num{8.6602}$&$\num{170}$&$\num{-30}-(\num{-5})$\\
        $\num{10}$&$\num{9.5068}$&$\num{170}$&$\num{-30}-(\num{-5})$\\
        $\num{11}$&$\num{9.8462}$&$\num{175}$&$\num{-15}-\num{3}$\\
        \bottomrule
    \end{tabular}
\end{table}
\FloatBarrier
Die Resonanzen können auch mittels des Frequenzspektrometers dargestellt werden.
\FloatBarrier
\begin{figure}
    \caption{Frequenzspektrum des Kugelresonators}
\includegraphics[width = \textwidth]{figure/Kugelresonanzspektrum.png}
\end{figure}

\subsubsection{Vermessung der Kugelflächenfunktion}
In dieser Messreihe werden für drei verschiedene Resonazen die Druckamplituden in Abhängigkeit des Drehwinkels $\alpha$ vermessen.
Es werden die Resonanzen der zweiten, vierten und sechsten Ordnung aus Tabelle \ref{tab:Resonanzfrequenzen} verwendet.
\FloatBarrier
\begin{table}
    \centering
    \caption{Messwerte für die Bestimmung der Ordnung der Kugelflächenfunktion}
    \label{tab:Amplituden_Kugelflächenfunktion}
    \begin{tabular}{c c c c c}
        \toprule
        Winkel $\alpha$ /\SI{}{\degree}&Winkel $\theta$ /\SI{}{\degree}&Amplitude 2 /\SI{}{\milli\volt}&Amplitude 4 /\SI{}{\milli\volt}& Amplitude 6 /\SI{}{\milli\volt}\\
        \midrule
        $\num{0}$  &$\num{90.0}$  &$\num{160}$&$\num{0.12}$&$\num{1.37}$\\
        $\num{10}$ &$\num{90.4}$ &$\num{160}$&$\num{0.68}$&$\num{1.45}$\\
        $\num{20}$ &$\num{91.7}$ &$\num{160}$&$\num{1.1}$&$\num{1.29}$\\
        $\num{30}$ &$\num{93.8}$ &$\num{160}$&$\num{1.4}$&$\num{1.05}$\\
        $\num{40}$ &$\num{96.7}$ &$\num{160}$&$\num{1.6}$&$\num{1.37}$\\
        $\num{50}$ &$\num{100.3}$ &$\num{160}$&$\num{2.0}$&$\num{1.21}$\\
        $\num{60}$ &$\num{104.5}$ &$\num{160}$&$\num{2.37}$&$\num{0.88}$\\
        $\num{70}$ &$\num{109.2}$ &$\num{160}$&$\num{2.61}$&$\num{0.8}$\\
        $\num{80}$ &$\num{114.4}$ &$\num{160}$&$\num{2.77}$&$\num{0.72}$\\
        $\num{90}$ &$\num{120.0}$ &$\num{160}$&$\num{2.7}$&$\num{0.84}$\\
        $\num{100}$&$\num{125.9}$&$\num{160}$&$\num{2.21}$&$\num{1.13}$\\
        $\num{110}$&$\num{132.1}$&$\num{160}$&$\num{1.45}$&$\num{1.37}$\\
        $\num{120}$&$\num{138.6}$&$\num{160}$&$\num{0.4}$&$\num{1.69}$\\
        $\num{130}$&$\num{145.2}$&$\num{160}$&$\num{1.29}$&$\num{2.01}$\\
        $\num{140}$&$\num{152.0}$&$\num{160}$&$\num{2.69}$&$\num{2.29}$\\
        $\num{150}$&$\num{158.9}$&$\num{160}$&$\num{4.34}$&$\num{2.61}$\\
        $\num{160}$&$\num{165.9}$&$\num{160}$&$\num{5.3}$&$\num{2.85}$\\
        $\num{170}$&$\num{172.9}$&$\num{160}$&$\num{6.03}$&$\num{3.34}$\\
        $\num{180}$&$\num{180.0}$&$\num{160}$&$\num{6.43}$&$\num{3.26}$\\
        \bottomrule
    \end{tabular}
\end{table}
\FloatBarrier
In der Tabelle \ref{tab:Amplituden_Kugelflächenfunktion} wird der Winkel $\theta$ mit der Funktion \eqref{eq:Theta_alpha} aus $\alpha$ bestimmt.
\begin{equation}
    \label{eq:Theta_alpha}
    \theta(\alpha) = \arccos\left( \frac{1}{2} \cos\left( \alpha \right) -\frac{1}{2} \right)
\end{equation}
Wie aus Tabelle \ref{tab:Amplituden_Kugelflächenfunktion} abzulesen werden kann, ist die Druckamplitude der zweiten Resonanz 
winkelunabhängig, daher müssen die Quantenzahlen $l=1,m=0$ sein.\\
Die Druckamplitude wird in einem Polarplot gegen $\theta$ aufgetragen. Dazu werden verschiedene Kugelflächenfunktionen 
geplottet, um diese miteinander vergleichen zu können. Da ohne Zwischenring keine Aufspaltung in $m$ zu sehen ist, wird $m=0$ gesetzt.
Die Kugelflächenfunktion werden somit nur in Abhängigkeit von $\theta$ geplottet.
\FloatBarrier
\begin{figure}
    \subfloat[][]{\includegraphics[width=0.5\textwidth]{figure/Resonanz_Drewinkel_Amplitude_4_n1.pdf}}
    \subfloat[][]{\includegraphics[width=0.5\textwidth]{figure/Resonanz_Drewinkel_Amplitude_4_n2.pdf}}\\
    \subfloat[][]{\includegraphics[width=0.5\textwidth]{figure/Resonanz_Drewinkel_Amplitude_4_n3.pdf}}
    \subfloat[][]{\includegraphics[width=0.5\textwidth]{figure/Resonanz_Drewinkel_Amplitude_4_n4.pdf}}
    \caption{Winkelabhängigkeit der Druckamplitude bei einer Resonanzfrequenz von \SI{5.0074}{\kilo\hertz} und verschiedene Kugelflächenfunktionen.}
    \label{fig:Amplitude_4}
\end{figure}
\begin{figure}
    \subfloat[][]{\includegraphics[width=0.5\textwidth]{figure/Resonanz_Drewinkel_Amplitude_6_n1.pdf}}
    \subfloat[][]{\includegraphics[width=0.5\textwidth]{figure/Resonanz_Drewinkel_Amplitude_6_n2.pdf}}\\
    \subfloat[][]{\includegraphics[width=0.5\textwidth]{figure/Resonanz_Drewinkel_Amplitude_6_n3.pdf}}
    \subfloat[][]{\includegraphics[width=0.5\textwidth]{figure/Resonanz_Drewinkel_Amplitude_6_n4.pdf}}
    \caption{Winkelabhängigkeit der Druckamplitude bei einer Resonanzfrequenz von \SI{6.5908}{\kilo\hertz} und verschiedene Kugelflächenfunktionen.}
    \label{fig:Amplitude_6}
\end{figure}
\FloatBarrier
Bei Betrachtung der Abbildungen \ref{fig:Amplitude_4} kann geschlussfolgert werden, 
dass die gemessene Verteilung der Kugelflächenfunktion mit $l=3$ ähnelt. \\
Die gemessene Verteilung bei der Resonanzfrequenz von \SI{6.5908}{\kilo\hertz} ähnelt der Kugelflächenfunktion von $l=2$, da
diese aufgrund der höheren Frequenz eigentlich höher energetisch sein muss, muss die radiale Mode sich um einen erhöht haben.

\subsubsection{Peakaufspaltung}
Bei dieser Messreihe wird die Peakaufspaltung der Resonanzfrequenz von $\SI{2.3}{\kilo\hertz}$ vermessen, indem verschieden dicke 
Ringe zwischen den Halbkugeln des Kugelresonators eingefügt werden.
Die dafür benötigten Messdaten sind in der folgenden Tabelle  aufgelistet.
\FloatBarrier
\begin{table}
    \centering
    \caption{Messwerte für die Bestimmung des Zusammenhangs zwischen der Dicke des Zwischerings und der Aufspaltung der Resonanzfrequenz}
    %\label{tab:Peakaufspaltung}
    \begin{tabular}{c c c c}
        \toprule
        Dicke / \SI{}{\milli\meter}& erste Resonanz / \SI{}{\kilo\hertz}& zweite Resonanz / \SI{}{\kilo\hertz}& Differenz / \SI{}{\kilo\hertz}\\
        \midrule
        $\num{3}$ &$\num{2239}$&$\num{2302}$&$\num{63}$\\
        $\num{9}$ &$\num{2173}$&$\num{2286}$&$\num{113}$\\
        $\num{12}$&$\num{2103}$&$\num{2277}$&$\num{174}$\\
        \bottomrule
    \end{tabular}
\end{table}
\FloatBarrier
Aufgrund der wenigen Kombinationsmöglichkeiten der beiden Zwischenringe, können für diesen Versuch nur drei Messwerte aufgenommen werden.
Da aus drei Punkten nicht auf komplexe Funktionen geschlossen werden kann, wird der Zusammenhang als linear angenommen.
Dafür wird die Funktion \eqref{eq:linear} durch die Daten gefittet.
\begin{equation}
    f\left(x\right) = ax +b
    \label{eq:linear}
\end{equation}
\FloatBarrier
\begin{figure}
    \centering
    \caption{Messdaten und Fit für den Zusammenhang zwischen Zwischenringdicke und Peakaufspaltung}
    \label{fig:Peakaufspaltung}
    \includegraphics[width=0.7\textwidth]{figure/Peak_Aufspaltung.pdf}
\end{figure}
\FloatBarrier
In der Abbildung \ref{fig:Peakaufspaltung} sind die Parameter auf 
\begin{align*}
    a&= \num{12(3)}\\
    b&= \num{23(26)}
\end{align*}
bestimmt worden.

\subsubsection{Bestimmung des Quantenzustandes}
Um die Quantenzahlen des Zustandes zu bestimmen, wird ein Zwischenring der Dicke $d=\SI{9}{\milli\meter}$ eingefügt.
Da der Zwischenring die Symmetrie der Kugel bricht, gibt es jetzt eine Vorzugsrichtung. Dadurch ist der $\alpha$-Winkel der $\phi$-Winkel 
der Kugelflächenfunktion und die Ausrichtung des Mikrofons wird durch den $\theta$-Winkel charakterisiert.
Bei einer Frequenzen von $\SI{2.3}{\kilo\hertz}$ wird die winkelabhängige Druckamplitude vermessen, mit dem vorherigen Versuch konnte für diese Frequenz $l=1$ bestimmt werden. Die Messdaten sind in Tabelle
\ref{tab:Messdaten_9mmZwischenring} aufgelistet.
\FloatBarrier
\begin{table}
    \centering
    \caption{Messwerte für die Bestimmung des Quantenzustandes bei einer Resonanzfrequenz von $\SI{2.3}{\kilo\hertz}$ und einem Zwischenring der Dicke $d=\SI{9}{\milli\meter}$}
    \label{tab:Messdaten_9mmZwischenring}
    \begin{tabular}{c c c c c}
        \toprule
        Winkel $\phi$ /\SI{}{\degree}&Amplitude  /\SI{}{\milli\volt}\\
        \midrule
        $\num{0}$  &$\num{640}$ \\
        $\num{10}$ &$\num{640}$ \\
        $\num{20}$ &$\num{600}$ \\
        $\num{30}$ &$\num{560}$ \\
        $\num{40}$ &$\num{480}$ \\
        $\num{50}$ &$\num{400}$\\
        $\num{60}$ &$\num{320}$\\
        $\num{70}$ &$\num{220}$\\
        $\num{80}$ &$\num{125}$\\
        $\num{90}$ &$\num{120}$\\
        $\num{100}$&$\num{233}$\\
        $\num{110}$&$\num{350}$\\
        $\num{120}$&$\num{440}$\\
        $\num{130}$&$\num{520}$\\
        $\num{140}$&$\num{590}$\\
        $\num{150}$&$\num{675}$\\
        $\num{160}$&$\num{700}$\\
        $\num{170}$&$\num{760}$\\
        $\num{180}$&$\num{780}$\\
        \bottomrule
    \end{tabular}
\end{table}
\FloatBarrier
\begin{figure}
    \hspace*{2cm}
    \begin{minipage}[b]{.4\linewidth} % [b] => Ausrichtung an \caption
        \hspace*{-2cm}
        \includegraphics[width=\linewidth]{figure/9mmZwischenring_m0.pdf}
        \caption{Winkelabhängigkeit der\\ Druckamplitude bei einer \\ Resonanzfrequenz von $\SI{2.3}{\kilo\hertz}$ und \\ Kugelflächenfunktion mit $l=1,m=-1$.}
     \end{minipage}
     \hspace{.1\linewidth}% Abstand zwischen Bilder
     \begin{minipage}[b]{.4\linewidth} % [b] => Ausrichtung an \caption
        \hspace*{-2cm}
        \includegraphics[width=\linewidth]{figure/9mmZwischenring_m1.pdf}
        \caption{Winkelabhängigkeit der\\ Druckamplitude bei einer \\ Resonanzfrequenz von $\SI{2.3}{\kilo\hertz}$ und \\ Kugelflächenfunktion mit $l=1,m=+1$.}
     \end{minipage}
\end{figure}
\begin{figure}
    \centering
    \includegraphics[width=0.5\textwidth]{figure/9mmZwischenring_m-1.pdf}
    \caption{Winkelabhängigkeit der Druckamplitude bei einer Resonanzfrequenz von $\SI{2.3}{\kilo\hertz}$ und Kugelflächenfunktion mit $l=1,m=-1$.}
    \label{fig:9mmZwischenring_n2}
\end{figure}
\FloatBarrier
Wie in Abbildung \ref{fig:9mmZwischenring_n2} zu sehen ist, ähnelt der Zustand am meisten den $l=1,m=\pm 1$ Kugelflächenfunktionen.

\subsection{Das Wasserstoffmolekül}
Für die folgenden Versuchsteilen werden zwei Kugelresonatoren übereinander gesteckt und mit einem Loch in der Mitte in Verbindung gebracht.

\subsubsection{Resonanzfrequenz in Abhängigkeit des Blendendurchmessers}
Da nur zwei Blenden zu Verfügung standen, kann auch hier nur ein linearer Zusammenhang in Erwägung gezogen werden.
Die beiden Spektren werden mit einer Blende mit einem Durchmesser von $d=\SI{10}{\milli\meter}$ und
$d=\SI{16}{\milli\meter}$ aufgenommen. Die Spektren sind in Abbildung \ref{fig:Spektren_WM_Blende} zu sehen.
\FloatBarrier
\begin{figure}
    \centering
    \includegraphics[width = 0.7\textwidth , keepaspectratio]{figure/WM_Blenden.pdf}
    \caption{Spektren für den Zusammenhang zwischen den Resonanzfrequenzen und dem Blendendurchmesser.}
    \label{fig:Spektren_WM_Blende}
\end{figure}
\FloatBarrier
Die Resonanzen sind in Tabelle \ref{tab:Resonanz_WM_Blenden} aufgelistet.
\FloatBarrier
\begin{table}
    \centering
    \caption{Resonanzen für den Zusammenhang zwischen Resonanzfrequenz und Blendendurchmesser.}
    \label{tab:Resonanz_WM_Blenden}
    \begin{tabular}{c c c}
        \toprule
        Durchmesser $d=\SI{10}{\milli\meter}$ &Durchmesser $d=\SI{16}{\milli\meter}$& \\
        Frequenz $f /\SI{}{\hertz}$& Frequenz $f /\SI{}{\hertz}$&Differenz $\Delta f /\SI{}{\hertz}$\\
        \midrule
        $\num{280}$ &$\num{370}$ &$\num{90}$\\
        $\num{2320}$&$\num{2320}$&$\num{0}$\\
        $\num{3730}$&$\num{3730}$&$\num{0}$\\
        $\num{4980}$&$\num{5040}$&$\num{60}$\\
        $\num{6230}$&$\num{6200}$&$\num{-30}$\\
        $\num{6600}$&$\num{6640}$&$\num{40}$\\
        $\num{7440}$&$\num{7400}$&$\num{-40}$\\
        \bottomrule
    \end{tabular}
\end{table}
\FloatBarrier
Wie an den den Differenzen aus Tabelle \ref{tab:Resonanz_WM_Blenden} zu erkennen ist, kann aus diesen Daten kein 
Zusammenhang zwischen Blendendurchmesser und Resonanz gezogen werden.

\subsubsection{Quantenzustand des Wasserstoffmoleküls}
Für die Bestimmung des Quantenzustandes des Wasserstoffmoleküls wird die Druckamplitude in Abhängigkeit des Drehwinkels 
vermessen. Hierbei ist die Frequenz auf $\SI{2.3}{\kilo\hertz}$ eingestellt.
Die gemessenen Daten sind in Tabelle \ref{tab:WMQZ} aufgelistet.
\FloatBarrier
\begin{table}
    \centering
    \caption{Daten für die Bestimmung des Quantenzustandes des Wasserstoffmoleküls}
    \label{tab:WMQZ}
    \begin{tabular}{c c}
        \toprule
        Drehwinkel $\alpha / \SI{}{\degree}$ & Druckamplitude / \SI{}{\volt}\\
        \midrule
        $\num{0}$  &$\num{1.05}$\\
        $\num{10}$ &$\num{1.05}$\\
        $\num{20}$ &$\num{0.96}$\\
        $\num{30}$ &$\num{0.96}$\\
        $\num{40}$ &$\num{0.96}$\\
        $\num{50}$ &$\num{0.88}$\\
        $\num{60}$ &$\num{0.88}$\\
        $\num{70}$ &$\num{0.88}$\\
        $\num{80}$ &$\num{0.84}$\\
        $\num{90}$ &$\num{0.84}$\\
        $\num{100}$&$\num{0.8}$\\
        $\num{110}$&$\num{0.8}$\\
        $\num{120}$&$\num{0.8}$\\
        $\num{130}$&$\num{0.72}$\\
        $\num{140}$&$\num{0.72}$\\
        $\num{150}$&$\num{0.72}$\\
        $\num{160}$&$\num{0.72}$\\
        $\num{170}$&$\num{0.72}$\\
        $\num{180}$&$\num{0.72}$\\
        \bottomrule
    \end{tabular}
\end{table}
\FloatBarrier
Die Daten werden in Abbildung \ref{fig:WMQZ} dargestellt.
\FloatBarrier
\begin{figure}
    \centering
    \includegraphics[width = 0.7\textwidth, keepaspectratio]{figure/WMQZ.pdf}
    \caption{Messdaten und verschiedene Kugelflächenfunktionen, für die Bestimmung des Quantenzustandes}
    \label{fig:WMQZ}
\end{figure}
\FloatBarrier
Aufgrund der aufgenommenen $\phi$-Abhängigkeit kann nur ein $\sigma$ Zustand in Betracht kommen.
Da der Zustand energetisch nahe am $l=1,m=0$-Zustand des ungestörten Wasserstoffatoms liegt und der Molekülzustand 
aus den Zuständen des Wasserstoffatoms zusammengesetzt ist, wird es ein $2\sigma$-Zustand sein.
Ungrade und grade Zustände können weiter unterschieden werden, in dem der Phasenunterschied in den beiden Kugelhälften 
mit verwendet wird.

\subsection{Der eindimensionale Festkörper}
Für die Simulation eines eindimensionalen Festkörpers werden Zylinder mit einer 
Länge von $\SI{50}{\milli\meter}$ verwendet. Diese werden durch verschiedene Blenden voneinander getrennt.
\FloatBarrier
\begin{figure}
    \centering
    \includegraphics[width=\textwidth,keepaspectratio]{figure/Zylinder_Ketten.pdf}
    \caption{Resonanzen der Zylinderketten mit unterschiedlichen Längen und jeweils einer $\SI{10}{\milli\meter}$ Blende}
    \label{fig:Zylinder_Ketten}
\end{figure}
\FloatBarrier
Wie Abbildung \ref{fig:Zylinder_Ketten} vermuten lässt, werden die Resonanzen durch mehr Zylinder aufgespalten. Die 
Aufspaltung entspricht der Anzahl an Zylindern.

Diese Versuchsreihe wird wiederholt, wobei die Blenden einen Durchmesser von $\SI{13}{\milli\meter}$ haben.
\FloatBarrier
\begin{figure}
    \centering
    \includegraphics[width=\textwidth,keepaspectratio]{figure/Zylinder_Ketten_13mm.pdf}
    \caption{Resonanzen der Zylinderketten mit unterschiedlichen Längen und jeweils einer $\SI{13}{\milli\meter}$ Blende}
    \label{fig:Zylinder_Ketten_13}
\end{figure}
\FloatBarrier
Wenn die Abbildungen \ref{fig:Zylinder_Ketten} und \ref{fig:Zylinder_Ketten_13} verglichen werden, ist zu sehen, dass die Aufspaltungen 
bei Blenden von $\SI{13}{\milli\meter}$ weiter auseinander sind.
\subsubsection{Austausch der Zylinder}
Im Folgenden werden einzelne Zylinder ausgetauscht. Einmal wird ein Zylinder der Länge $\SI{75}{\milli\meter}$ eingesetzt und
einmal werden zwei Zylinder der Länge $\SI{12.5}{\milli\meter}$ hintereinander gesetzt, wobei zwischen ihnen keine Blende verwendet wird.
\FloatBarrier
\begin{figure}
    \centering
    \includegraphics[width=\textwidth,keepaspectratio]{figure/Austausch_75mm.pdf}
    \caption{Resonanzen der Zylinderketten mit 10 Zylindern, wobei ein Zylinder mit einem Zylinder der Länge $\SI{75}{\milli\meter}$ ausgetauscht wird}
    \label{fig:Austausch_75}
\end{figure}
\FloatBarrier
Wie an der Abbildung \ref{fig:Austausch_75} zu sehen ist, sind die Resonanzen weniger ausgeprägt.
\FloatBarrier
\begin{figure}
    \centering
    \includegraphics[width=\textwidth,keepaspectratio]{figure/Austausch_25mm.pdf}
    \caption{Resonanzen der Zylinderketten mit 10 Zylindern, wobei bei ein Zlyinder mit zwei Zylinder der Länge $\SI{12.5}{\milli\meter}$ ausgetauscht wird}
    \label{fig:Austausch_25}
\end{figure}
\FloatBarrier
In Abbildung \ref{fig:Austausch_25} ist zu sehen, dass die erste Resonanz erhöht ist. Die anderen sind allerdings wieder weniger ausgeprägt.
Bei beiden Spektren ist zu sehen, dass zwischen ein weiterer schwach ausgeprägter Peak zwischen Peak eins und Peak zwei entsteht.
Beim \SI{75}{\milli\meter} Zylinder ist der neu entstandende Peak näher beim zweiten Peak und beim \SI{12.5}{\milli\meter} Zylinder
ist dieser näher beim ersten Peak. Der \SI{75}{\milli\meter} Zylinder kann als Donator und der \SI{12.5}{\milli\meter} Zylinder 
als Akzeptor gesehen werden.


