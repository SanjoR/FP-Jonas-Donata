\section{Motivation}
\label{sec:Motivation}
Ziel dieses Versuches ist, eine Analogie zwischen akustischen und quantenmechanischen 
Modellen herzustellen. Dabei handelt es sich genauer um eine Resonatorkette aus 
Hohlzylindern, 
welche mit der quantenmechanischen Darstellung eines eindimensionalen Festkörpers 
verglichen werden soll,um einen Kugelresonator der mit einem Wasserstoffatom 
vergichen werden soll und um eine 
Kopplung aus zwei Kugelresonatoren, die mit einem Wasserstoffmolekül verglichen werden 
soll.

\section{Theorie}
\label{sec:Theorie}

\subsection{akustische Modelle}
Schallwellen, also Gasdruckschwankungen im Raum, breiten sich in verschieden geformten 
Hohlräumen spezifisch aus und zeigen zu unterschiedlichen Frequenzen unterschiedliche 
physikalische Phänomene. Grundlage ist dabei das Prinzip der stehenden Wellen.\\

Ein stehendes Wellenmuster bildet sich durch Überlagerung von Wellen und es kommt zu 
sogenannten Knotenpunkten, -linien oder -flächen im Raum. Das kann beispielsweise 
passieren, indem eine einlaufende Welle an einer harten Wand reflektiert wird und ihre 
Wellenlänge einem halbzahligen Vielfachen der Länge der schwingenden Substanz entspricht.
Die zu den Wellenlängen gehörigen Frequenzen werden Eigenfrequenzen genannt.
Die Beschreibung stehender Wellen benötigt zusätzlich Kenntnis über den Zustand der 
Randbedingungen der schwingenden Substanz.\\

Wird der eindimensionale Fall einer Welle betrachtet, wird für stehende Wellen der Fall 
eines festen Endes und der Fall zwei fester Enden betrachtet, welche zu verschiedenen 
Beschreibungen führen. Dieser Zusammenhang lässt 
sich auf Wellen in zwei oder drei Dimensionen übertragen. Für diesen Versuch sind 
Schallwellen in abgeschlossenen Hohlraumresonatoren von Relevanz, sodass
sich die entsprechenden Randbedingungen ergeben.\\

Wird ein abgeschlossener Zylinder betrachtet, in welchen ein akustisches Signal 
eingeleitet wird, dann ist es notwendig, dass dich am Boden und Deckel des Zylinders 
Knotenpunkte bilden um eine stehende Welle zu realisieren. Die Randbedingungen ergeben 
sich über die Druckänderungen an beiden Enden.\\

Ist der Körper, in welchen das akustische Signal geleitet wird jedoch kugelförmig, so 
werden zur Beschreibung der Eigenfrequenz andere mathematische Hilfsmittel benötigt.
Eigenmoden des Gases innerhalb einer Kugel sind weiterhin über stationäre Knotenpunkte 
definiert. Die möglichen Eigenfrequenzen lassen sich zu den entsprechenden 
Eigenfunktionen, den Kugelflächenfunktionen bestimmmen. 

\subsection{quantenmechanische Modelle}
Den zentralen Aspekt der Quantenmechanik bildet die Schrödingergleichung eines Teilchens,
durch die es möglich ist, für bestimmte Wellenfunktionen Energieeigenwerte zu den 
entsprechenden Frequenzen zu finden. Die zentrale Erkenntnis in diesem Bereich beruht 
auf  der Tatsache, das Wellenfunktionen und die dadurch beschriebenen Teilchen nur für 
quantisierte Eigenenergien die Schrödingergleichung erfüllen.\\

Für den Fall, dass ein Teilchen in einem Potentialtopf gefangen ist, ergeben sich
Randbedingungen an die Wellenfunktion des Teilchens. An den Rändern das Topfes muss die 
Wellenfunktion verschwinden. Dadurch sind notwendigerweise nur stehende Wellen erlaubt, 
deren Wellenlängen halbzahlige Vielfache der Ausdehnung des Potentialtopfes entsprechen.\\

Wird das Wasserstoffatom betrachtet, muss sowohl das Elektron, als ausch das Proton 
und die Kraft zwischen den beiden durch die Schrödingergleichung beschrieben werden.
Zur Lösung dieses Zweikörperproblems, wird die Schwerpunktmasse der teilchen angesetzt, 
sodass sich eine spezielle Form der Schrödingergleichung ergibt. Sie wird durch Funktionen 
gelöst, die die Kugelflächenfunktionen enthalten und von drei Quantenzahlen abhängen.\\

Die für die Lösung des Wasserstoffatoms notwendigen Quantenzahlen sind die Hauptquantenzahl,
die Drehimpulsquantenzahl und die magnetische Quantenzahl. Für alle möglichen Kombinationen dieser
Indizes lässt sich das Wasserstoffatom lösen bzw. lassen sich Eigenfunktionen aufstellen.\\

Soll ein Wasserstoffmolekül beschrieben werden, handelt es sich um ein Dreikörperproblem,
da zwei Protonen mit einem Elektron wechselwirken. Über einen Schwerpunktsanstz wird
eine Beschreibung durch elliptische Koordinaten möglich, während es beim Wasserstoffatom noch 
Kugelkoordinaten waren. Die Beschreibung der Eigenfunktionen dieses Systems sind dennoch durch 
die selben Quantenzahlen charakterisierbar.

\subsection{Modell des eindimensionalen Festkörpers}
Die Beschreibung eines Festkörpers erfolgt zunächst über die Modelldarstellung durch ein 
Gitter, durch das die periodisch angeordneten Atome des Festkörpers dargestellt werden können.
Für einen eindimensionalen Festkörper benötigt man entsprechend ein eindimenisonales Gitter,
also eine Aufreihung von äquidistanten Punkten entlang einer Linie.\\

Entsprechend dieser Äquidistanzen, lassen sich periodische Potentiale beschreiben, 
welche auf Elektronen wirken. Die zugehörige Schrödingergleichung wird mit 
Blochfunktionen gelöst. Es zeigt sich, dass die Elektronen nicht beliebige Energien 
einnehmen können, denn es bildet sich eine sogennante Energielücke heraus, in der 
keine Energieniveaus besetzt werden können.\\

Diejenigen Enegiebereiche, die von Elektronen besetzt werden können, werden Energiebänder 
genannt und hängen von den Paramtern des periodischen Potentials ab. Durch die 
Tatsache, dass es sich in der Realität nicht um unendlich ausgedehnte Festkörper handelt, 
sorgen die entsprechenden Randbedingungen dafür, dass selbst innnerhalb eines Bandes 
nur diskrete Energiewerte besetzt werden können.\\

\subsection{Analogien zwischen den Modellen}
Es ist zu erwarten, dass sich die gemessenen akustischen Eigenfrequenzen der Wellen,
welche durch zylindrische Hohlraumresonatoren geleitet werden, mit der Darstellung eines
Bändermodells aus der Festkörperphysik beschreiben lassen.\\
Das gleiche erwartet man für den Zusammenhang zwischen den Energiezuständen eines 
Wasserstoffatoms und den Eigenfrequenzen eines Kugelresonators, da beide sich über die 
Kugelflächenfunktionen ausdrücken lassen.\\
Die Erweiterung auf ein Wasserstoffmolekül, ließe sich dann entsprechend durch 
zwei Hohlraumresonatoren modellieren, da deren Eigenfunktionen einen elliptisch 
geformten Raum durchlaufen und die Beschriebung eines Wasserstoffmoleküls ebenfalls 
durch elliptische Koordinaten erfolgt und sich über die Kugelflächenfunktionen
lösen lässt, aus denen sich auch die Eigenfunktionen der gekoppelten Kugelresonatoren 
bilden lassen. 

\cite{sample}
demtröder exp phy 1\\
demtröder exp phy 3\\
kathi breagelmann protokoll\\
