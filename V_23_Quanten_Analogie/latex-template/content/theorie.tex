\section{Motivation}
\label{sec:Motivation}
Ziel dieses Versuches ist es, eine Analogie zwischen akustischen und quantenmechanischen Modellen herzustellen.
Dabei handelt es sich um eine Resonatorkette aus Hohlzylindern, welche mit der quantenmechanischen Darstellung eines 
eindimensionalen Festkörpers verglichen werden soll,um einen Kugelresonator der mit einem Wasserstoffatom und um eine Kopplung aus zwei Kugelresonatoren, 
die mit einem Wasserstoffmolekül verglichen werden sollen.

\section{Theorie}
\label{sec:Theorie}

\subsection{akustische Modelle}
Schallwellen, also Gasdruckschwankungen im Raum, breiten sich in verschieden geformten 
Hohlräumen spezifisch aus und zeigen zu unterschiedlichen Frequenzen unterschiedliche 
physikalische Phänomene. Grundlage ist dabei das Prinzip der stehenden Wellen.\\

Ein stehendes Wellenmuster bildet siihc durch Überlagerung von Wellen  und es kommt zu 
sogenannten Knotenpunkten, -linien oder -flächen im Raum. Das kann beispielsweise 
passieren, indemm  eine einlaufende Welle an einer harten Wand reflektiert wird und ihre 
Wellenlänge eiem halbzahligen Vielfachen der Länge der schwnigenden Substanz entspricht.
Die zu den Wellenlängen gehörigen Frequenzen werden EIgenfrequenzen genannt.
Die Beschreibung stehender Wellen benötigt zusätzlich Kenntniss über den Zustand der 
Randbedingeungen der schwingenden Substanz.\\

Wird der einddimensionale Fall eienr Welle betrachtet, wird für stehende Wellen der Fall 
eines Festen Endes und der Fall zwei fester Enden betrachtet, welche zu verschiedenen 
Beschreibungen führen. Dieser Zusammenhang lässt 
sich auf Wellen in zwei oder drei dimensionen Übertragen. Für diesen Versuch sind 
Schallwellen in abgeschlossenen HOhlraumresonatoren von relevanz, sodass
sich die entsprechenden RAndbedingungen ergeben.\\

Wird ein abgeschlossener Zylinder betrachtet, in welchen ein akustisches Signal eingeleitet
wird, dann ist es notwendig, dass dich am Boden und Deckel des Zylinders Knotenpunkte bilden 
um eine stehende welle zu realisieren. Die Randbedingungen ergeben sich über die Druckänderungen 
an beiden Enden.\\

Ist der Körper, in welchen das akustische Signal geleitet wird jedoch kugelförmig, so werden 
zur Beschreibung der EIgenfrequenz ander mathematische Hilfsmittel benötigt.
Eigenmoden des Gases inerhalb einer Kugel sind weiterhin über stationöre Knotenpunkte definiert.
Die Möglichen Eiegnfrequenzen lassen sich zu den entsprechenden EIgenfunktionen, den 
Kugelflächenfunktionen bestimmmen. 

\subsection{quantenmechanische Modelle}
Den zentralen aspekt der Quantenmechanik bildet die Schrödingergleichung eines Teilchens,
durch die es möglich ist für bestimmte Wellenfunktionionen Energieeiegnwerte zu den entsprechenden 
Frequenzen zu finden. Die Zentrale Erkenntnis in diesem Bereich beruht auf  der tatsache, 
das Wellenfunktionen und die dadurch beschrieben Teilchen nur für quantisierte Eigenenergien
die Schrödingergleichung erfüllen.\\

Für den Fall, dass ein Teilchen in einem Potentialtopf gefangen ist, ergeben sich
Randbedingungen an die Wellenfunktion des Teilchens. An den Rändern das Topfes muss die 
Wellenfunktion verschwinden. Dadurch sind notwendiger weise nur stehende Wellen erlaubt, 
deren Wellenlängen halbuahlige vielfache der ausdehnung des Potentialtopfes entsprechen.\\

Wird das Wasserstoffatom betrachtet, muss sowohl das Elektron, als ausch das Proton 
und die Kraft zwischen den beiden durch die Schrödingergleichung Beschreibung finden.
Zur Lösung dieses zweikörperproblems, wird die Schwerpunktsmasse angesetzt, sodass sich eine 
spezielle form der Schrödingergleihcung ergibt. Sie wird durch funtkionen gelöst die von drei 
quantenzahlen abhängen und die Kugelflächenfunktionen enthalten.\\

Die für die Lösung des Wasserstoffatoms notwendigen Quantenzahlen sind die HAuptquantenzahl,
die Drehimpulsquantenzahl und die magnetische Quantenzahl. Für ale möglichen Kombinationen Dieser
indizes lässt sich das Wasserstoffatom lösen bzw. lassen sich eigenfunktionen aufstellen.\\

Soll ein Wasserstoffmolekühl beschrieben werden, handelt es sich um ein dreikörperproblem,
da sich zwei Protonen über ein Elektron wechselwirken. Über einen Schwerpunktsanstz kommt 
auf eine Beschriebung dudrch elliptische Koordinaten, während es beim Wasserstoffatom noch 
Kugelkoordinaten waren. Die Beschriebung der EIgenfunktionen dieses Systems sind dennoch durch 
die selben Quantenzahlen chrakterisierbar.

\subsection{Modell des eindimensionalen Festkörpers}
Die Beschreibung eines Festkörpers erfolgt zunächst über die Modelldarstellung durch ein 
Gitter, durch das die periodisch angeordneten Atome des Festkörpers dargestellt werden können.
Für einen eindimensionalen Festkörper benötigt man entsprechend ein eindimenisonales GItter,
also eine Aufreihung von äquidistanten Punkten entlang einer Linie.\\

Entsprechend dieser Äquidistanzen, lassen sich periodische Potentiale beschrieben, 
welche auf Elektronen wirken. Die zugehörige Schrödingergleichung wird mit 
Blochfunktionen gelöst. Es zeigt sich, dass die Elektronen nicht beliebige Energien 
einnehmen können, denn es bildet sich eine sogennante Energielücke hreasu, in der 
keine Energieniveaus besetz werden können.\\

Diejenigen Enegiebereich, die von Elektronene besetzt werden können, werden Energiebänder 
genannt und hängen von den Paramtern des periodischen Potentials ab. Durch die 
Tatsache, dass es sich in der Realität nicht um unendlich ausgedehte Festkörper handelt, 
sorgen die entsprechenden Randbedingungen dafür, dass selbst innnerhalb eines BAndes 
nur diskrete Energiewerte besetzt werden können.\\

\subsection{Analogien zwischen den Modellen}
Es ist zu erwarten, dass sich die gemessenen akustischen Eigenfrequenzen der Wellen
welche durch Zylindrische Hohlraumresonatoren geleitet werden mit der Darstellung eines
Bändermodells aus der Festkörperphysik beschreiben lassen.\\
DAs gleiche Erwartet man für den Zusammenhang zwischen den ENergiezuständen eines 
Wasserstoffatoms und den Eigenfrequenzen eines Kugelresonators, da beide sich über die 
Kugelflächenfunktionen ausdrücken lassen.\\
Die Erweiterung auf ein Wasserstoffmolekühl, ließe sich dann entsprechend durch 
zwei HOhlraumresonatoren modellieren, da deren eigenfunktionen einen elipptisch 
geformten Raum durchlaufen und die beschriebung eines Wasserstoffmoleküls ebenfalls 
durch elliptische Koordinaten erfolgt und sich über die Kugelflächenfunktionen
lösen lässt, aus denen sich auch die eigenfunktionen der gekoppelten Kugelrsonatoren 
bidlenden lassen. 

\cite{sample}
