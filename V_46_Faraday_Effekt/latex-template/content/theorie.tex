\section{Motivation}
\label{sec:Motivation}
Über den Faraday Effekt kann die effektive Masse der Leitungselektronen
in n-dotiertem Galliumarsenid (GaAs) bestimmt werden. Dafür ist die
Kenntnis über die spezifischen Drehwinkel der Polarisationebene nötig.
Daher ist das Ziel dieses Versuches die Messung der Drehwinkel für
verschiedene Wellenlängen und daraus die Berechnung der effektiven Masse
der
Leitungselektronen in GaAs.

\section{Theorie}
\label{sec:Theorie}

\subsection{Bandstruktur und Dotierung von Halbleitern}
Ein Festkörper, dessen elektrische Leitfähigkeit sich zwischen $\SI{10e-8}{\siemens\per\centi\meter}$
und $\SI{10e4}{\siemens\per\centi\meter}$ befindet und bei steigender Temperatur 
zunimmt, wird als Halbleiter bezeichnet. 
Es wird zwischen direkten und indirekten Halbleitern unterschieden.
Diese Unterscheidung wird über die Bandstruktur des Festkörpers getroffen.
Die Bandstruktur wird im Impulsraum angegeben und beschreibt die 
Dispersionsreation von Elektronen in einem Festkörper bzw. Halbleiter. 
Das heißt, sie beschreibt den Zusammenhang zwischen der Energie und dem Wellenvektor.
Sie ist eine Darstellungsform des quantenmechanischen Energiebändermodells, welches 
die energetischen Zustände von Elektronen in einem Einkristall beschreiben kann.
Dabei wird von der quantenmechanischen Grundannahme ausgegangen, dass Elektronen, die 
im Potential eines Atomkerns gebunden sind, nur auf diskreten Energieniveaus 
(bzw. Atomorbitalen) existieren können. Werden Atome einander nah genug angenähert,
dann überschneiden sich ihre Atomorbitale, sodass sich die möglichen Energiezustände 
nach unten und oben verschieben, weil nach dem Pauli-Prinzip jeder mögliche Zustand nur 
einfach besetzt sein darf. Elektronen die sich besonders nah
am Atomkern bewegen, also auf den unteren 
Energieniveaus, sind stärker gebunden als Elektronen, die sich auf hohen 
Energieniveaus befinden und weiter vom Atomkern und seinem Potential entfernt sind. 
Diese schwächer gebundenen Elektronen können die Kernpotentiale leichter überwinden 
und können so keinem Atom mehr eindeutig zugeordnet werden. Sie können sich also von einem 
Atom zum anderen bewegen, während sie sich auf einem Energiebereich befinden, der sich über 
den gesamten Kristall erstreckt und daher auch von anderen Elektronen besetzt ist.
Dieser Energiebereich wird Valenzband genannt. Er ist das höchste besetzte Energieniveau 
eines nicht angeregten Festkörpers. Wenn Elektronen im Valenzband angeregt werden, können 
sie in das Leitungsband übergehen. Für Halbleiter befindet es sich energetisch über 
dem Valenzband. Das heißt, Elektronen müssen eine Mindestenergie aufbringen um das 
Band wechseln zukönnen. Der Energiebereich zwischen Valenz- und Leitungsband 
wird Bandlücke genannt. Bei Isolatoren ist er im Gegensatz zu Halbleitern unüberwindbar.
Bei Metallen gibt es dagegen gar keine 
Bandlücke. Das heißt, die Bänder überschneiden sich. 
Der Unterschied zwischen direkten und indirekten Halbleitern besteht darin, dass Elektronen 
bei einem direkten Halbleiter lediglich ihre Energie hinreichend erhöhen müssen, um in 
das Leitungsband überzugehen. Indirekte Halbeiter müssen zusätzlich eine Änderung 
des Elektronimpulses realisieren, damit ein entsprechender Übergang stattfinden kann.
Um die Leitfähigkeit eines Halbleiters zu erhöhen können Fremdatome in den Einkristall 
eingebracht werden. Dieser Vorgang wird Dotierung genannt.
Es wird zwischen n- und p-Dotierung unterschieden. Bei der n-Dotierung (negativ) werden 
Elektronendonatoren in den Halbleiter eingefügt. Dabei handelt es sich um Fremdatome, die
Elektronen zur Verfügung stellen. Dadurch, dass die Elektronenanzahl erhöht wird, 
steigt die Leitfähigkeit durch Elektronentransport.
Bei der p-Dotierung (positiv) werden hingegen Elektronenakzeptoren hinzugefügt. 
Dabei handelt es sich um Fremdatome, die Elektronenlöcher zur Verfügung stellen. 
Dadurch wird umgekehrt die Leitfähigkeit durch Löcherleitung erhöht. 
Generell bietet es sich an für die Dotierung Fremdatome zu verwenden, deren 
Energieniveaus sich im Bereich der Bandlücke des Trägermaterial befinden, da so die 
notwendige Energiehöhe zum Übergang zwischen Valenz- und Leitungsband verringert wird, 
weil die Fremdatome Zwischenniveaus erzeugen.

\subsection{Definition und Eigenschaften der effektiven Masse}
Die Masse eines Elektrons in einem Kristall wird durch die effektive Masse beschrieben, 
da sie von der tatsächlichen Masse abweichen kann. 
Für ein Elektron in einem Einkristall gilt nach dem zweiten Newton´schen Gesetz die 
Bewegungsgleichung:

\begin{equation}
    a = \frac{1}{\hbar} \frac{d^2\epsilon}{dk^2}\cdot eE.
    \label{eq1}
\end{equation}

In Gleichung \ref{eq1} bezeichnet $a$ die Beschleunigung, $\epsilon$ die 
Dispersionsrelation, $k$ die Wellenzahl, $e$ die Elektronenladung und $E$ das
äußere elektrische Feld.
Im Gegensatz zu einem freien Elektron im Vakuum ist die Dispersionsrelation eines 
Elektrons im Leitungsband eines Festkörpers im allgemeinen nicht quadratisch
sondern geschwindigkeitsabhängig. 

\begin{equation}
    m^* = \hbar^2 \left( \frac{d^2\epsilon}{dk^2} \right)^{-1}
    \label{eq2}
\end{equation}

Im Bereich von Extremstellen kann die Dispersionrelation jedoch quadratisch angenähert 
werden, sodass das Modell einer effektiven Masse (Gleichung \ref{eq2}) in diesen 
Energiebereichen besonders nützlich ist.

\subsection{Polarisation und zirkulare Doppelbrechung}
Unter Polarisation wird in der Physik im Bezug auf Wellen die Richtung der 
Schwingungsamplitude im Vergleich zur Ausbreitungsrichtung der Gesamtwelle 
verstanden. 
Longitudinale Wellen sind nicht polarisiert bzw. longitudinal polarisiert.
Transversale Wellen können linear, zirkular oder elliptisch polarisiert sein. 
Bei der linearen Polarisation ist die Schwingungrichtung konstant, während die 
Auslenkung aus der Ruhelage periodisch ihren Betrag und ihr Vorzeichen ändert.
Sie kann als überlagerung zweier zirkular Polarisierter Wellen interpretiert werden
Bei der zirkularen Polarisation ist hingegen die Auslenkung konstant, während sich die 
Schwingungsrichtung mit einer konstanten Winkelgeschwindigkeit ändert.
Unter elliptischer Polarisation wird eine Mischform der linearen und zirkularen 
Polarisation verstanden.
Optisch anisotrope Medien haben einen Brechungsindex, welcher von der 
Ausbreitungsrichtung und Polarisation abhängt. 
Wenn ein Lichtstrahl auf ein entsprechendes Medium trifft, wird er in zwei 
senkrecht zueinander polarisierte Teilbündel aufgespalten.
Das heißt, wenn ein Medium die Eigenschaft aufweist, Licht in einen 
links- und rechts-zirkular polarisierten Anteil zu spalten, dann wird von 
zirkularer Doppelbrechung gesprochen.
Aufgrund der verschiedenen Brechungsindices ($n_r$, $n_l$) der Teilstrahlen, setzt sich der 
Drehwinkel $\theta$ auch aus den zugehörigen Phasengeschwindigkeiten $\nu_r$ und $\nu_l$ 
für den links- und rechts-polarisierten Strahl zusammen.

\begin{equation}
    \theta = \frac{L\omega}{2} \left(\frac{1}{\nu_r} - \frac{1}{\nu_l} \right) = \frac{L\omega}{2c} (n_r - n_l )
    \label{eq3}
\end{equation}

In Gleichung \ref{eq3} entspricht $\omega$ der Kreisfrequen des einfalenden Lichtstrahles, 
$L$ der Länge des Mediums und $c$ der Lichtgeschwindigkeit im Vakuum.
Für isotrope Medien ergibt sich der Drehwinkel $\theta$ mit dem Brechungsindex $n$ und 
der dielektrischen Suszeptibilität $\chi$ des Mediums über Gleichung \ref{eq4}.

\begin{equation}
    \theta = \frac{L\omega}{2cn} \chi
    \label{eq4}
\end{equation}

\subsection{Faraday Effekt}
Unter dem Faraday-Effekt wird die Drehung der Polarisationsebene von linear 
polarisierter
elektromagnetischer Strahlung unter dem Einfluss eines Magnetfeldes verstanden.
In einem anisotropen Material unterscheiden sich die Brechungsindices der 
zirkular polarisierten Anteile, wenn ein Magnetfeld mit Feldlinien, welche 
parallel zur Ausbreitungsrichtung sind, angelegt ist. Somit 
unterscheiden sich auch die Wellenlängen für den im Uhrzeigersinn polarisierten 
Anteil ($\lambda_{r}$) und den Anteil, der entgegen des Uhrzeigersinns polarisiert 
ist ($\lambda_{r}$).

\begin{align}
    \lambda_{r} = \frac{c}{f\cdot n_{r}}\\
    \lambda_{l} = \frac{c}{f\cdot n_{l}}
    \label{eq5}
\end{align}

In den Gleichungen \ref{eq5} entspricht $f$ der inversen Schwingungsdauer.
Während einer Schwingungsdauer wird die Polarisatinosebene also um den Winkel 
$\delta \beta$ weiter gedreht. Der Zusammenhang zu den Brechungsindices wird aus 
Gleichung \ref{eq6} ersichtlich.

\begin{equation}
    \delta \beta = \pi \left( \frac{n_r}{n_l} -1 \right)
    \label{eq6}
\end{equation}

Der Drehwinkel kann allerdings auch mit Hilfe der Verdet-Konstante $V$, die sich 
über den Zusammenhang in Gleichung \ref{eq7} ergibt, berechnet werden.
Dabei entspricht $e$ der Elementarladung, $m_e$ der Elektronenmasse und 
$\frac{dn}{d\lambda}$ der 
Dispersion des betrachteten Materials.

\begin{equation}
    V = \frac{-e}{m_e} \frac{\lambda}{2c} \frac{dn}{d\lambda}
    \label{eq7}
\end{equation}

Der Drehwinkel wird dann über Gleichung \ref{eq8} bestimmt, wobei 
$B$ der magnetischen Flussdichte und $d$ der zurückgelegten Strecke 
im Medium entspricht.

\begin{equation}
    \beta = V \cdot d \cdot B
    \label{eq8}
\end{equation}

Dieses Kapitel wurde mit den Literaturquellen
\cite{sample}, \cite{demt}, \cite{exp}, \cite{switch} und \cite{eng}.

