\section{Diskussion}
\label{sec:Diskussion}
Bei diesem Versuch wird die effektive Masse der Leitungselektronen von Galliumarsenid bestimmt.
Der Wert wird mit zwei Versuchsreihen bestimmt. 
Beide Versuchsreihen werden mit dem Literaturwert \cite{LitWert} verglichen.
\begin{align*}
    m^*_1 =& \SI{3.2(6)e-32}{\kilo\gram}\\
    m^*_2 =& \SI{4.9(1)e-32}{\kilo\gram}\\
    m^*_{\text{lit}}=& \SI{6.103e-32}{\kilo\gram}
\end{align*}
Der Literaturwert liegt nicht in der Unsicherheit der beiden Messungen, allerdings ist die Größenordnung richtig.
Die Abweichung lässt sich dadurch erklären, dass manche Messwerte von der Fitfunktion abweichen. Insgesamt kann gesagt werden, 
das mit diesem Versuch der Faraday-Effekt untersucht werden kann.
