\section{Durchführung}
\label{sec:Durchführung}

Der Aufbau dieses Versuches besteht neben der Strahlungsquelle aus einer Plattform auf der
das zu untersuchende Medium positioniert werden kann. 
Strahlung, die das Medium durchqueren konnte, trifft auf einen Szintillationsdetektor.
Er besteht aus einem Material, dem Szintillator, dessen Moleküle durch die 
Gammastrahlung angeregt werden können und diese Energie in Form von Photonen abgeben.
Die Intensität dieser Photonen ist proportional zur Intensität der Eingangsstrahlung .
Daher befindet sich hinter dem Szintillator ein 
Photomultiplier, der durch ein an einer Photodiode einfallendes Photon, ein Elektron 
aussendet, welches aufgrund von Folgeelektroden beschleunigt wird und dort jeweils weitere Elektronenlawinen 
auslösen kann. Diese werden dann als Stromsignal messbar. Das Signal wird daraufhin in 
einen Multikanalalaysator gleitet, der das Signal einem Kanal zuordnet, sodass Spannungspulse
(vorgeschalteter Diskriminator wandelt Strom in Spannung um) 
pro Kanal gezählt werden.

\begin{figure}
	\centering
	\includegraphics[width=0.6\textwidth]{figure/Aufbau.pdf}
	\caption{Auf dieser Abbildung ist der Versuchsaufbau dargestellt \cite{sample}.}
	\label{abb3}
\end{figure}

Die erste Messung wird ohne Medium durchgeführt. Die Intensität wird  für einen Zeitraum von 
$\SI{300}{\second}$
aufgenommen, damit die relative statistische Unsicherheit dieses Zählexperiments auf
unter 3\% sinkt.
Danach wird ein leerer Würfel aus einer Aluminiumhülle vermessen. Dafür wird 
zunächst ein Spektrum aufgenommen, bei dem die Gammastrahlung senkrecht auf einer 
Würfelfläche ein- und austritt. Danach wird der Würfel gedreht, sodass der Strahl diagonal 
durch den Würfel verläuft und daraufhin so, dass der Strahl die Nebendiagonale des 
Würfels durchquert. Die selben Messungen werden danach für zwei weitere $\SI{3}{\centi\meter} \times \SI{3}{\centi\meter}$-Würfel 
in einer Aluminiumhülle durchgeführt. Diese Würfel bestehen aus einem Innenmaterial, dass 
über die Bestimmung des Absorptionkoeffizienten ermittelt werden soll. 

\newpage

Für die letzte Messreihe wird ein Würfel vermessen, der  
$\SI{1}{\centi\meter} \times \SI{1}{\centi\meter}$-Teilwürfel aus unterschiedlichem Material
enthält. Um die Teilwürfel der mittleren Ebene korrekt beschreiben zu können, müssen 
zwölf Teilmessungen durchgeführt werden.
In Abbildung \ref{abb2} sind die Verläufe der Strahlung dargestellt.

\begin{figure}
	\centering
	\includegraphics[width=0.55\textwidth]{figure/Intensitäten.pdf}
	\caption{Auf dieser Abbildung sind die Verläufe der Strahlung für jede Teilmessung 
	dargestellt \cite{1}.}
	\label{abb2}
\end{figure}

Mit Hilfe der gemessenen Intensitäten können daraufhin auch für den letzten Würfel die 
Materialien bestimmt werden. Dafür muss die Matrix $A$ (siehe Kapitel \ref{sec:Absorp}) 
aufgestellt werden. Da die Teilwürfel alle Seitenlängen von 
$\SI{1}{\centi\meter}$ haben, ist $A$ durch die Matrix 

\begin{equation}
	A = 	 
	\begin{bmatrix}
		0 		 & \sqrt{2}  & 0 		& \sqrt{2} & 0 		  & 0 		 & 0 		& 0 	   & 0\\
		0 		 & 0 		 & \sqrt{2} & 0 	   & \sqrt{2} & 0 		 & \sqrt{2} & 0 	   & 0   \\
		0 		 & 0 		 & 0 		& 0 	   & 0 		  & \sqrt{2} & 0 		& \sqrt{2} & 0 \\
		1 		 & 1 		 & 1 		& 0 	   & 0 		  & 0        & 0 		& 0 	   & 0  \\
		0 		 & 0 		 & 0 		& 1 	   & 1		  & 1		 & 0		& 0		   & 0  \\
		0 		 & 0 		 & 0 		& 0		   & 0		  & 0		 & 1		& 1		   & 1  \\
		0 		 & \sqrt{2}  & 0 		& 0		   & 0		  & \sqrt{2} & 0		& 0		   & 0 \\
		\sqrt{2} & 0 		 & 0 		& 0		   & \sqrt{2} & 0		 & 0		& 0		   & \sqrt{2} \\
		0 		 & 0 		 & 0 		& \sqrt{2} & 0		  & 0		 & 0		& \sqrt{2} & 0  \\ 
		0 		 & 0 		 & 1 		& 0		   & 0		  & 1		 & 0		& 0		   & 1\\
		0 		 & 1 		 & 0 		& 0		   & 1		  & 0		 & 0		& 1		   & 0  \\
		1 		 & 0 		 & 0 		& 1		   & 0		  & 0		 & 1		& 0		   & 0  
	\end{bmatrix}
\end{equation}

gegeben \cite{sample}.
\newpage