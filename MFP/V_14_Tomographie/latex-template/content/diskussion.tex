\section{Diskussion}
\label{sec:Diskussion}

Allgemein muss gesagt werden, dass der Multikanalanalysator zeitweise nicht gut funktioniert hat. 
Der Peakinhalt für die $I_0$-Messung lag zuerst bei $\num{550}$-$\num{1450}$ bei ähnlich großen Messzeiten. Nach der $I_0$-Messung 
waren die Werte allerdings auf $\num{10000}$-$\num{25000}$ gestiegen. Dadurch waren die Absorptionskoeffizienten negativ, was unphysikalisch ist.
Daher wurde am Ende der Durchführeung erneut die Messreihe zu $I_0$ durchgeführt und die Werte haben sich ca. verzehnfacht. Ein anderer Beweis für 
die Fehlerhaftigkeit des Messgerätes, war das Auftreten von negativen Peakinhalten, diese nahmen bei erneutem Messen jedoch 
sinnvolle Werte an.

\subsection{Bestimmung der Absorptionskoeffizienten der Würfel 2 und 3}

Die Würfel 2 und 3 bestehen jeweils aus nur einem Material. Laut Versuchsanleitung besteht Würfel 4 nur aus den 
Materialien von Würfel 2 und 3. Würfel 2 besteht laut Messergebnissen aus Eisen und Würfel 3 aus Delrin. Da Eisen eine 
höhere Dichte hat als Delrin, muss Würfel 2 schwerer sein als Würfel 3. Das ist auch der Fall, allerdings konnte durch 
ein Loch in der Aluminiumhülle ein messingfarbenes Metall entdeckt werden. Da auch Messing schwerer wäre als Würfel 
3, könnte dieses Material auch in Frage kommen. Da die Literaturwerte von Eisen und Messing sich nur um $\SI{0.032}{1\per\centi\meter}$
unterscheiden, können diese Materialien mit diesem Gerät nicht unterschieden werden. Die Farbe spricht jedoch eher für Messing als für Eisen.
Da Würfel 3 recht leicht war, ist das ein Hinweis darauf, dass es sich um Aluminium oder Delrin handelt. 
Bei der Bestimmung des Absorptionskoeffizientens über die Diagonalen kam ein negativer Wert heraus, dass 
wird daran liegen, das der Wert von $I_0$ in der Diagonalen zu niedrig bestimmt wurde. Daher ist der $\frac{I_0}{I_{D}}<1$ und 
der Logarithmus negativ. Die nicht negativen Werte sind sehr niedrig. Das Material wird als Delrin bestimmt, da es das Material
mit dem niedrigsten Absorptionskoeffizienten ist. Auch hier konnte über ein Loch in der Aluminiumhülle 
ein delrinfarbiges (milchigweißes) Material beobachtet werden.

\subsection{Bestimmung der Absorptionskoeffizienten der Würfel in Würfel 4}

Die meisten Elementarwürfel des Würfels 4 wurden mit Delrin identifiziert. Dafür spricht, dass Würfel 4 leichter war 
als Würfel 2 und schwerer als Würfel 3. Dagegen spricht, dass viele der kleinsten Absorptionskoeffizienten 
um mehrere Größenordnungen kleiner als der Literaturwert von 
Delrin sind.
Mit der Information, dass Würfel 4 nur aus Materialien von Würfel 2 und 3 besteht und mit den Ergebnissen aus den Messreihen
von Würfel 2 und 3, kann die Vermutung aufgestellt werden, dass Würfel 4 aus Delrin und Messing besteht. 
\newpage