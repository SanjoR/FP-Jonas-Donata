\section{Durchführung}
\label{sec:Durchführung}

\subsection{Versuchsaufbau}


\subsection{1. Messreihe}

Zu Beginn des Versuches wird die Kaliumbromidplatte, welche sich zwischen den 
Plattenkondensatoren befindet auf 47°C erhitzt.
Danach wird an die Kondensatorplatten eine Spannung von 950V angelegt. 
Dabei richten sich die Dipole im Kristall entlang der E-Feldlinien aus.
Der Aufbau wird für 900s in diesem Zustand beibehalten, um eine 
Ausrichtung möglichst vieler Dipole zu realisieren.
Nach dieser Zeit wird die Probe mit flüssigem Stickstoff auf unter -60°C abgekült.
Wenn diese Temperatur erreicht ist, werde die Kondesnatorplatten für 5 min kurzgeschlossen
um ein vollständiges abflachen des EFeldes zu ermöglichen.
Jetzt wird mit einer konstanten Heizrate von 2°C pro Minute ausgehend von 
-60^C erhitzt bis eine Temperatur von 57°C überschritten wird. 
Währenddessen werden alle 30s die Wertepaare Temperatur und Depolarisationsstrom 
aufgenommen.


\subsection{2. Messreihe}

Für den zweiten Messdurchgang wird die Probe erneut einem elektrischen Feld ausgesetzt,
welches für 15 min anliegt. Dadurch soll wie im ersten Durchflauf eine Maximalanzahl,
der am EFeld ausgerichteten Dipole erzeugt werden.
Danach wird der Kondensator mit der Probe im Inneren durch flüssigen Stickstoff auf 
unter -40°C abgekült. Bevor die Messwerte aufgenommen werden können, muss wieder 
eine Wartezeit von 5 min eingehalten werden, währenddessen der Kondensator 
kurzgeschlossen wird, sodass das Eeld komplett abflachen kann.
Danach wird die Temperatur wieder konstant erhöht, wobei minütlich die Wertepaare
Temperatur und Depolarisationsstrom aufgenommen werden.