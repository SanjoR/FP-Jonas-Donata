\section{Auswertung}
\label{sec:auswertung}
In diesem Versuch werden zwei Temperatur-Strom-Kurven mit unterschiedlichen Heizraten $b$ analysiert.
Messreihe A hat eine Heizrate von $b_{\text{A}}=\SI{2}{\kelvin\per\minute}$ und es wurde in dem Temperaturbereich
$T\in\left[\SI{213.1}{\kelvin},\SI{330.3}{\kelvin}\right]$ gemessen. Für Messreihe B beträgt die Heizrate
$b_{\text{B}}=\SI{1}{\kelvin\per\minute}$ und der vermessene Temperaturbereich ist $T\in\left[\SI{232.7}{\kelvin},\SI{289.6}{\kelvin}\right]$.
In beiden Fällen ist Kaliumbromid $\ce{KBr}$ der verwendete Kristall. Die Messdaten sind in Tabellen \ref{tab:Messdaten_roh}
abgebildet. 
\newpage
\begin{table}
    \centering
    \caption{Temperatur und Strom für Messreihe A und B, bei A wurden halbminütig und bei B jede Minute ein Datenpaar aufgezeichnet.}
    \label{tab:Messdaten_roh}
    \begin{tabular}{c c c c c}
        \toprule
        &\multicolumn{2}{c}{Messreihe A}&\multicolumn{2}{c}{Messreihe B}\\
        Zeit $t \,/\,\SI{}{\minute}$&Temperatur $T \,/\,\SI{}{\kelvin}$ &Strom $I \,/\,\SI{e-11}{\ampere}$&Temperatur $T \,/\,\SI{}{\kelvin}$ &Strom $I \,/\,\SI{e-11}{\ampere}$\\
        \midrule
        $\num{0.0}$&$\num{213.1}$&$\num{0.08}$&$\num{232.7}$&$\num{0.1}$\\
        $\num{0.5}$&$\num{215.3}$&$\num{0.1}$&-&-\\
        $\num{1.0}$&$\num{217.3}$&$\num{0.08}$&$\num{234.5}$&$\num{0.13}$\\
        $\num{1.5}$&$\num{218.1}$&$\num{0.09}$&-&-\\
        $\num{2.0}$&$\num{219.0}$&$\num{0.08}$&$\num{235.5}$&$\num{0.15}$\\
        $\num{2.5}$&$\num{219.8}$&$\num{0.09}$&-&-\\
        $\num{3.0}$&$\num{220.4}$&$\num{0.08}$&$\num{236.3}$&$\num{0.16}$\\
        $\num{3.5}$&$\num{221.4}$&$\num{0.07}$&-&-\\
        $\num{4.0}$&$\num{222.3}$&$\num{0.07}$&$\num{237.3}$&$\num{0.19}$\\
        $\num{4.5}$&$\num{223.1}$&$\num{0.07}$&-&-\\
        $\num{5.0}$&$\num{224.6}$&$\num{0.07}$&$\num{238.4}$&$\num{0.21}$\\
        $\num{5.5}$&$\num{225.5}$&$\num{0.07}$&-&-\\
        $\num{6.0}$&$\num{226.2}$&$\num{0.07}$&$\num{239.6}$&$\num{0.25}$\\
        $\num{6.5}$&$\num{227.1}$&$\num{0.08}$&-&-\\
        $\num{7.0}$&$\num{228.0}$&$\num{0.08}$&$\num{240.6}$&$\num{0.28}$\\
        $\num{7.5}$&$\num{228.9}$&$\num{0.08}$&-&-\\
        $\num{8.0}$&$\num{230.0}$&$\num{0.09}$&$\num{241.5}$&$\num{0.31}$\\
        $\num{8.5}$&$\num{231.2}$&$\num{0.08}$&-&-\\
        $\num{9.0}$&$\num{231.2}$&$\num{0.08}$&$\num{242.4}$&$\num{0.35}$\\
        $\num{9.5}$&$\num{232.1}$&$\num{0.11}$&-&-\\
        $\num{10.0}$&$\num{233.0}$&$\num{0.12}$&$\num{243.4}$&$\num{0.39}$\\
        $\num{10.5}$&$\num{233.9}$&$\num{0.13}$&-&-\\
        $\num{11.0}$&$\num{234.7}$&$\num{0.14}$&$\num{244.4}$&$\num{0.44}$\\
        $\num{11.5}$&$\num{235.5}$&$\num{0.15}$&-&-\\
        $\num{12.0}$&$\num{236.2}$&$\num{0.16}$&$\num{245.4}$&$\num{0.49}$\\
        $\num{12.5}$&$\num{237.0}$&$\num{0.16}$&-&-\\
        $\num{13.0}$&$\num{238.0}$&$\num{0.2}$&$\num{246.4}$&$\num{0.54}$\\
        $\num{13.5}$&$\num{239.0}$&$\num{0.22}$&-&-\\
        $\num{14.0}$&$\num{239.9}$&$\num{0.25}$&$\num{247.3}$&$\num{0.59}$\\
        $\num{14.5}$&$\num{240.7}$&$\num{0.28}$&-&-\\
        $\num{15.0}$&$\num{241.5}$&$\num{0.31}$&$\num{248.2}$&$\num{0.63}$\\
        $\num{15.5}$&$\num{242.5}$&$\num{0.35}$&-&-\\
        $\num{16.0}$&$\num{243.8}$&$\num{0.41}$&$\num{249.1}$&$\num{0.67}$\\
        $\num{16.5}$&$\num{244.8}$&$\num{0.46}$&-&-\\
        $\num{17.0}$&$\num{245.7}$&$\num{0.53}$&$\num{249.9}$&$\num{0.67}$\\
        $\num{17.5}$&$\num{246.5}$&$\num{0.59}$&-&-\\
        $\num{18.0}$&$\num{247.4}$&$\num{0.65}$&$\num{250.7}$&$\num{0.73}$\\
        $\num{18.5}$&$\num{248.4}$&$\num{0.72}$&-&-\\
        $\num{19.0}$&$\num{249.5}$&$\num{0.82}$&$\num{251.5}$&$\num{0.75}$\\
        $\num{19.5}$&$\num{250.5}$&$\num{0.92}$&-&-\\
        $\num{20.0}$&$\num{251.5}$&$\num{1.05}$&$\num{252.3}$&$\num{0.75}$\\
        $\num{20.5}$&$\num{252.4}$&$\num{1.1}$&-&-\\
        $\num{21.0}$&$\num{253.2}$&$\num{1.2}$&$\num{253.2}$&$\num{0.76}$\\
        $\num{21.5}$&$\num{254.0}$&$\num{1.3}$&-&-\\
        $\num{22.0}$&$\num{254.9}$&$\num{1.35}$&$\num{254.2}$&$\num{0.76}$\\
        $\num{22.5}$&$\num{255.9}$&$\num{1.45}$&-&-\\
        $\num{23.0}$&$\num{257.1}$&$\num{1.55}$&$\num{255.2}$&$\num{0.75}$\\
        $\num{23.5}$&$\num{258.2}$&$\num{1.65}$&-&-\\
        $\num{24.0}$&$\num{259.2}$&$\num{1.65}$&$\num{256.5}$&$\num{0.78}$\\
        $\num{24.5}$&$\num{259.8}$&$\num{1.66}$&-&-\\  
        \bottomrule                                                                                                                                                                                                                                                                                                                                                                                                                                                                                                                                                                                                                                                      
    \end{tabular}
\end{table}
\begin{table}
    \centering
    \begin{tabular}{c c c c c}
        \toprule
&\multicolumn{2}{c}{Messreihe A}&\multicolumn{2}{c}{Messreihe B}\\
Zeit $t \,/\,\SI{}{\minute}$&Temperatur $T \,/\,\SI{}{\kelvin}$ &Strom $I \,/\,\SI{e-11}{\ampere}$&Temperatur $T \,/\,\SI{}{\kelvin}$ &Strom $I \,/\,\SI{e-11}{\ampere}$\\
\midrule
        $\num{25.0}$&$\num{260.5}$&$\num{1.55}$&$\num{257.5}$&$\num{0.75}$\\
        $\num{25.5}$&$\num{261.2}$&$\num{1.5}$&-&-\\
        $\num{26.0}$&$\num{262.0}$&$\num{1.4}$&$\num{258.3}$&$\num{0.69}$\\
        $\num{26.5}$&$\num{262.8}$&$\num{1.3}$&-&-\\
        $\num{27.0}$&$\num{263.7}$&$\num{1.25}$&$\num{259.0}$&$\num{0.62}$\\
        $\num{27.5}$&$\num{264.7}$&$\num{1.15}$&-&-\\
        $\num{28.0}$&$\num{265.6}$&$\num{1.1}$&$\num{260.0}$&$\num{0.59}$\\
        $\num{28.5}$&$\num{266.6}$&$\num{1.0}$&-&-\\
        $\num{29.0}$&$\num{267.7}$&$\num{0.95}$&$\num{261.3}$&$\num{0.55}$\\
        $\num{29.5}$&$\num{268.8}$&$\num{0.86}$&-&-\\
        $\num{30.0}$&$\num{269.7}$&$\num{0.8}$&$\num{262.2}$&$\num{0.49}$\\
        $\num{30.5}$&$\num{270.5}$&$\num{0.73}$&-&-\\
        $\num{31.0}$&$\num{271.2}$&$\num{0.67}$&$\num{262.8}$&$\num{0.43}$\\
        $\num{31.5}$&$\num{272.3}$&$\num{0.64}$&-&-\\
        $\num{32.0}$&$\num{273.2}$&$\num{0.63}$&$\num{263.5}$&$\num{0.38}$\\
        $\num{32.5}$&$\num{274.3}$&$\num{0.64}$&-&-\\
        $\num{33.0}$&$\num{275.4}$&$\num{0.65}$&$\num{264.5}$&$\num{0.36}$\\
        $\num{33.5}$&$\num{276.6}$&$\num{0.67}$&-&-\\
        $\num{34.0}$&$\num{277.4}$&$\num{0.68}$&$\num{265.5}$&$\num{0.34}$\\
        $\num{34.5}$&$\num{278.0}$&$\num{0.67}$&-&-\\
        $\num{35.0}$&$\num{278.7}$&$\num{0.67}$&$\num{266.2}$&$\num{0.32}$\\
        $\num{35.5}$&$\num{279.5}$&$\num{0.69}$&-&-\\
        $\num{36.0}$&$\num{280.6}$&$\num{0.71}$&$\num{267.0}$&$\num{0.31}$\\
        $\num{36.5}$&$\num{281.7}$&$\num{0.75}$&-&-\\
        $\num{37.0}$&$\num{282.7}$&$\num{0.78}$&$\num{268.0}$&$\num{0.31}$\\
        $\num{37.5}$&$\num{283.6}$&$\num{0.8}$&-&-\\
        $\num{38.0}$&$\num{284.3}$&$\num{0.81}$&$\num{269.0}$&$\num{0.32}$\\
        $\num{38.5}$&$\num{285.1}$&$\num{0.82}$&-&-\\
        $\num{39.0}$&$\num{286.0}$&$\num{0.84}$&$\num{269.6}$&$\num{0.32}$\\
        $\num{39.5}$&$\num{286.9}$&$\num{0.86}$&-&-\\
        $\num{40.0}$&$\num{287.9}$&$\num{0.89}$&$\num{270.4}$&$\num{0.32}$\\
        $\num{40.5}$&$\num{289.0}$&$\num{0.94}$&-&-\\
        $\num{41.0}$&$\num{290.1}$&$\num{1.0}$&$\num{271.5}$&$\num{0.335}$\\
        $\num{41.5}$&$\num{291.0}$&$\num{1.0}$&-&-\\
        $\num{42.0}$&$\num{291.7}$&$\num{1.0}$&$\num{272.3}$&$\num{0.34}$\\
        $\num{42.5}$&$\num{292.2}$&$\num{1.05}$&-&-\\
        $\num{43.0}$&$\num{293.2}$&$\num{1.05}$&$\num{273.2}$&$\num{0.35}$\\
        $\num{43.5}$&$\num{294.2}$&$\num{1.05}$&-&-\\
        $\num{44.0}$&$\num{295.1}$&$\num{1.11}$&$\num{274.2}$&$\num{0.365}$\\
        $\num{44.5}$&$\num{296.1}$&$\num{1.11}$&-&-\\
        $\num{45.0}$&$\num{297.1}$&$\num{1.15}$&$\num{275.2}$&$\num{0.385}$\\
        $\num{45.5}$&$\num{297.8}$&$\num{1.2}$&-&-\\
        $\num{46.0}$&$\num{298.6}$&$\num{1.25}$&$\num{276.2}$&$\num{0.4}$\\
        $\num{46.5}$&$\num{299.4}$&$\num{1.3}$&-&-\\
        $\num{47.0}$&$\num{300.3}$&$\num{1.35}$&$\num{277.1}$&$\num{0.41}$\\
        $\num{47.5}$&$\num{301.3}$&$\num{1.4}$&-&-\\
        $\num{48.0}$&$\num{302.3}$&$\num{1.5}$&$\num{278.0}$&$\num{0.425}$\\
        $\num{48.5}$&$\num{303.3}$&$\num{1.6}$&-&-\\
        $\num{49.0}$&$\num{304.3}$&$\num{1.7}$&$\num{279.0}$&$\num{0.442}$\\
        $\num{49.5}$&$\num{305.0}$&$\num{1.75}$&-&-\\
        \bottomrule
    \end{tabular}
\end{table}
\begin{table}
    \centering
    \begin{tabular}{c c c c c}
        \toprule
&\multicolumn{2}{c}{Messreihe A}&\multicolumn{2}{c}{Messreihe B}\\
Zeit $t \,/\,\SI{}{\minute}$&Temperatur $T \,/\,\SI{}{\kelvin}$ &Strom $I \,/\,\SI{e-11}{\ampere}$&Temperatur $T \,/\,\SI{}{\kelvin}$ &Strom $I \,/\,\SI{e-11}{\ampere}$\\
\midrule
        $\num{50.0}$&$\num{305.9}$&$\num{1.75}$&$\num{280.0}$&$\num{0.46}$\\
        $\num{50.5}$&$\num{306.6}$&$\num{1.95}$&-&-\\
        $\num{51.0}$&$\num{307.5}$&$\num{2.0}$&$\num{280.9}$&$\num{0.47}$\\
        $\num{51.5}$&$\num{308.4}$&$\num{2.1}$&-&-\\
        $\num{52.0}$&$\num{309.4}$&$\num{2.25}$&$\num{281.9}$&$\num{0.49}$\\
        $\num{52.5}$&$\num{310.3}$&$\num{2.35}$&-&-\\
        $\num{53.0}$&$\num{311.3}$&$\num{2.5}$&$\num{282.9}$&$\num{0.515}$\\
        $\num{53.5}$&$\num{312.2}$&$\num{2.6}$&-&-\\
        $\num{54.0}$&$\num{313.2}$&$\num{2.75}$&$\num{283.9}$&$\num{0.525}$\\
        $\num{54.5}$&$\num{314.2}$&$\num{2.85}$&-&-\\
        $\num{55.0}$&$\num{315.1}$&$\num{2.95}$&$\num{284.9}$&$\num{0.55}$\\
        $\num{55.5}$&$\num{316.0}$&$\num{3.1}$&-&-\\
        $\num{56.0}$&$\num{316.8}$&$\num{3.1}$&$\num{286.0}$&$\num{0.575}$\\
        $\num{56.5}$&$\num{317.8}$&$\num{3.2}$&-&-\\
        $\num{57.0}$&$\num{318.7}$&$\num{3.2}$&$\num{287.1}$&$\num{0.615}$\\
        $\num{57.5}$&$\num{319.7}$&$\num{3.2}$&-&-\\
        $\num{58.0}$&$\num{320.5}$&$\num{3.2}$&$\num{288.7}$&$\num{0.63}$\\
        $\num{58.5}$&$\num{321.3}$&$\num{3.1}$&-&-\\
        $\num{59.0}$&$\num{322.2}$&$\num{3.1}$&$\num{289.6}$&$\num{0.645}$\\
        $\num{59.5}$&$\num{323.0}$&$\num{3.0}$&-&-\\
        $\num{60.0}$&$\num{323.9}$&$\num{2.9}$&-&-\\
        $\num{60.5}$&$\num{324.7}$&$\num{2.8}$&-&-\\
        $\num{61.0}$&$\num{325.5}$&$\num{2.6}$&-&-\\
        $\num{61.5}$&$\num{326.3}$&$\num{2.5}$&-&-\\
        $\num{62.0}$&$\num{327.1}$&$\num{2.3}$&-&-\\
        $\num{62.5}$&$\num{327.9}$&$\num{2.15}$&-&-\\
        $\num{63.0}$&$\num{328.8}$&$\num{2.0}$&-&-\\
        $\num{63.5}$&$\num{329.5}$&$\num{1.8}$&-&-\\
        $\num{64.0}$&$\num{330.3}$&$\num{1.65}$&-&-\\
        \bottomrule
    \end{tabular}
\end{table}










































\section{Auswertung,Kevin}
Es werden in diesem Versuch zwei Messreihen mit unterschiedlicher Heizrate
$b$ durchgeführt. Zunächst müssen die Messwerte von Untergründen bereinigt
werden, auf die im Folgenden eingegangen wird.
Für die erste Messreihe -- im Folgenden mit dem Index A gekennzeichnet --
gilt $b_\text{A} = \SI{2}{\kelvin\per\minute}$, für die Zweite, mit B
gekennzeichnete Messreihe gilt $b_\text{B} = \SI{4}{\kelvin\per\minute}$.

Alle Fehler in dieser Auswertung werden -- wenn nicht anders beschrieben --
mittels Gaußscher Fehlerfortpflanzungen mit Hilfe des \texttt{python}-Paketes
\texttt{uncertainties} \cite{py-uncertainties} berechnet.

Zur Beschreibung des oben genannten Untergrundes wird bei Messreihe A eine
exponentielle Funktion $I_\text{bkg,A}(T)$ und bei Messreihe B eine lineare
Funktion
$I_\text{bkg,B}(T)$ von den Daten abgezogen. Die Variablen der Funktionen werden
durch Fit an einen Teil der Daten bestimmt, der in Abbildungen
\ref{fig:data-a} und \ref{fig:data-b} markiert ist.
Die Fitfunktionen haben die Gestalt
\begin{align*}
    I_\text{bkg,A}(T) &= A_1 e^{B_1 (T-T_0)} + C_1\,,\\
    I_\text{bkg,B}(T) &= A_2\cdot T + B_2\,.
\end{align*}
Das Resultat, sowie der Fit sind ebenfalls in oben genannter Abbildung
dargestellt und die entsprechenden Daten in den Tabellen \ref{tab:data1} und
\ref{tab:data2} aufgeführt.
Im weiteren Verlauf wird dabei jeweils ein offset $I_\text{off}$ von den Daten
abgezogen. Dieser beträgt für Datensatz A $I_\text{off} = \SI{1.46}{\pico\ampere}$
und für Datensatz B $I_\text{off} = \SI{8.1}{\pico\ampere}$.
Bei der ersten Messreihe ist auffällig, dass nur ein einzelner, flacher Peak
zu erkennen ist. Dies macht die spätere Auswertung extrem schwierig.
Aus diesem Grund werden die Ergebnisse der erste Messreihe im weiteren
Verlauf nicht betrachtet.

\subsection{Approximative Ausgleichsrechnung}
\label{subsec:approx}
Zunächst wird die in Abschnitt \ref{sub:berechnung_der_aktivierungsenergie_w_}
aufgeführte Gleichung \ref{eqn:approx} zur Bestimmung der Aktivierungsenergie
$W$ benutzt. Dabei wird diese mit einem nicht-linearen Fit in einem Bereich
$T$ zwischen \SIrange{250}{267}{\kelvin} an die Daten angepasst.
Das Ergebnis des Fits ist in Abbildung \ref{fig:fit_approx_set2} dargestellt.
Der Fit liefert von Messreihe A liefert



das entsprechende Ergebnis aus Messreihe B liefert



Die zugehörigen Fits sind in Abbildungen \ref{fig:fit_approx_set1}
und \ref{fig:fit_approx_set2} dargestellt.

\subsection{Ausgleichsrechnung mit Integration}
\label{subsec:integration}
Als zweite Methode zur Bestimmung der Aktivierungsenergie $W$ wird ein
linearer Fit der Form
\begin{equation*}
    F(T) = \alpha\cdot\frac{1}{T} + \beta
\end{equation*}
an die inversen Temperaturdaten $1/T$ durchgeführt.
Hierbei ist $F(T)$ wie in Gleichung \ref{eqn:integrate} definiert als
\begin{equation*}
    F(T) = \frac{\int_T^{T^\ast} i(T^\prime)\mathrm{d}T^\prime}{i(T)}\,.
\end{equation*}
Aus der Steigung $A$ lässt sich somit die Aktivierungsenergie $W$
berechnen durch
\begin{equation*}
    W = \alpha\cdot k_\text{B}\,.
\end{equation*}
Außerdem wird schließlich die Relaxationszeit $\tau_0$ aus der Temperatur
bei maximalem Strom $T_\text{max}$ bestimmt:
\begin{equation}
    \label{eqn:tau0}
    \tau_0 = \frac{k_\text{B}T_\text{max}^2}{Wb}
             \exp\!\left[-\frac{W}{k_\text{B}T_\text{max}} \right]
\end{equation}
Die Fits sind in Abbildungen \ref{fig:integrate_fit1} und \ref{fig:integrate_fit2}
dargestellt, sie liefern



für Datensatz A in einem Bereich $T \in [\SI{260}{\kelvin}, \SI{310}{\kelvin}]$ bei  und
für Datensatz B in einem Bereich $T \in [\SI{250}{\kelvin}, \SI{280}{\kelvin}]$ bei .
Die Fehler dieser Werte berücksichtigen jedoch keine systematischen Unsicherheiten
und werden daher stark unterschätzt. Darauf wird in der Diskussion weiter eingegangen.
\newpage
\section{Diskussion}
\label{sec:diskussion}
Alle hier durchgeführten Methoden zur Bestimmung der Aktivierungsenergie $W$
liefern Werte im erwarteten Bereich von einigen Millielektronvolt.
Die verschiedenen Messreihen liefern unterscheiden sich jedoch um einen Faktor
 (approximative Methode) beziehungsweise
 (integrierende Methode), was
auf systematische Fehler hindeutet und mit weiteren Messungen untersucht
werden muss.

Die mit Hilfe der integrierenden Methode bestimmten Werte der Relaxationszeit
$\tau_0$ sind schließlich nicht signifikant.
Die Fehler berücksichtigen nicht die starke Abhängigkeit von $\tau_0$ von der
Aktivierungsenergie $W$, die in Abbildung \ref{fig:_tau_test} dargestellt ist.
Bei einer Änderung von $W$ um einige \SI{100}{\milli\electronvolt} ändert
sich der Wert der Relaxationszeit bereits um etliche Größenordnungen.
Änderungen in diesem Bereich treten bereits auf, wenn der Wertebereich des Fits
leicht verändert wird und einige Meßpunkte mehr oder weniger betrachtet werden.
Die damit verbundenen Fehler machen eine Aussage über den wahren Wert von
$\tau_0$ unmöglich.
Die Unsicherheit $\Delta\tau_0$ sollte daher wesentlich zu größeren Werten hin
korrigiert werden.