\section{Diskussion}
\label{sec:Diskussion}
Der Lieraturwert des Erdmagnetfeldes liegt innerhalb der Unischerheit des gemessenen Wertes.
Daraus kann gefolgert werden, dass das Erdmagnetfeld gut kompensiert wurde und der 
systematische Fehler dieses Versuchsteils vernachlässigbar klein ist.\\\\
Der Kernspin, welcher mit dem ersten Peak berechnet wurde ergibt den Wert $I=\num{1.62(21)}$ 
wärend der Kernspin, der mit dem zwieten Peak bestimmt wurde $I=\num{2.48(14)}$ beträgt.
Der Kernspin von $\ce{^85Rb}$ ist als $I_{85}=\num{2.5}$ bekannt und für 
$\ce{^87Rb}$ gilt $I_{87}=\num{1.5}$.
Daher wird der erste Peak mit $\ce{^87Rb}$ und der zweite Peak mit $\ce{^85Rb}$
identifiziert.
Auch bei diesem Auswertungsteil befinden sich die zugehörigen Literaturwerte 
innerhalb der Unsicherheit der bestimmten Ergebnisse.\\\\
Bei der Bestimmung des Verhältnisses zwischen der Anzahl von $\ce{^85Rb}$- zu $\ce{^87Rb}$-Isotopen
im verwendeten Gas, kam es zu einem Wert der geringer als die Verteilung des natürlichen 
Vorkommens ist. Aus diesem Ergebnis kann geschlossen werden, dass die Probe 
angereichert wurde.
\newpage
%In der Unsicherheit des gemessenen Erdmagnetfeldes liegt der Literaturwert. Daher kann angenommen werden, dass das Erdmagnetfeld
%in diesem Versuch gut kompensiert wurde. Der systematische Fehler der dabei entsteht ist daher klein.
%Da der Kernspin, welcher mit Peak 1 erzeugt wurde $I=\num{1.62(21)}$ und der des zweiten Peaks $I=\num{2.48(14)}$ beträgt 
%und der Kernspin von $\ce{^85Rb}$ $I_{85}=\num{2.5}$ und von $\ce{^87Rb}$ $I_{87}=\num{1.5}$ ist, kann Peak 2 mit 
%$\ce{^85Rb}$ und Peak 1 mit $\ce{^87Rb}$ identifiziert werden. 
%Alle Literaturwerte liegen auch hier im Bereich der Unsicherheiten.
%Das bestimmte Isotopenverhältnis von $\ce{^85Rb}$ zu $\ce{^87Rb}$ ist geringer als in der Natur, dass bedeutet, das die Probe 
%angereichert wurde.
