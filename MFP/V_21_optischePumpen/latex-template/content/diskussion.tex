\section{Diskussion}
\label{sec:Diskussion}
In der Unsicherheit des gemessenen Erdmagnetfelds liegt der Literaturwert. Daher kann angenommen werden, dass das Erdmagnetfeld
in diesem Versuch gut kompensiert wurde. Der systematische Fehler der dabei entsteht ist daher klein.
Da der Kernspin welcher mit Peak 1 erzeugt wurde $I=\num{1.62(21)}$ und der des zweiten Peaks $I=\num{2.48(14)}$ beträgt 
und der Kernspin von $\ce{^85Rb}$ $I_{85}=\num{2.5}$ und von $\ce{^87Rb}$ $I_{87}=\num{1.5}$ ist, kann Peak 2 mit 
dem $\ce{^85Rb}$ und Peak 1 mit dem $\ce{^87Rb}$ identifiziert werden. Alle Literaturwerte liegen auch hier im Bereich der Unsicherheiten.
Das Bestimmte Isotopenverhältnis von $\ce{^85Rb}$ zu $\ce{^87Rb}$ ist geringer als in der Natur, dass bedeutet das die Probe 
angereichert wurde.
