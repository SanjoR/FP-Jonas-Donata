\section{Auswertung}
\label{sec:Auswertung}
Bei diesem Versuch soll der Kernspin, die Landé-Faktoren und das Verhältnis von $\ce{^85Rb}$ und $\ce{^87Rb}$ ermittelt 
werden. Zusätzlich kann die Horizontalkomponente des Erdmagnetfelds bestimmt werden. Die dafür benötigten Daten sind in 
Tabelle \ref{tab:Messdaten_roh} aufgelistet.
\FloatBarrier
\begin{table}
  \centering
  \caption{Messdaten für die Bestimmung des Kernspins, die Landé-Faktoren und des Erdmagnetfelds.}
  \label{tab:Messdaten_roh}
  \begin{tabular}{c c c c c}
    \toprule
    Frequenz / $\SI{}{\kilo\hertz}$&Sweep Peak 1&Sweep Peak 2&Horizontal Peak 1&Horizontal Peak 2\\
    \midrule
    $\num{100}$&$\num{5.49}$&$\num{6.68}$&$\num{13.8}$&$\num{13.8}$\\
    $\num{200}$&$\num{6.01}$&$\num{8.36}$&$\num{13.86}$&$\num{13.86}$\\
    $\num{300}$&$\num{5.43}$&$\num{8.97}$&$\num{13.92}$&$\num{13.92}$\\
    $\num{400}$&$\num{3.79}$&$\num{8.47}$&$\num{14.02}$&$\num{14.02}$\\
    $\num{500}$&$\num{2.64}$&$\num{8.54}$&$\num{14.10}$&$\num{14.10}$\\
    $\num{600}$&$\num{1.86}$&$\num{8.98}$&$\num{14.26}$&$\num{14.26}$\\
    $\num{700}$&$\num{1.37}$&$\num{3.64}$&$\num{14.22}$&$\num{14.36}$\\
    $\num{800}$&$\num{3.76}$&$\num{3.73}$&$\num{14.36}$&$\num{14.44}$\\
    $\num{900}$&$\num{2.55}$&$\num{7.30}$&$\num{14.30}$&$\num{14.44}$\\
    $\num{1000}$&$\num{4.94}$&$\num{4.71}$&$\num{14.30}$&$\num{14.58}$\\
    \bottomrule
  \end{tabular}
\end{table} 
\FloatBarrier
Die Messdaten aus Tabelle \ref{tab:Messdaten_roh} müssen zunächst in eine Stromstärke umgerechnet werden. Die Sweepeinstellung 
muss mit dem Wert $\SI{0.1}{\ampere}$ und die Horizontaleinstellung mit $\SI{0.3}{\ampere}$ multipliziert werden um eine Stromstärke
zu erhalten. Bei der Horizontalkomponente ist ein Offset von $\num{13.8}$ vorhanden, daher muss dieser Wert von den Daten 
subtrahiert werden. Um aus den Stromstärken ein $B$-Wert zu erhalten muss die Formel 
\begin{equation*}
  B=\mu_0\frac{8\cdot N \cdot I}{\sqrt{125}R}
\end{equation*}
angewendet werden. Die Sweep- und Horizontalwerte müssen zum gesamt $B$-Wert addiert werden. Die Stromstärken und die 
$B$-Werte sind in Tabelle \ref{tab:Messdaten} zu sehen.
\begin{table}
  \centering
  \caption{Stromstärken und Magnetfeldstärke in abhängigkeit der Frequenz.}
  \label{tab:Messdaten}
  \begin{tabular}{c c c c c c c}
    \toprule
    &\multicolumn{3}{c}{Peak 1}&\multicolumn{3}{c}{Peak 2}\\
    \cmidrule(lr){2-4}\cmidrule(lr){5-7}
    $f\,/\,\SI{}{\kilo\hertz}$&$I_{\text{sweep}}\,/\,\SI{}{\ampere}$&$I_{\text{horizontal}}\,/\,\SI{}{\ampere}$&$B\,/\,\SI{}{\micro\tesla}$&$I_{\text{sweep}}\,/\,\SI{}{\ampere}$&$I_{\text{horizontal}}\,/\,\SI{}{\ampere}$&$B\,/\,\SI{}{\micro\tesla}$\\
    \midrule
    $\num{100}$ &$\num{0.549}$&$\num{0.0}$  &$\num{33.1}$&$\num{0.668}$&$\num{0.0}$&$\num{40.3}$\\
    $\num{200}$ &$\num{0.601}$&$\num{0.018}$&$\num{52.1}$&$\num{0.836}$&$\num{0.018}$&$\num{66.2}$\\
    $\num{300}$ &$\num{0.543}$&$\num{0.036}$&$\num{64.3}$&$\num{0.897}$&$\num{0.036}$&$\num{85.7}$\\
    $\num{400}$ &$\num{0.379}$&$\num{0.066}$&$\num{80.8}$&$\num{0.847}$&$\num{0.066}$&$\num{109.0}$\\
    $\num{500}$ &$\num{0.264}$&$\num{0.089}$&$\num{94.9}$&$\num{0.854}$&$\num{0.09}$&$\num{130.5}$\\
    $\num{600}$ &$\num{0.186}$&$\num{0.138}$&$\num{132.2}$&$\num{0.898}$&$\num{0.138}$&$\num{175.2}$\\
    $\num{700}$ &$\num{0.137}$&$\num{0.126}$&$\num{118.8}$&$\num{0.364}$&$\num{0.168}$&$\num{169.3}$\\
    $\num{800}$ &$\num{0.376}$&$\num{0.168}$&$\num{170.0}$&$\num{0.373}$&$\num{0.192}$&$\num{190.9}$\\
    $\num{900}$ &$\num{0.255}$&$\num{0.15}$ &$\num{146.9}$&$\num{0.73}$&$\num{0.192}$&$\num{212.4}$\\
    $\num{1000}$&$\num{0.494}$&$\num{0.15}$ &$\num{161.4}$&$\num{0.471}$&$\num{0.234}$&$\num{233.6}$\\
    \bottomrule
  \end{tabular}
\end{table}