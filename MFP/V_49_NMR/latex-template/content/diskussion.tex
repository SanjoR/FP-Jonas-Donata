\section{Diskussion}
\label{sec:Diskussion}
Da die Fehler die durch die Justage kommen, nicht bestimmt werden können, beinhalten die Messdaten 
einen systematischen Fehler, der nicht berücksichtigt werden kann. Das gleiche gilt für die Temperaturabhängigkeit 
der Messgrößen.
\subsection{Bestimmung der Relaxationszeiten}
Die Relaxationszeiten wurden auf die Werte 
\begin{align}
    T_{\text{1}}=& \SI{1.92(24)}{\second} \qquad T_{\text{1\_lit}}=& \SI{3.09(15)}{\second}\\
    T_{\text{2}}=& \SI{1.4(18)}{\second} \qquad T_{\text{2\_lit}}=& \SI{1.52(9)}{\second}
\end{align}
bestimmt.
Die dazugehörigen Literaturwerte kommen aus der Quelle \cite{Quelle_8}. Der $T_{\text{1}}$-Wert hat eine relative Abweichung 
von 38\% und der $T_{\text{2}}$-Wert hat eine von 8\%. Die Abweichungen können durch die oben genannten Fehler kommen, oder 
daher, dass die Relaxationszeiten auch von der Reinheit des Wassers abhängt und die in dem Versuch unterschiedlich von der 
Probe aus der literatur sein kann.
\subsection{Diffusionsmessung und Bestimmung des Molekülradius}
Der gemessene Diffusionskoeffizent liegt bei $D=\SI{6.6e-20}{\square\meter\per\second}$. Der Literaturwerte beträgt 
$D_{\text{lit}}= \SI{2.78(35)e-9}{\square\meter\per\second}$. Die Abweichung kann nicht durch Messfehler erklärt werden.
Diese könnte damit erklärt werden, dass der Fit mit dem $T_{\text{D}}$ bestimmt wurde nicht gut funktioniert hat oder 
das die Gradientenstärke falsch bestimmt wurde.
$D$ weicht um elf Größenordnungen ab, daher weicht auch der Molekülradius um zehn Größenordnungen ab, da dieser 
mit $D$ bestimmt wird.