\section{Zielsetzung}
\label{sec:Zielsetzung}

In diesem Versuch soll die Diffusionskonstante von Wasser experimentell bestimt werden,
indem eine Probe bezüglich ihrer Spinausrichtung unter einem Hochfrequenzfeld untersucht
wird. 

\section{Theorie}
\label{sec:Theorie}

\subsection{Magnetisierung}
\label{sec:Magnetisierung}

Wird ein äußeres Magnetfeld $\vec{B}_{\text{0}} = B_{\text{0}} \vec{e}_{\text{B}}$
um die Probe angelegt, führt dies zu einer
Aufspaltung der Kernspinzustände in $m = 2S+1$ Unterniveaus, wobei $S$ die Spinquantenzahl
und $m$ magnetische Quantenzahl darstellt. 
%Für Protonen ($S=\frac{1}{2}$) ergibt sich  unter
%der Annahme $m\gamma B_{\text{0}} \ll k_{\text{B}}T$ in linearer Näherung die 
%$z$-Komponente des Kernspins als 
%\begin{equation}
%    \left< I_{\text{z}} \right> = - \frac{\hbar^2 \gamma B_0}{4 k_{\text{B}} T}.
%    \label{eq1}
%\end{equation}
%Hierbei ist $\hbar$ das reduzierte Planck´sche Wirkungsquantum, $\gamma$ das 
%gyromagnetische Verhältnis des Kerns, $k_{B}$ die Boltzmannkonstante und $T$ die 
%Temperatur. 
%Für den Erwartungswert der makroskopischen Magnetisierung folgt dann 
%\begin{equation}
%    M_0 = \frac{1}{4} \mu_0 \gamma^2 N \frac{\hbar^2 B_0}{k_\text{B} T},
%    \label{eq2}
%\end{equation}
%wobei $\mu_{\text{0}}$ die permeabilität des Vakuums und $N$ die Anzahl der 
%magnetischen Momente pro Volumeneinheit beschreibt. 
Die Präzession der Momente um $\vec{B}_{\text{0}}$ kann mittels der Larmorfrequenz 
$\omega_{\text{L}}= \gamma B_{\text{0}}$
beschrieben werden.
Nach Abschalten von $\vec{B}_{\text{0}}$ bewegen sich die Momente wieder zurück in 
ihre Ausgangslagen (Relaxation). 
Dabei werden die Spin-Gitter-Relaxationszeit $T_{\text{1}}$ 
(parallel zu $\vec{B}_{\text{0}}$) und Spin-Spin-Relaxationszeit $T_{\text{2}}$
(senkrecht zu $\vec{B}_{\text{0}}$) unterschieden.
Zur Beschreibung der zeitlichen Entwicklung der Magnetisierung $\vec{M}$ einer Probe 
werden die Bloch-Gleichungen verwendet 
\begin{equation}
    \frac{d \vec{M}}{dt} 
    = \gamma \vec{M} \times \vec{B}_{\text{0}} 
    - \vec{e}_{\text{x}} \frac{M_{\text{x}}}{T_{\text{2}}} 
    - \vec{e}_{\text{y}} \frac{M_{\text{y}}}{T_{\text{2}}} 
    - \vec{e}_{\text{z}} \frac{M_{\text{z}}-M_{\text{0}}}{T_{\text{1}}}.
  \label{eq3}
\end{equation}

\subsection{Hochfrequenzfeld}
\label{sec:Hochfrequenzfeld}

Um die magnetischen Momente aus ihrer Ruhelage auszulenken wird ein Hochfrequenzfeld  
$\vec{B}_{\text{H}} = 2 \vec{B}_{\text{1}} \cos{\omega t}$ ($\vec{B}_{\text{1}} 
\perp \vec{e}_{\text{B}}$) anglegt.
%sodass die Komponenten durch 
%\begin{align}
%    B_{\text{x}} = B_{\text{1}} \cos{\omega t} \qquad B_{\text{y}} = B_{\text{1}} \sin{\omega t}
%    \label{eq5}
%\end{align}
%gegeben sind. 
Die Lösung der Bloch-Gleicheungen vereinfacht sich durch Transformation 
in ein mit der Frequenz $\omega$ um $\vec{B}_{\text{0}}$ rotierendes 
Koordinatensystem. $\sphericalangle\!\left(\vec{M}, \vec{B}_1\right) = \SI{90}{\degree}$
Das äußere Magnetfeld transformiert sich dann zu 
$\vec{B}_\text{eff} = \vec{B}_0 + \vec{B}_1 + \frac{\vec{\omega}}{\gamma}$.
Das Einschalten eines Hochfrequenzfeldes für die Zeit 
\begin{equation}
    \Delta t_{\text{90}} = \frac{\pi}{2 \gamma B_{\text{1}}}
    \label{eq4}
\end{equation}
während $\vec{B}_{\text{eff}} = \vec{B}_{\text{1}}$ und 
$\sphericalangle\!\left(\vec{M}, \vec{B}_1\right) = \SI{90}{\degree}$ gilt, 
realisiert eine Drehung der Magnetierung von der $z$- in die $y$-Richtung
(90°-Puls). Analoges gilt für einen 180°-Puls.


\subsection{freier Induktionszerfall}
\label{sec:Induktionszerfall}


\cite{sample}
