\section{Zielsetzung}
\label{sec:Zielsetzung}

In diesem Versuch soll die Diffusionskonstante von Wasser experimentell bestimmt werden,
indem eine Probe bezüglich ihrer Spinausrichtung unter einem Hochfrequenzfeld untersucht
wird. 

\section{Theorie}
\label{sec:Theorie}

\subsection{Magnetisierung}
\label{sec:Magnetisierung}

Wird ein äußeres Magnetfeld $\vec{B}_{\text{0}} = B_{\text{0}} \vec{e}_{\text{B}}$
um die Probe angelegt, führt dies zu einer
Aufspaltung der Kernspinzustände in $m = 2S+1$ Unterniveaus, wobei $S$ die Spinquantenzahl
und $m$ die magnetische Quantenzahl darstellt. 
Die Präzession der Momente um $\vec{B}_{\text{0}}$ kann mittels der Larmorfrequenz 
$\omega_{\text{L}}= \gamma B_{\text{0}}$
beschrieben werden.
Nach Abschalten von $\vec{B}_{\text{0}}$ bewegen sich die Momente wieder zurück in 
ihre Ausgangslagen (Relaxation). 
Dabei werden die Spin-Gitter-Relaxationszeit $T_{\text{1}}$ 
(parallel zu $\vec{B}_{\text{0}}$) und Spin-Spin-Relaxationszeit $T_{\text{2}}$
(senkrecht zu $\vec{B}_{\text{0}}$) unterschieden.
Zur Beschreibung der zeitlichen Entwicklung der Magnetisierung $\vec{M}$ einer Probe 
werden die Bloch-Gleichungen verwendet 
\begin{equation}
    \frac{d \vec{M}}{dt} 
    = \gamma \vec{M} \times \vec{B}_{\text{0}} 
    - \vec{e}_{\text{x}} \frac{M_{\text{x}}}{T_{\text{2}}} 
    - \vec{e}_{\text{y}} \frac{M_{\text{y}}}{T_{\text{2}}} 
    - \vec{e}_{\text{z}} \frac{M_{\text{z}}-M_{\text{0}}}{T_{\text{1}}}.
  \label{eq3}
\end{equation}

\subsection{Hochfrequenzfeld}
\label{sec:Hochfrequenzfeld}

Um die magnetischen Momente aus ihrer Ruhelage auszulenken wird ein Hochfrequenzfeld  
$\vec{B}_{\text{H}} = 2 \vec{B}_{\text{1}} \cos{(\omega t)}$ ($\vec{B}_{\text{1}} 
\perp \vec{e}_{\text{B}}$) angelegt.
Die Lösung der Bloch-Gleicheungen vereinfacht sich durch Transformation 
in ein mit der Frequenz $\omega$ um $\vec{B}_{\text{0}}$ rotierendes 
Koordinatensystem. 
Das äußere Magnetfeld transformiert sich dann zu 
$\vec{B}_{\text{eff}} = \vec{B}_{\text{0}} + \vec{B}_{\text{1}} + \frac{\vec{\omega}}{\gamma}$.
Das Einschalten eines Hochfrequenzfeldes für die Zeit 
\begin{equation}
    \Delta t_{\text{90}} = \frac{\pi}{2 \gamma B_{\text{1}}}
    \label{eq4}
\end{equation}
während $\vec{B}_{\text{eff}} = \vec{B}_{\text{1}}$ und 
$\sphericalangle\!\left(\vec{M}, \vec{B}_1\right) = \SI{90}{\degree}$ gilt, 
realisiert eine Drehung der Magnetisierung von der $z$- in die $y$-Richtung
(90°-Puls). Analoges gilt für einen 180°-Puls ( von $z$- in $-z$-Richtung).
Unter dem freien Induktionszerfall (FID) wird die Präzession der Magnetisierung 
senkrecht zu $\vec{B}_{\text{0}}$ und die anschließende Relaxation zurück in die 
Ruhelage nach einer Auslenkung um 90° verstanden. Ursächlich für den FID ist die 
Inhomogenität eines real erzeugten Magnetfeldes mit der 
Relaxationszeit $T_{\Delta \text{B}}$ und die
Wechselwirkung mit den magnetischen Momenten der Nachbarkerne und -schalen. 
Dadurch werden die Spins schließlich dephasiert und die transversale Magentisierung 
zerfällt. Die Gesamtrelaxationszeit ergibt sich als
\begin{equation}
    T^*_{2} = \frac{1}{T^{-1}_{2} + T^{-1}_{\text{\Delta B}}}.
\end{equation}

\subsection{Spin-Echo, Carr-Purcell und Meiboom-Gill Methode}
\label{sec:SpinEchoVerfahren}

Während der Dephasierung nach einem initialen 90°-Puls
rotieren die Spins mit größerer Larmourfrequenz im Uhrzeigersinn.
Die Spins mit kleinerer Larmorfrequenz jedoch in die entgegengesetzte Richtung.
Nach der Zeit $\tau \ll T_{\text{\Delta B}}$ wird ein 180°-Puls auf die Probe 
gerichtet, sodass die Spins wieder aufeinander zulaufen. Es entsteht das 
sogenannte Hahn-Echo, d.h. eine transversale Magnetisierung in
entgegengesetzter Richtung nach der Zeit $2\tau$. 
Die Höhe des Spin-Echos kann über den Zusammenhang 
\begin{equation}
    M_{\text{y}}(t) = M_{\text{0}} \exp \left( - \frac{t}{T_{\text{2}}} \right)
    \label{eq5}
\end{equation}
bestimmt werden.
Um mehrere Echos in einer Messung aufnehmen zu können, werden in der Carr-Purcell 
Methode (CPM) mehrere 180°-Pulse hintereinander mit Abstand $2\tau$ gesendet, sodass 
eine Fokussierung der transversalen Magnetisierung erzeugt wird. Ist die Einstellung 
der 180°-Pulse zu ungenau, wird die Magnetisierung in der $x$-$y$-Ebene zu klein,
weil die Spins nicht genau in diese Ebene hinein gedreht werden, sodass eine 
aussagekräftige Bestimmung von $T_{\text{2}}$ nicht möglich ist.
Die Meiboom-Gill Methode (MGM) umgeht diese Problematik, indem sie die 180°-Pulse
gegen die initialen 90°-Pulse um 90° phasenverschoben aussendet. Das führt zu einer 
Drehung der Spins um die $y$-Achse anstelle der $x$-Achse , sodass Fehler aufgrund 
von ugenauen 180°-Pulseinstellungen durch den folgenden PPuls wieder 
ausgeglichen werden können.

\subsection{Spinrelaxation einer Flüssigkeit}
\label{sec:Spinrelaxation}

Für Flüssigkeiten müssen die Bloch´schen Gleichungen durch einen Diffusionsterm ergänzt 
werden, da die Brown´sche Molekularbewegung ein zeitlich veränderliches 
Magnetfeld erzeugt. Sie sind dann durch  
\begin{equation}
  \frac{\partial M}{\partial t} =
  \underbrace{\gamma \left(\vec{M} \times \vec{B}\right)}_{\text{Präzession}}
  \underbrace{- \frac{M_\text{x} \vec{x} + M_\text{y} \vec{y}}{T_2}
  - \frac{\left(M_\text{z} - M_0\right) \vec{z}}{T_1}}_{\text{Relaxation}}
  \underbrace{+ \left(\vec{x} + \vec{y} + \vec{z}\right) D\:\Delta M}_{\text{Diffusion}}
  \label{eq6}
\end{equation}
gegeben. Aufgrund dieser Bedingungen muss Geichung \ref{eq5} angepasst werden und 
unter der Annahme eines innerhalb der Probe konstanten Feldgradienten $G$ ergibt  
sich \ref{eq5} zu
\begin{equation}
  M_\text{y}\!\left(t\right) = M_0
  \exp\!\left(- \frac{t}{T_2}\right)
  \exp\!\left(- \frac{t}{T_\text{D}}\right)
  \label{eq7}
\end{equation}
wobei $T_{\text{D}} = \frac{3}{D \gamma^2 G^2 \tau^2}$ mit der Diffusionskonstante $D$.
CPM und MGM können für $T_{\text{D}} \gg T_{\text{2}}$ weiterhin verwendet werden, 
wogegen bei der Betrachtung der Amplitude des ersten Spin-Echso beachtet werden 
muss, dass $t=2\tau$ gilt.
Zur Bestimmung von $T_{\text{1}}$ wird zunächst ein 180°-Puls (Spinrichtung: 
$\vec{z} \rightarrow -\vec{z}$) und nach $\tau$ ein 90°-Puls ausgesendet, 
woraufhin die Spins in der $x$-$y-$Ebene präzidieren. Für die Magnetisierung in 
$\vec{z}$-Richtung gilt dann
\begin{equation}
  M_\text{z}\!\left(\tau\right) =
  M_0 \left(1 - 2 \exp\!\left[-\frac{\tau}{T_1}\right]\right).
  \label{eq8}
\end{equation}

\cite{theo1}
