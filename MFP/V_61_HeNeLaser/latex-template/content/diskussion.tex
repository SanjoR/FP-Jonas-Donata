\section{Diskussion}
\label{sec:Diskussion}

\subsection{Stabilitätbedingung}
Die Stabilitätbedingung für die Spiegelkonstellation mit einem flachen und einem konkaven Spiegel, konnte 
nicht bis zu dem Theoriewert überprüft werden. Das kann daran liegen, dass diese Konstellation schwierig zu 
justieren ist. Daher ist der Laserstrahl auch bei geringen Abständen zusammengebrochen, sobald die Justierschrauben 
berührt wurden. Die Konstellation mit den zwei konkaven Spiegeln ist einfacher zu justieren, sodass hier die Stabilitätbedingung
soweit überprüft werden konnte, wie es der Versuchsaufbau zulässt.

\subsection{Multimodenbetrieb}
Die Wellenlänge eines Lasers wird durch das aktive Medium bestimmt. Durch den Dopplereffekt kann die Wellenlänge variieren.
Mit der mittleren Geschwindigkeit der Neon-Atome, welche aus der Maxwell-Boltzmann-Verteilung bestimmt werden kann, kann 
die Breite des optischen Übergangs bestimmt werden. Bei einem Helium-Neon-Laser beträgt die Breite des optischen Übergangs $\SI{0.002}{\nano\meter}$.
Mit einer Wellenlänge von $\lambda_{\text{lit}}=\SI{633}{\nano\meter}$ 
\cite{leifi} kann eine Bandbreite von $\SI{1.5}{\giga\hertz}$ berechnet werden. In diesem Bereich kommt es zu Schwebungen,
welche als Moden bezeichnet werden. Der Abstand der Moden kann über die Funktion $f=\frac{c}{2d}$ bestimmt werden. Das bedeutet, 
dass bei einem größeren Abstand mehr Moden gemessen werden können. Über die Differenz der ersten und letzten Mode könnte die 
Breite des optischen Übergang bestimmt werden. Bei einer Länge von $\SI{1}{\meter}$ ist der Abstand der Moden $\SI{0.15}{\giga\hertz}$
und es müssten 10 Moden gemessen werden können. Da aber zum Beispiel bei einer Länge von $\SI{1.763}{\meter}$ nur sieben Moden gemessen werden konnten,
sind die Frequenzdifferenzen zu niedrig um die Breite des optischen Übergangs zu bestimmen. Allerdings kann gezeigt werden, dass
die gemessenen Moden einen Abstand von $\frac{c}{2d}$ aufweisen. Hierbei sind die Unsicherheiten der Messwerte unterschätzt, 
da diese nur durch die Fitfunktion erzeugt werden. Die Unsicherheiten von der Längen- oder Frequenzmessung werden 
nicht berücksichtigt.

\subsection{TEM-Moden}
Die Messung der TEM-Moden hat nicht gut funktioniert, da zu wenig Daten aufgenommen wurden, sodass Abweichungen 
von der theoretisch erwarteten Kurve zu sehen sind. Die Abweichungen könnten mit mehr Statistik geringer werden. 
Die $\text{TEM}_{00}$-Mode lässt sich mit einem Gaußfit beschreiben, dies folgt auch aus der Theorie. Die 
$\text{TEM}_{10}$-Mode wird mit einem Hermite-Polynom ersten Grades und einer Gaußfunktion beschrieben, welche beide 
quadriert werden. So eine Funktion lässt sich auch an die Daten anpassen. Bei den Daten ist zu sehen, dass rechts 
vom Mittelpunkt höhere Intensitäten gemessen wurden als links vom Mittelpunkt. Ein Grund dafür könnte eine nicht 
exakt vertikale Ausrichtung des Drahtes sein. 

\subsection{Polarisation}
Die gemessene Verteilung der Polarisation stimmt mit der Theorie überein. Da das Intensitätsmaximum bei $\varphi=\SI{1.5}{\radian}\approx\SI{86}{\degree}$
liegt, muss der Laserstrahl nahezu parallel zum Tisch polarisiert gewesen sein.

\subsection{Wellenlänge}
Die gemessene Wellenlänge liegt bei $\lambda = \SI{652(7)}{\nano\meter}$. Der Literaturwert liegt bei $\lambda_{\text{lit}}=\SI{633}{\nano\meter}$ 
\cite{leifi}.
Die Wellenlänge die nur mit Gitter A bestimmt wurde, liegt näher an dem Literaturwert, da mehr Maxima bestimmt werden konnten. 
Der Literaturwert liegt zwar nicht in den Unsicherheiten der gemessenen Werte, jedoch sind die Messwerte in der richtigen 
Größenordnung. 
\newpage