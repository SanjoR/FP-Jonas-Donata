\section{Diskussion}
\label{sec:Diskussion}

\subsection{Stabilitätbedingung}
Die Stabilitätbedingung für die Spiegelkonstellation mit einem flachen und einem konkaven Spiegel, konnte 
nicht bis zu dem Theoriewert überprüft werden. Das kann daran liegen, dass diese Konstellation schwierig zu 
justieren ist. Daher ist der Laserstrahl auch bei geringen Abständen zusammengebrochen, sobald die Justierschrauben 
berührt wurden. Die Konstellation mit den zwei konkaven Spiegeln ist einfacher zu justieren, sodass hier die Stabilitätbedingung
soweit überprüft werden konnte, wie es der Versuchsaufbau zulässt.

\subsection{Multimodenbetrieb}
Bei diesem Versuchsteil soll die Schwebung der Intensität untersucht werden. Da die Steigungen der Geraden, welche an die 
Messdaten angepasst wurden, in der gleichen Größenordnung wie die theoretisch bestimmten Geraden liegt, kann damit die 
Theorie bestätigt werden. Dass die Theoriewerte nicht in den Unsicherheiten der Steigungen liegen, kann daran liegen, dass
die Unsicherheiten nur durch die Fitfunktion erzeugt werden. Die Unsicherheiten von der Längen- oder Frequenzmessung werden 
nicht berücksichtigt, daher sind die Unsicherheiten der Steigung unterschätzt.

\subsection{TEM-Moden}
Die Messung der TEM-Moden hat nicht gut funktioniert, da zu wenig Daten aufgenommen wurden, sodass Abweichungen 
von der theoretisch erwarteten Kurve zu sehen sind. Die Abweichungen könnten mit mehr Statistik geringer werden. 
Die $\text{TEM}_{00}$-Mode lässt sich mit einem Gaußfit beschreiben, dies folgt auch aus der Theorie. Die 
$\text{TEM}_{10}$-Mode müsste sich mit einem Hermite-Polynom der ersten Ordnung multipliziert mit einer Gaußfunktion 
beschreiben lassen. Durch die unterschiedlichen Maximalintensitäten links und rechts vom Mittelpunkt, lässt sich 
eine entsprechende Funktion nicht an die Daten anpassen. Daher wurden an die beiden Maxima jeweils ein Gaußfit angepasst und die 
Superposition gebildet. Da die Parameter der Gaußfunktion in der gleichen Größenordnung liegen und nur die Amplitude 
abweicht, bestätigt das die Theorie. Nur das rechts vom Minimum generell höhere Intensitäten gemessen wurden, was auf einen systematischen
Fehler hinweist.

\subsection{Polarisation}
Die gemessene Verteilung der Polarisation stimmt mit der Theorie überein. Da das Intensitätsmaximum bei $\varphi=\SI{1.5}{\radian}\approx\SI{86}{\degree}$
liegt, muss der Laserstrahl nahezu parallel zum Tisch polarisiert gewesen sein.

\subsection{Wellenlänge}
Die gemessene Wellenlänge liegt bei $\lambda = \SI{652(7)}{\nano\meter}$. Der Literaturwert liegt bei $\lambda_{\text{lit}}=\SI{633}{\nano\meter}$ 
\cite{leifi}.
Die Wellenlänge die nur mit Gitter A bestimmt wurde, liegt näher an dem Literaturwert, da mehr Maxima bestimmt werden konnten. 
Der Literaturwert liegt zwar nicht in den Unsicherheiten der gemessenen Werte, jedoch sind die Messwerte in der richtigen 
Größenordnung. 
\newpage