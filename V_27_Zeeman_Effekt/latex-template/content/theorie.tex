\section{Motivation}
\label{sec:Motivation}
Ziel dieses Versuches ist die Untersuchung des Zeeman-Effektes. 
Er beschreibt die Aufspaltung und Polarisation von Energieniveaus in Form von 
Spektrallinien unter dem Einfluss eines Magnetfeldes.
In diesem Versuch wird die Aufspaltung für rote und blaue Linien von 
Cadmium Atomen gemessen.

\section{Theorie}
\label{sec:Theorie}

Neben dem Eigendrehimpuls des Atomkerns (Kernspin), der in dieser Betrachtung jedoch 
vernachlässigt wird, müssen der Gesamtbahndrehimpuls $\vec{L}$ und der 
Gesamtspin $\vec{S}$ der Hüllenelektronen betrachtet werden.
Beide setzen sich additiv aus den Einzelspins $\vec{s}$ und -bahndrehimpulsen 
$\vec{l}$ der Hüllenelektronen zusammen. 
Für ihre Beträge gilt der Zusammenhang:

\begin{align}
    |\vec{L}| = \sqrt{L(L+1)}\hbar,\\
    |\vec{S}| = \sqrt{S(S+1)}\hbar.
    \label{eq0}
\end{align}

Die Beträge dieser EInzeldrehimpulse 
ergeben sich über die zugehörigen Bahndrehimpulsquanenzahlen $l$ und 
Spinquantenzahlen $s$:

\begin{align}
    |\vec{l}| = \sqrt{l(l+1)}\hbar,\\
    |\vec{s}| = \sqrt{s(s+1)}\hbar.
    \label{eq1}
\end{align}

Die Bahndrehimpulsquantenzahl ist abhängig von der Hauptquantenzahl $n$, also der 
Zahl des Hauptenergieniveaus eines Elektrons. $l$ kann die Werte 0 bis $n-1$
annehmen. Darüber lassen sich auch die zu den Drehimpulsen gehörigen magnetischen 
Momente definieren:

\begin{align}
    \vec{\mu}_l = - \frac{\mu_B}{\hbar} \vec{l} = - \mu_B \sqrt{l(l+1)} \vec{l}_e,\\
    \vec{\mu}_s = - g_s\frac{\mu_B}{\hbar} \vec{s} = - g_s\mu_B \sqrt{s(s+1)} \vec{s}_e.
    \label{eq2}
\end{align}

In Gleichung \ref{eq2} entspricht $\mu_B$ dem magnetischen Moment eines Elektrons. 
Dieses magnetische Moment wird auch Bohrsches Magneton genannt:

\begin{equation}
    \mu_B = - \frac{1}{2}e_0\frac{\hbar}{m_0}.
    \label{eq3}
\end{equation}

In der Gleichung zum Bohrschen Magneton \ref{eq3} entspricht $e_0$ der 
Elementarladung, $m_0$ der Masse des Elektrons und $g_s$ dem Landé-Faktor
des Elektrons, welcher unter Berücksichtigung relativistischer Korrekturen 
ungefähr den Wert 2 annimmt.
Der Gesamtdrehimpuls $\vec{J}$ der Elektronenhülle ergibt sich durch die 
Kopplung des Gesamtspins und des Gesamtdrehimpulses:

\begin{equation}
    \vec{J} = \vec{L} + \vec{S}.
    \label{eq4}
\end{equation}

Der Betrag des Gesamtdrehimppulses ergibt sich dann äquivalent zu:

\begin{equation}
    |\vec{J}| = \sqrt{J(J+1)}\hbar
    \label{eq5}
\end{equation}

und sein magnetisches Moment wird aus der Superposition der magnetischen Momente des
Bahndrehimpulses und des Spins zusammengesetzt:

\begin{equation}
    \vec{\mu}_J = \vec{\mu}_L + \vec{\mu}_S.
    \label{eq6}
\end{equation}

Der Betrag dieses Momentes ist zu einem Landé-Faktor proportional:

\begin{equation}
    |\vec{\mu}_J| = g_J \mu_B\sqrt{J(J+1)}.
    \label{eq7}
\end{equation}

Das Prinzip der Richtungsquantelung besagt nun, dass ein äußerlich angelegtes 
Magnetfeld dazu führt, dass die $z$-Komponente des magnetischen Momentes des 
Gesamtdrehimpulses ein ganzzahliges Vielfaches vom Produkt zwischen 
Bohrschem Magneton und Landé-Faktor sein muss:

\begin{equation}
    \mu_{J_z} = -m g_J \mu_B.
    \label{eq8}
\end{equation}

Die Variable $m$ aus Gleichung \ref{eq8} wird orientierungsquantenzahl genannt und 
die Werte $m = -J, -J+1, ... , J-1,J$ annehmen kann. Für die Energie $E$ eines 
magnetischen Momentes in einem äußeren Magnetfeld gilt:

\begin{equation}
    E = - \vec{\mu}_J \cdot \vec{B} = m g_j \mu_B B.
    \label{eq9}
\end{equation}

Das bedeutet, dass die Energie $2J+1$ verschiedene, zueinander 
äquidistante Werte annehmen kann. 
Hierbei wird dann von der Aufspaltung der Energieniveuas gesprochen. 
Diese führt dann zu einer Aufspaltung der Spektrallinien.
Dieser Effekt wird Zeeman-Effekt genannt.
Es wird zwischen dem normalen und dem anormalen Zeeman-Effekt unterschieden.
Beim normalen Zeeman-Effekt verschwindet der Gesamtspind der Elektronenhülle.
Dadurch hat der Landé-Faktor des Gesamtdrehimpulses den Wert 1 und 
die Verschiebung der Energieniveaus ergibt sich zu:

\begin{equation}
    \delta E = m \mu_B B.
    \label{eq10}
\end{equation}

Die Aufspaltung einer Spektrallinie ergibt sich in drei Unterlinien,
die nach der Art der Polarisation charakterisiert werden. 
Die $\sigma$-Linien werden mit zirkularer Polarisation und die $\pi$-Linie 
mit liniearer Polarisation verknüpft. Lineare Polarisation entspricht einer Polarisation
parallel zum Magnetfeld sodass die $\pi$-Linie bei $\delta m = 0$ auftritt.
Die $\sigma_-$-Linie tritt bei $\delta m = -1$ und die $\sigma_+$-Linie entprechend 
bei $\delta m = +1$.
Beim anormalen Zeeman-Effekt spielt der Spin mit in die Energieaufspaltung mit 
ein, was zu einer viel feineren Spektrallinienaufspaltung führt. 
Die Verschiebung der Energieniveaus ergibt sich für den annormalen Zeeman-Effekt 
daher zu:

\begin{equation}
    \delta E = (m_j -m_i)g_j \mu_B B + E_0
    \label{eq11}
\end{equation}

In Gleichung \eqref{eq11} stellen $m_j$ und $m_i$ zwei benachbarte Orientierungsquantenzahlen
dar je zwei benachbarte Energieniveaus charakterisieren, zwischen denen ein Übergang 
stattfinden kann.